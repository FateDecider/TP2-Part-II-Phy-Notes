\documentclass[a4paper]{article}
%% Language and font encodings
\usepackage[english]{babel}
\usepackage[utf8x]{inputenc}
\usepackage[T1]{fontenc}
\usepackage{float}
%% Sets page size and margins
\usepackage[a4paper,top=3cm,bottom=2cm,left=3cm,right=3cm,marginparwidth=1.75cm]{geometry}

%% Useful packages
\usepackage{tensor}
\usepackage{tikz}
\usepackage{fancyhdr}
\pagestyle{fancy}
\usepackage{amsmath}
\usepackage{amstext}
\usepackage{amsthm}
\usepackage{enumitem}
\usepackage{eqnarray}
\usepackage{float}
\usepackage{esint}
\usepackage{wrapfig}
\usepackage{gensymb}
\usepackage{lipsum}
\usepackage{amssymb}
\usepackage{array}
\usepackage{tikz}
\usetikzlibrary{arrows,decorations.markings}
\usepackage[colorlinks=true, allcolors=blue]{hyperref}
\usepackage{graphicx}
\usepackage{amsmath}
\usepackage{amssymb}

\usepackage{graphicx}
\usepackage{mathtools}
\usepackage[colorlinks=true, allcolors=blue]{hyperref}
\DeclareMathOperator{\Id}{Id}
\DeclareMathOperator{\Sym}{Sym}
\DeclareMathOperator{\Auto}{Aut}
\DeclareMathOperator{\lcm}{lcm}
\DeclareMathOperator{\Tr}{Tr}
\DeclareMathOperator{\R}{Im}
\DeclareMathOperator{\Ker}{Ker}
\DeclareMathOperator{\sech}{sech}
\DeclareMathOperator{\diag}{diag}
\DeclareMathOperator{\sgn}{sgn}
\DeclareMathOperator{\Mod}{mod}
\DeclareMathOperator{\cl}{cl}
\newcommand{\iso}{\xrightarrow{
   \,\smash{\raisebox{-0.65ex}{\ensuremath{\scriptstyle\sim}}}\,}}
\newtheorem{post}{Postulate}[section]
\newtheorem{eg}{Example}[section]
\newtheorem{remarks}{Remarks}[section]
\newtheorem{notation}{Notation}[section]
\newtheorem{Note}{Note}[section]
\definecolor{darkblue}{RGB}{	0, 0, 139}
\newtheoremstyle{new}% <name>
{2pt}% <Space above>
{2pt}% <Space below>
{\color{darkblue}}% Body font
{}% <Indent amount>
{\bfseries\color{black}}% Theorem head font
{:}% <Punctuation after theorem head>
{.5em}% <Space after theorem headi>
{}% <Theorem head spec (can be left empty, meaning `normal')>
\theoremstyle{new}
\newtheorem{qns}{Problems}[section]
\newtheorem{ans}{Answers}[section]
\newtheorem{law}{Law}[section]
\newtheorem{defi}{Definition}[section]
\newtheorem{thm}{Theorem}[section]
\newtheorem{prop}{Proposition}[section]
\newtheorem{lemma}{Lemma}[section]
\newtheorem{cor}{Corollary}[section]

\title{\textbf{TP2 (Theoretical Physics 2) Part II Phy}}
\author{Tai Yingzhe, Tommy (ytt26)}
\date{}
\setlength{\parindent}{0cm}
\begin{document}
\maketitle
\usetikzlibrary{decorations.markings}
\tikzset{->-/.style={decoration={
  markings,
  mark=at position #1 with {\arrow{>}}},postaction={decorate}}}
\tikzset{-<-/.style={decoration={
  markings,
  mark=at position #1 with {\arrow{<}}},postaction={decorate}}}
\tableofcontents
\subsection*{Acknowledgements:}
Many thanks to my demonstrators Nur \"{U}nal, Benjamin Remez and Luca Donini, and the lecturer Nigel Cooper for their guidance. Some parts of this notes are heavily influenced by the Part II DAMTP courses Applications of Quantum Mechanics, and Principles of Quantum Mechanics.
\newpage
\section{Quantum Dynamics}
\subsection{Three Pictures}
\begin{defi}[Heisenberg's picture versus Schr\"{o}dinger's picture]
Instead of working with time-dependent states, we can notice
\begin{equation}
\langle\psi(t)|\mathbf{Q_S}|\psi(t)\rangle=\langle\psi(0)|U^{-1}(t)\mathbf{Q_S}U(t)|\psi(0)\rangle\label{Heisenberg}
\end{equation}
and hence work with $|\psi(0)\rangle$ using time-dependent operators $\mathbf{Q_H}(t)=U^{-1}(t)\mathbf{Q_S}U(t)$ via conjugation. This is known as the Heisenberg's picture, where states are time-independent but operators vary with time. This is in contrast with the Schr\"{o}dinger's picture (states change by time-dependent Schr\"{o}dinger's equation but operators are time-independent).
\end{defi}
\begin{thm}\leavevmode
\begin{itemize}
    \item $A_H(t=0)=A_S$
    \item $C_S=A_SB_S\implies C_H(t)=A_H(t)B_H(t)$
    \item $[A_S,B_S]=C_S\implies[A_H(t),B_H(t)]=C_H(t)$
    \item $\langle A_H(t)\rangle=\langle A_S\rangle$
\end{itemize}
\end{thm}
\begin{thm}[Heisenberg's equation of motion]
The rate of change of a time-dependent Heisenberg operator $Q_H(t)$ is
\begin{equation}
\frac{d}{dt}Q_H(t)=\frac{i}{\hbar}[H,Q_H(t)]+U^{-1}\frac{\partial Q_S}{\partial t}U(t)\label{EoM}
\end{equation}
where $Q_S$ is the corresponding Schr\"{o}dinger's operator.
\end{thm}
\begin{proof}
Differentiating $Q_H(t)$ (Eqn.~\ref{Heisenberg}) with respect to time
$$\frac{d}{dt}Q_H(t)=\frac{d}{dt}(U^{-1}(t)Q_SU(t))=\frac{i}{\hbar}U^{-1}(t)[H,Q_S]U(t)+U(t)^{-1}\frac{\partial Q_S}{\partial t}U(t)=\frac{i}{\hbar}[H,Q_S(t)]+U(t)^{-1}\frac{\partial Q_S}{\partial t}U(t)$$
where we used the time-dependent Schr\"{o}dinger's equation and that since $U(t)$ depend only on $H$ $\implies[U(t),H]=0$.
\end{proof}
\begin{eg}
The creation and annihilation operators satisfy the Heisenberg's equation of motion:
$$\frac{d}{dt}a_H(t)=-\frac{i}{\hbar}[a_H(t),H]=-\frac{i}{\hbar}U^{-1}[a_S,H]U=-i\omega a_H(t)$$
where $[a_S,a_S^\dag]=1\implies[a_S,a_S^\dag a_S]=a_S$, which has the solution $a_H(t)=e^{-i\omega t}a_S$ since $a_S=a_H(t=0)$. If the energy of the system is reduced by a single excitation at $t$, the phase accrued during forward propagation is greater than that during backward propagation, regardless of the original state.
\end{eg}
\begin{notation}
We will use $a_H=a(t)$ and $a_S=a$ interchangably. 
\end{notation}
If $H(t)=H_0+\Delta(t)$ where $H_0$ is independent of time (and can be solved) and $\Delta$ is time-dependent even in the Schr\"{o}dinger's picture. 
\begin{defi}[Interaction picture state]
Define the interaction picture state
\begin{equation}
|\Psi_I(t)\rangle=U_0^{-1}(t)|\Psi_S(t)\rangle,\quad U_0(t)=e^{-iH_0t/\hbar}\label{interactionpicture}
\end{equation}
is the time evolution operator just for $H_0$, $|\Psi_S(t)\rangle$ is the Schr\"{o}dinger picture state for the full TDSE using $H(t)=H_0+\Delta(t)$.
\end{defi}
\begin{prop}
The Schr\"{o}dinger's equation in the interaction picture is
\begin{equation}
i\hbar\frac{\partial}{\partial t}|\Psi_I(t)\rangle=U_0^{-1}(t)\Delta(t)U_0(t)|\Psi_I(t)\rangle\label{interactionpicture2}
\end{equation}
\end{prop}
\begin{proof}
In the absence of $\Delta(t)$, the Schr\"{o}dinger state evolves via  $|\Psi_S(t)\rangle=U_0(t)|\Psi_S(0)\rangle$ such that by Eqn.~\ref{interactionpicture}, $|\Psi_I(t)\rangle_{\Delta=0}=|\Psi_S(0)\rangle$ is time-independent. Thus $|\Psi_I(t)\rangle$ evolves only due to the perturbation. 
\begin{align}
i\hbar\frac{\partial}{\partial t}|\Psi_I(t)\rangle&=-H_0e^{iH_0t/\hbar}|\Psi_S(t)\rangle+e^{iH_0t/\hbar}(H_0+\Delta(t))|\Psi_S(t)\rangle\nonumber\\&=\Delta(t)e^{iH_0t/\hbar}|\Psi_S(t)\rangle=U_0^{-1}(t)\Delta(t)U_0(t)|\Psi_I(t)\rangle\nonumber
\end{align}
where we took the time derivative.
\end{proof}
\begin{prop}
The interaction picture of an arbitrary operator is
\begin{equation}
A_I(t)=U_0^{-1}(t)A_SU_0(t)\label{interactionpicture3}
\end{equation}
while the corresponding Heisenberg's equation of motion gives
\begin{equation}
\frac{d}{dt}A_I(t)=\frac{i}{\hbar}[H_0,A_I(t)]+U_0^{-1}(t)\frac{\partial A_I(t)}{\partial t}U_0(t)\label{interactionpicture4}
\end{equation}
\end{prop}
\begin{proof}
Since the expectation value is independent of the picture, we have
$$\langle\Psi_S(t)|A|\Psi_S(t)\rangle=\langle\Psi_I(t)|U_0^{-1}(t)AU_0(t)|\Psi_I(t)\rangle\implies A_I(t)=U_0^{-1}(t)A_SU_0(t)$$
Take the time derivative to find the equation of motion.
\end{proof}
\begin{remarks}
Schr\"{o}dinger states evolve according to the Hamiltonian. Schr\"{o}dinger operators do not evolve unless they are explicitly time-dependent. Heisenberg states do not evolve but Heisenberg operators evolve like $A_H(t)=U^{-1}(t)A_SU(t)$. Interaction states evolve only through the time-dependent perturbation $\Delta(t):=H(t)-H_0$, i.e. $\Delta_I(t)=U_0^{-1}(t)\Delta(t)U_0(t)$. Interaction operators evolve only through $U_0(t)$ (indirectly via $H_0$) and may have explicit time-dependence.
\end{remarks}
\begin{prop}
The time evolution operator may be written as a Dyson series.
\begin{equation}
U_I(t)=\sum_{n=0}^\infty(-i/\hbar)^n\int_0^t\int_0^{t_1}\int_0^{t_{n-1}}U_0^{-1}(t_1)\Delta(t)U_0(t_1-t_2)\Delta(t_2)...U_0(t_{n-1}-t_n)\Delta(t_n)U_0(t_n)dt_n...dt_1\label{Dyson}
\end{equation}
\end{prop}
\begin{proof}
If we define $\Delta_I(t):=U_0^{-1}(t)\Delta(t)U_0(t)$ as the perturbation in the interaction picture, we can say that $|\Psi_I(t)\rangle=U_I(t)|\Psi(0)\rangle$ for some unitary evolution operator that depends only on $\Delta(t)$. However, we should be careful: it is not usually true that $U_I(t)=\exp(-\frac{i}{\hbar}\int_0^t\Delta(t')dt')$. This is not true since typically $[\Delta(t'),\Delta(t'')]\neq0$. The Schr\"{o}dinger's equation in the interaction picture (Eqn.~\ref{interactionpicture2}) may be rewritten as
$$i\hbar\frac{\partial}{\partial t}|\Psi_I(t)\rangle=\Delta_I(t)|\Psi_I(t)\rangle=\Delta_I(t)U_I(t)|\Psi(0)\rangle$$
which is true $\forall|\Psi(0)\rangle\in\mathcal{H}$. We thus obtain an integral equation
$$U_I(t)=1-\frac{i}{\hbar}\int_0^t\Delta_I(t')U_I(t')dt'$$
This equation is exact but not immediately useful. To make progress, we approximate by repeatedly substituting the LHS of the integral equation into the integral on the RHS, hence obtaining a time-nested integral.
\begin{eqnarray}
U_I(t)&=&1-\frac{i}{\hbar}\int_0^t\Delta_I(t_1)dt_1+(-i/\hbar)^2\int_0^t\Delta_I(t_1)\int_0^{t_1}\Delta_I(t_2)U_I(t_2)dt_2dt_1\nonumber\\&=&\sum_{n=0}^\infty(-i/\hbar)^n\int_0^t\int_0^{t_1}\int_0^{t_{n-1}}U_0^{-1}(t_1)\Delta(t)U_0(t_1-t_2)\Delta(t_2)...U_0(t_{n-1}-t_n)\Delta(t_n)U_0(t_n)dt_n...dt_1\nonumber
\end{eqnarray}
Thus, at $n$th order in the expansion, the state evolves according to $H_0$ except for $n$ `strikes' of the perturbation.
\end{proof}
\subsection{Time evolution and time ordering}
Recall that time evolution is achieved by acting the state using a unitary operator in Hilbert space.
\begin{defi}[Time evolution operator]
The state vector at $t$ is related to that at some earlier time $t_0$ through
$$|\psi(t)\rangle=U(t,t_0)|\psi(t_0)\rangle$$
\end{defi}
\begin{eg}
The time evolution operator may be written as a series expansion of the Hamiltonian
$$U(t)=\Id-\frac{iHt}{\hbar}-\frac{1}{2}\bigg(\frac{Ht}{\hbar}\bigg)^2+\dots$$
But, $H$ has a complete orthonormal eigenbasis $|n\rangle$ ($H|n\rangle=E_n|n\rangle$), then we may write $U(t)$ in terms of the eigenbasis:
$$U(t)=\sum_ne^{-iE_nt/\hbar}|n\rangle\langle n|$$
\end{eg}
\begin{thm}
The time evolution operator is
\begin{itemize}
    \item If $H$ is time-independent, $U(t,t_0)=e^{-iH(t-t_0)/\hbar}$.
    \item If $[H(t_1),H(t_2)]=0$ $\forall t_1\neq t_2$, $U(t,t_0)=e^{-i\hbar^{-1}\int_{t_0}^tH(t')dt'}$.
    \item If $[H(t_1),H(t_2)]\neq 0$, then $U(t,t_0)$ is a time-ordered exponential, i.e.
    \begin{equation}
    U(t,t_0)=\mathcal{T}\bigg[\exp\bigg\{\bigg(\frac{-i}{\hbar}\bigg)\int_{t_0}^t H(t')dt'\bigg\}\bigg]\label{time-ordered}
    \end{equation}
\end{itemize}
\end{thm}
\begin{proof}
If $H$ is time-independent, we have from Schr\"{o}dinger equation $\frac{dU}{dt}=-i\hbar^{-1}HU$ and can be solved easily. If $H$ is time-dependent, we verify the given claim. Let $R(t):=-\frac{i}{\hbar}\int_{t_0}^tH(t')dt'$, then by Fundamental Theorem of Calculus, $R'=-\frac{i}{\hbar}H(t)$ and we can show $[R'(t),R(t)]=0$ as long as $[H(t),H(t')]=0$. Then, we try $\frac{d}{dt}e^R$, which gives us $R'e^R$ and hence recover Schr\"{o}dinger equation, as desired. Now, if Hamiltonians at different times don't commute, we have
\begin{align}
U(t,t_0)&=1-\frac{i}{\hbar}\int_{t_0}^tH(t')U(t',t_0)dt'\nonumber\\&=1+(-i\hbar^{-1})\int_{t_0}^tH(t_1)dt_1+(-i\hbar^{-1})^2\int_{t_0}^t\int_{t_0}^{t_1}H(t_1)H(t_2)dt_2dt_1+\dots\nonumber
\end{align}
where without loss of generality, let $t\geq t_0$. The term time-ordered refers to the fact that in the $n$th term of the series, we have a product of $H(t_1)H(t_2)...H(t_n)$ of non-commuting operators with integration ranges that force ordered times $t_1\geq t_2\geq...\geq t_n$. Consider for instance:
\begin{align}
\int_{t_0}^t\int_{t_0}^{t_1}H(t_1)H(t_2)dt_2dt_1&=\int_{t_0}^t\int_{t_0}^tH(t_1)H(t_2)\Theta(t_1-t_2)dt_1dt_2\nonumber\\&=\int_{t_0}^{t}\int_{t_0}^tH(t_2)H(t_1)\Theta(t_2-t_1)dt_1dt_2\nonumber\\&=\frac{1}{2}\int_{t_0}^t\int_{t_0}^t\mathcal{T}[H(t_2)H(t_1)]dt_1dt_2\nonumber
\end{align}
where $\Theta(t_1-t_2)$ is a step function that only gives 1 if $t_1\geq t_2$, which allows the limit on the second integral to be extended to $t$. The third line follows by exchanging of variables. The last line includes the time ordering operator which gives $H(t_2)H(t_1)$ if $t_2>t_1$ and $H(t_1)H(t_2)$ if $t_2<t_1$. By induction, we can show that
$$\int_{t_0}^t\int_{t_0}^{t_1}\int_{t_0}^{t_2}\dots\int_{t_0}^{t_{n-1}}H(t_1)H(t_2)\dots H(t_n)dt_1\dots dt_2=\mathcal{T}\bigg[\frac{1}{n!}\int_{t_0}^t\dots\int_{t_0}^t H(t_1)H(t_2)\dots H(t_n)dt_1\dots dt_n\bigg]$$
Hence, $U(t,t_0)$ is a series, i.e.
$$U(t,t_0)=\mathcal{T}\bigg[\sum_{n=0}^\infty\bigg(-\frac{i}{\hbar}\bigg)^n\frac{1}{n!}\int_{t_0}^t\dots\int_{t_0}^t H(t_1)H(t_2)\dots H(t_n)dt_1dt_2\dots dt_n\bigg]$$
where we note that for the $O(1/\hbar^n)$ term, there are a total of $n!$ permutations we have to account for. Finally, this is basically the power series of the exponential of an operator.
\end{proof}
\begin{remarks}
Now suppose $t<t_0$ instead, then $\mathcal{T}$ is the anti-time ordering operator which gives $H(t_1)H(t_2)$ if $t_2>t_1$ and vice-versa.
\end{remarks}
\begin{eg}[Driven harmonic oscillator]
Now consider a driven quantum harmonic oscillator
$$H(t)=\frac{p^2}{2m}+\frac{1}{2}m\omega^2x^2-F(t)x=\hbar\omega(a^\dag a+0.5)-F(t)\sqrt{\frac{\hbar}{2m\omega}}(a+a^\dag)$$
From Ehrenfest's theorem,
$$\frac{da_H}{dt}=-\frac{i}{\hbar}[a_H,H]=-i\omega a_H-(-F)\frac{i}{\hbar}\sqrt{\frac{\hbar}{2m\omega}}$$
But, $\tilde{a}(t)=e^{i\omega t}a_H$, so
$$\frac{d\tilde{a}}{dt}=i\omega\tilde{a}+e^{i\omega t}\frac{da_H}{dt}=\frac{iF(t)e^{i\omega t}}{\sqrt{2m\hbar\omega}}\implies\tilde{a}(t)=\tilde{a}(0)+\frac{i}{\sqrt{2m\hbar\omega}}\int_0^tF(t')e^{i\omega t'}dt'$$ Hence, $a_H=ae^{-i\omega t}+e^{-i\omega t}\frac{i}{\sqrt{2m\hbar\omega}}\int_0^tF(t')e^{i\omega t'}dt'$. Consider the time-dependent position:
\begin{align}
    x(t)=\sqrt{\frac{\hbar}{2m\omega}}(a_H+a_H^\dag)&=\sqrt{\frac{\hbar}{2m\omega}}(ae^{-i\omega t}+a^\dag e^{i\omega t})+\frac{i}{\sqrt{2m\hbar\omega}}\sqrt{\frac{\hbar}{2m\omega}}\int_0^tF(t')[e^{i\omega(t'-t)}-e^{-i\omega(t'-t)}]dt'\nonumber\\&=\sqrt{\frac{\hbar}{2m\omega}}(ae^{-i\omega t}+a^\dag e^{i\omega t})+\frac{1}{m\omega}\int_0^tF(t')\sin(\omega(t-t'))dt'\nonumber
\end{align}
where $F(t')\sin(\omega(t-t'))$ is the Green's function of the classical solution of a forced oscillator.
\end{eg}
\begin{eg}[Coherent state]
Consider the ground state $|0\rangle$. Since $a(t)=U^\dag(t) aU(t)\implies aU(t)=U(t)a(t)$, hence for the driven harmonic oscillator
$$aU(t)|0\rangle=\frac{i}{\sqrt{2m\hbar\omega}}\int_0^tF(t')e^{i\omega(t'-t)}dt'~U(t)|0\rangle$$
Hence, $U(t)|0\rangle$ is an eigenstate of $a$, i.e. $U(t)|0\rangle$ is a coherent state with eigenvalue $\frac{i}{\sqrt{2m\hbar\omega}}\int_0^tF(t')e^{i\omega(t'-t)}dt'$.
\end{eg}
\begin{remarks}[Coherent state properties recap]
A coherent state $|\alpha\rangle$ is defined as a state which is unchanged when operated on by an annihilation operator, i.e. $a|\alpha\rangle=\alpha|\alpha\rangle$. $a$ is not Hermitian and thus $\alpha$ is in general complex. It is constructed by displacing the ground state of a harmonic oscillator by the displacement operator, i.e. $D(\alpha)|0\rangle=e^{\alpha a^\dag-\alpha^*a}|0\rangle=e^{\alpha a^\dag}e^{-\alpha^* a}e^{-|\alpha|^2/2}|0\rangle$, but $e^{-\alpha^*a}|0\rangle=\frac{(\alpha^*a)^0}{\sqrt{0!}}|0\rangle=|0\rangle$. Since the number basis is complete, we may write the coherent state $|\alpha\rangle$ in terms of the number basis, i.e. $|\alpha\rangle=e^{-|\alpha|^2/2}\sum_{n=0}^\infty\frac{(\alpha)^n}{n!}\sqrt{n!}|n\rangle$. Introduce time-dependence
$$|\alpha(t)\rangle=e^{-i\omega t/2}e^{-|\alpha|^2/2}\sum_{n=0}^\infty\frac{\alpha^n}{\sqrt{n!}}e^{-in\omega t}|n\rangle=e^{-i\omega t/2}|e^{-i\omega t}\alpha\rangle$$
where $|n(t)\rangle=e^{-iE_nt/\hbar}|n(0)\rangle$, $E_n=\hbar\omega(n+0.5)$. Up to a phase, this is still a coherent state. The coherent state preserves the coherent state form as it evolves, where $|\alpha(t)\rangle$ has eigenvalue $\alpha e^{-i\omega t}$ (for the eigenoperator $a$), then the expectations may replace $\alpha\rightarrow\alpha e^{-i\omega t}$: $\langle x\rangle=\sqrt{\frac{\hbar}{2m\omega}}2\text{Re}[\alpha e^{-i\omega t}],\quad \langle p\rangle=-i\sqrt{\frac{m\hbar\omega}{2}}2i\text{Im}[\alpha e^{-i\omega t}]$. Write $\alpha=|\alpha|e^{i\phi}$, then the position and momentum of centre of the coherent state follow that of the classical state, i.e. $x_0(t)=\sqrt{\frac{2\hbar}{m\omega}}|\alpha|\cos(\phi-\omega t)$, $p_0(t)=\sqrt{2m\hbar\omega}|\alpha|\sin(\phi-\omega t)$. The width of the coherent state wavepacket $\Delta x$ remains time independent. The classical character is lost once the energy or photon number is measured.
\end{remarks}
\subsection{Two-level systems and Rabi oscillations}
\begin{defi}[Two-level Hamiltonian]
Since every 2 by 2 Hermitian operator may be expanded as a linear combination of Pauli matrices, we may write the Hamiltonian as
\begin{equation}
    H(t)=\mathbf{h}(t)\cdot\mathbf{S}+\frac{\hbar}{2}h_0\Id=\frac{\hbar}{2}\begin{pmatrix}h_z(t)+h_0&h_x(t)-ih_y(t)\\h_x(t)+ih_y(t)&-h_z(t)+h_0\\\end{pmatrix}\label{2level}
\end{equation}
where $\mathbf{S}=\frac{\hbar}{2}\sigma$. 
\end{defi}
\begin{remarks}
$\mathbf{h}(t)$ is a real 3-component vector that traces a path along the Bloch sphere $\mathcal{S}^2$. $h_0$ is an energy offset and not relevant to the dynamics, and we will thus drop it.
\end{remarks}
\begin{prop}[Spin precession]
The spin precesses about the axis along $\mathbf{h}(t)$:
\begin{equation}
    \frac{d\mathbf{S}(t)}{dt}=\mathbf{h}(t)\times\mathbf{S}(t)\label{spinprecession}
\end{equation}
\end{prop}
\begin{proof}
Looking at the Heisenberg picture, the spin operator is $\mathbf{S}(t)=U^\dag(t)\mathbf{S}U(t)$. Heisenberg's equation of motion gives
$$\frac{dS_i}{dt}=\frac{i}{\hbar}h_j[S_j,S_i]=\frac{i}{\hbar}h_ji\varepsilon_{ijk}S_k\hbar=\varepsilon_{ijk}h_jS_k$$
where we used $[S_i,S_j]=\hbar i\varepsilon_{ijk}S_k$.
\end{proof}
\begin{remarks}
We could have also written the spin evolution via a rotation operator that relates the initial and final spin state.
\begin{equation}
\mathbf{S}(t)=R(t,t')\mathbf{S}(t'),~ R(t,t')=\mathcal{T}\exp\bigg(\int_{t'}^t\Omega(t_i)dt_i\bigg),~\Omega_{jk}(t)=-\frac{1}{\hbar}h_i(t)\varepsilon_{ijk}=\frac{1}{\hbar}\begin{pmatrix}0&-h_z(t)&h_y(t)\\h_z(t)&0&-h_x(t)\\-h_y(t)&h_x(t)&0\\\end{pmatrix}\label{rotation}
\end{equation}
where $\Omega(t)$ is a matrix that describes infinitesimal rotations. A time ordering was necessary since rotations at different times do not commute.
\end{remarks}
\begin{eg}
Suppose $\mathbf{h}$ is purely in the $\mathbf{\hat{z}}$ direction, which up to a constant, corresponds to a magnetic field in the $z$ direction: $H(t)=\frac{B}{2}\diag(1,-1)$.
The time evolution operator is
$$U(t,t')=\mathcal{T}\exp\bigg(-\frac{i}{\hbar}\int_{t'}^tH(t_i)dt_i\bigg)=\exp\bigg(-\frac{i}{\hbar}\begin{pmatrix}B(t-t')/2&0\\0&-B(t-t')/2\\\end{pmatrix}\bigg)=\begin{pmatrix}e^{-\frac{iB}{2\hbar}(t-t')}&0\\0&e^{-\frac{iB}{2\hbar}(t-t')}\\\end{pmatrix}$$
On the other hand, since $\Omega_{jk}=-B\delta_{iz}\varepsilon_{ijk}$, then by Eqn.~\ref{rotation}, $\Omega(t)=iB\sigma_y\oplus 0$. We may either diagonalize to exponentiate or
$$R(t,t')=e^{iB(t-t')\sigma_y}\oplus \exp(0)=\bigg(\cos(B(t-t'))+i\sigma_y\sin(B(t-t'))\bigg)\oplus 1=\begin{pmatrix}\cos B(t-t')&-\sin B(t-t')&0\\\sin B(t-t') & \cos B(t-t') &0\\0&0&1\\\end{pmatrix}$$
\end{eg}
\begin{remarks}
\begin{equation}
    \exp(-i\boldsymbol{\theta}\cdot\mathbf{S}/\hbar)=\Id\cos(\theta/2)-i\mathbf{n}\cdot\boldsymbol{\sigma}\sin(\theta/2),\quad\boldsymbol{\theta}=\theta\mathbf{\hat{n}},~\mathbf{S}=\frac{\hbar}{2}\boldsymbol{\sigma}\label{expidentity}
\end{equation}
\end{remarks}
\begin{prop}
Consider a rotating magnetic field that gives the form $\mathbf{h}(t)=(h_R\cos\omega t\mathbf{\hat{x}}+h_R\sin\omega t\mathbf{\hat{y}}+h_0\mathbf{z})$, then the two-level state precesses about a fixed axis $(h_R,0,h_0-\omega)$ at the Rabi frequency
\begin{equation}
    \omega_R=\frac{1}{\hbar}\sqrt{(h_0-\hbar\omega)^2+h_R^2}\label{rabi}
\end{equation}
\end{prop}
\begin{proof}
The resulting time-dependent Hamiltonian is $H(t)=h_0S_z+\frac{1}{2}h_R(S_+e^{-i\omega t}+S_-e^{i\omega t})$. We perform a trick - transform to the rotating frame to obtain an effective time-independent Hamiltonian. The state in the rotating frame is $|\psi_{\text{rot}}(t)\rangle=e^{i\omega tS_z/\hbar}|\psi(t)\rangle$, where $U=e^{i\omega t}S_z$. Take the time derivative:
$$i\hbar\frac{d}{dt}|\psi_{\text{rot}}\rangle=-\omega S_z|\psi_{\text{rot}}(t)\rangle+e^{i\omega tS_z}i\hbar\frac{d}{dt}|\psi(t)\rangle=-\hbar\omega S_z|\psi_{\text{rot}}(t)\rangle+e^{i\omega tS_z}H(t)e^{-i\omega tS_z}e^{i\omega tS_z}|\psi(t)\rangle$$
RHS is $(-\hbar\omega S_z+H_{\text{rot}})|\psi_{\text{rot}}\rangle$, where $H_{\text{rot}}=UHU^\dag$. But, we have
$$H_{\text{rot}}=UHU^\dag =U(h_0S_z+\frac{1}{2}h_R(S_+e^{-i\omega t}+S_-e^{i\omega t})U^\dag =h_0S_z+\frac{1}{2}h_R(e^{i\omega tS_z}S_+e^{-i\omega tS_z}e^{-i\omega t}+\text{h.c})$$
Now, take the derivative $\frac{d}{dt}(e^{i\omega tS_z}S_+e^{-i\omega tS_z})$:
$$\frac{d}{dt}(e^{i\omega tS_z}S_+e^{-i\omega tS_z})=i\omega e^{i\omega tS_z}[S_z,S_+]e^{-i\omega tS_z}=i\omega e^{i\omega tS_z}S_+e^{-i\omega tS_z}\implies e^{i\omega tS_z}S_+e^{-i\omega tS_z}=e^{i\omega t}S_+$$
where $[S_z,S_+]=[S_z,S_x+iS_y]=iS_y+i(-iS_x)=S_+$.Hence, the rotating Hamiltonian is
$$H_{\text{rot}}=h_0S_z+\frac{h_R}{2}(S_+e^{i\omega t}e^{-i\omega t}+\text{h.c})=h_0S_z+h_RS_x\implies i\hbar\frac{d}{dt}|\psi_{\text{rot}}\rangle=(H_{\text{rot}}-\omega S_z)|\psi_{\text{rot}}\rangle$$
giving the effective time-independent Hamiltonian, the Rabi Hamiltonian - $H_{\text{Rabi}}=(h_0-\omega)S_z+h_RS_x$. Diagonalizing this directly gives the Rabi frequency.
\end{proof}
\begin{remarks}
The amplitude of the oscillations in $S_z$ due to this precession is maximal when $h_0=\omega$, i.e. match the precession frequency. The spin precesses about the $x$ axis, i.e. $h_RS_x$. The spin which points initially along the $z$ axis will rotate towards the $y$ axis, about the $x$ axis.
\end{remarks}
\subsection{Adiabatic theorem}
\begin{defi}[Adiabatic process]
Adiabatic processes are processes in which a time dependence is introduced into the physics of a system by letting some parameters that control the dynamics vary slowly in time. The change in time of these parameters, say $\lambda(t)$, is said to be adiabatic if it is slow compared with natural time scales of the system $\tau$, i.e.
\begin{equation}
    \tau\bigg|\frac{d\lambda}{dt}\bigg|<<|\lambda|\label{adiabatic}
\end{equation}
\end{defi}
\begin{remarks}
Often, the natural time scale of the evolving system is itself changing in time, we can take $\lambda$ to be $\tau$ itself and Eqn.~\ref{adiabatic} gives $|\frac{d\tau}{dt}|<<1$.
\end{remarks}
When the parameter $\lambda(t)$ varies slowly, physical quantities that were formerly conserved can also begin to vary slowly, at least after averaging out rapid oscillations. Like $\lambda$, they can also accumulate finite changes over the long time $T$.
\begin{defi}[Adiabatic invariant]
An adiabatic invariant $I$, constructed in terms of $\lambda(t)$ and other slowly varying quantities that essentially remains constant. Throughout the time interval $[0,T]$ in which $\lambda$ changes by $\Delta\lambda$, the adiabatic invariant $I$ changes very little, i.e. for any $t\in[0,T]$,
$$\lim_{T\rightarrow\infty}|I(t)-I(0)|=0$$
\end{defi}
\begin{eg}[Harmonic oscillator]
Consider the classical harmonic oscillator. Suppose we are going to change $\omega$ through a finite amount $\Delta\omega$, comparable to $\omega$, during the time interval $t\in[0,T]$. The change in $\omega$ is said to be adiabatic if by Eqn.~\ref{adiabatic} gives
$$\tau_\omega|\dot{\omega}|<<\omega\implies|\dot{\omega}\omega^{-2}|<<1$$
where $\tau_\omega=\frac{2\pi}{\omega}$. It can be shown that the adiabatic invariant is $I(t)=\frac{H(t)}{\omega(t)}$, where $H(t)$ is the Hamiltonian of the harmonic oscillator. $I$ has a nice geometrical interpretation in phase space - the area of the closed ellipse traced out in phase space, i.e. $\oint pdx=2\pi I$, which is similar to the Bohr-Sommerfield quantization condition.
\end{eg}
\begin{defi}[Instantaneous eigenstate]
For a time-dependent Hamiltonian $H(t)$, the states $|\psi(t)\rangle$ that satisfy
$$H(t)|\psi(t)\rangle=E(t)|\psi(t)\rangle$$
are called instantaneous eigenstates. It is, however, not a solution of the time-dependent Schr\"{o}dinger's equation. In the adiabatic approximation, $|\psi(t)\rangle$ is an approximate solution to the Schr\"{o}dinger's equation.
\end{defi}
\begin{remarks}[Phase ambiguity]
Consider a time-independent Hamiltonian that depends on some parameters, i.e. $H(\{R_i\})$. The eigenstates will be $H(\mathbf{R})\psi(\mathbf{x},\mathbf{R})=E(\mathbf{R})\psi(\mathbf{x},\mathbf{R})$. Now imagine the parameters become time-dependent $\mathbf{R}\rightarrow\mathbf{R}(t)$. Since the time-independent equation holds for arbitrary values of the parameters, we have that for any value of time
$$H(\mathbf{R}(t))\psi(\mathbf{x},\mathbf{R}(t))=E(\mathbf{R}(t))\psi(\mathbf{x},\mathbf{R}(t))$$
However, the instantaneous eigenstates $\psi(\mathbf{x},\mathbf{R})$ has a phase ambiguity $e^{i\gamma(\mathbf{R})}$ where $\gamma(\mathbf{R})$ is a real function of the $R$ parameters. 
\end{remarks}
\begin{thm}[Adiabatic theorem]
Let $H(t)$ be a slowly varying Hamiltonian for $0\leq t\leq T$. Let the state of the system at $t=0$ be $|\Psi(0)\rangle=|\psi_n(0)\rangle$ for some $n$, then at any time $t\in[0,T]$ we have $|\psi(t)\rangle\approx|\psi_n(t)\rangle$ up to a calculable phase. The amplitude to transition to any other instantaneous eigenstate is of order $1/T$. 
\end{thm}
\begin{proof}
Using the basis provided by the instantaneous eigenstate, our time-dependent state is $|\Psi(t)\rangle=\sum_nc_n(t)|\psi_n(t)\rangle$. The Schr\"{o}dinger's equation gives
$$i\hbar\sum_n\bigg(\dot{c}_n|\psi_n(t)\rangle+c_n|\dot{\psi}_n(t)\rangle\bigg)=\sum_nc_n(t)E_n(t)|\psi_n(t)\rangle\implies i\hbar\dot{c}_k=(E_k-i\hbar\langle\psi_k|\dot{\psi}_k\rangle)c_k-i\hbar\sum_{n\neq k}\langle\psi_k|\dot{\psi}_n\rangle c_n$$
The terms in the sum couple the $n\neq k$ eigenstates to the $k$ eigenstates, i.e. produce transitions. But, take the time derivative of $H(t)|\psi_n(t)\rangle=E_n(t)|\psi_n(t)\rangle$:
$$\frac{dH(t)}{dt}|\psi_n(t)\rangle+H(t)|\dot{\psi}_n(t)\rangle=\frac{dE_n(t)}{dt}|\psi_n(t)\rangle+E_n(t)|\dot{\psi}_n(t)\rangle$$
Multiply by $\langle\psi_k(t)|$ from the left with $k\neq n$:
$$\langle\psi_k(t)|\dot{H}|\psi_n(t)\rangle+E_k(t)\langle\psi_k(t)|\dot{\psi}_n(t)\rangle=E_n(t)\langle\psi_k(t)|\dot{\psi}_n(t)\rangle\implies\langle\psi_k(t)|\dot{\psi}_n(t)\rangle=\frac{\langle\psi_k(t)|\dot{H}|\psi_n(t)\rangle}{E_n(t)-E_k(t)}=\frac{\dot{H}_{kn}}{E_n-E_k}$$
Plug back into the Schr\"{o}dinger's equation:
$$i\hbar\dot{c}_k=(E_k-i\hbar\langle\psi_k|\dot{\psi}_k\rangle)c_k-i\hbar\sum_{n\neq k}\frac{\dot{H}_{kn}}{E_n-E_k}c_n$$
The transition amplitudes are suppressed by factors of $1/T$.
\end{proof}
\begin{remarks}
For a two-level system, the requirement for adiabaticity is to have vanishing off-diagonal terms, i.e.
$$\bigg|\frac{\hbar\langle+,t|\dot{H}|-,t\rangle}{E_+-E_-}\bigg|<<|E_+-E_-|$$
where $|\pm,t\rangle$ are the instantaneous eigenstates. This is valid away from a degeneracy. No transitions will occur if $H(t)$ is sufficiently slowly varying or in the semiclassical limit where the action $S>>\hbar$.
\end{remarks}
\begin{cor}
\begin{equation}
    |\Psi(t)\rangle\approx e^{i\theta_n(t)}e^{i\gamma_n(t)}|\psi_n(t)\rangle\label{phase}
\end{equation}
where the phases are
\begin{equation}
\theta_k(t)=\frac{-1}{\hbar}\int_0^tE_k(t')dt',\quad\gamma_k(t)=\int_0^t\nu_k(t')dt',\quad\nu_k(t)=i\langle\psi_k(t)|\dot{\psi}_k(t)\rangle\label{phase2}
\end{equation}
respectively are called dynamical phase and Berry phase. $\nu_k$ is called the Berry's connection.
\end{cor}
\begin{proof}
If we ignore the transition causing terms, the Schr\"{o}dinger's equation would previously give
$$i\hbar\dot{c}_k=(E_k-i\hbar\langle\psi_k|\dot{\psi}_k\rangle)c_k\implies c_k(t)=c_k(0)e^{\frac{1}{i\hbar}\int_0^t E_k(t')dt'}e^{i\int_0^ti\langle\psi_k|\dot{\psi}_k\rangle dt'}$$
\end{proof}
\begin{eg}
In the Born-Oppenheimer approximation used in the study of molecules, we assume the nuclei move slowly compared to the atoms such that the nuclei motion induces adiabatic changes of the electronic state.
\end{eg}
\subsection{Landau-Zener transition}
An idealized two-state system considered by Landau and Zener can be used to examine non-adiabatic transitions and calculate their probability. 
\begin{eg}[Avoided crossings]
If the energies of two levels do not cross, this is called an avoided crossing. Avoided crossings are common. The energy eigenvalues of a generic two-level system is
$$E_\pm=h_0\pm|\mathbf{h}|\implies E_+-E_-=2|\mathbf{h}|$$
For the levels to cross, we need $|\mathbf{h}|=0\implies\mathbf{h}=\boldsymbol{0}$ which requires fine-tuning three out of the four parameters, but still possible in the presenc eof a symmetry.
\end{eg}
\begin{prop}
Consider a time-dependent two-level Hamiltonian of the form
\begin{equation}
    H(t)=\begin{pmatrix}0.5\alpha t&H_{12}\\H^*_{12}&-0.5\alpha t\\\end{pmatrix},\quad\alpha>0\label{LandauZener}
\end{equation}
The evolution is adiabatic if $\frac{|H_{12}|^2}{\hbar\alpha}>>1$.
\end{prop}
\begin{proof}
First consider Eqn.~\ref{LandauZener} with off-diagonal elements to be zero. Let the basis states be $|1\rangle=(1,0)$ and $|2\rangle=(0,1)$. Since $H(t)$ is now diagonal, $|1\rangle$ and $|2\rangle$ are instantaneous eigenstates with energies $E_1(t)=\frac{1}{2}\alpha t$ and $E_2(t)=-\frac{1}{2}\alpha t$. Moreover, we get level crossings when the diagonal elements vanish, which happens at $t=0$. Although the states cross, they ignore each other, with the $|1\rangle$ and $|2\rangle$ remaining in their states forever.\\[5pt]
Now, consider constant off-diagonal terms $H_{12}\in\mathbb{C}$. The energies would be $E_\pm(t)=\pm\sqrt{|H_{12}|^2+0.25\alpha^2t^2}$. We now have an avoided crossing but the states never become degenerate. The upper and lower branch corresponds to instantaneous eigenstates $|\psi_+(t)\rangle$ and $|\psi_-(t)\rangle$ respectively. For large time, the effect of the off-diagonal terms is negligible and the states must coincide with the states determined for vanishing off-diagonal terms
$$|\psi_+(t)\rangle\rightarrow\begin{pmatrix}0\\1\\\end{pmatrix}\text{ as }t\rightarrow-\infty,\quad|\psi_+(t)\rangle\rightarrow\begin{pmatrix}1\\0\\\end{pmatrix}\text{ as }t\rightarrow+\infty$$
$$|\psi_-(t)\rangle\rightarrow\begin{pmatrix}1\\0\\\end{pmatrix}\text{ as }t\rightarrow-\infty,\quad|\psi_-(t)\rangle\rightarrow\begin{pmatrix}0\\1\\\end{pmatrix}\text{ as }t\rightarrow+\infty$$
Adiabatic changes can occur - as time goes from $-\infty$ to $+\infty$, the instantaneous eigenstates evolve - $|\psi_+\rangle$ from $(0,1)$ to $(1,0)$ and $|\psi_-\rangle$ from $(1,0)$ to $(0,1)$. Non-adiabatic changes are those $|\psi_\pm\rangle\rightarrow|\psi_\mp\rangle$ as $t$ goes from $-\infty$ to $+\infty$. The system, here, jumps across instantaneous eigenstates but staying in $|1\rangle=(1,0)$ or $|2\rangle=(0,1)$ respectively.\\[5pt]
The timescales relevant for the transition are the quantum timescale $T_{12}=2\pi/\omega_{12}=2\pi\hbar/|H_{12}|$ and the duration $\tau_d$ ($2\tau_d$ is the time required for $E_1$ or $E_2$ to change by a magnitude $|H_{12}|=\frac{1}{2}\alpha2\tau_d\implies\tau_d=\frac{|H_{12}|}{\alpha}$). The evolution is adiabatic (Eqn.~\ref{adiabatic}) if $T_{12}<<\tau_d\implies\omega_{12}\tau_d=\frac{|H_{12}|^2}{\hbar|\alpha|}>>1$.
\end{proof}
\begin{remarks}
To calculate the probability of non-adiabatic transitions, we need to find exact solutions. To build exact solutions of the Schr\"{o}dinger equation from these instantaneous eigenstates, consider $|\psi_1(t)\rangle=f_1(t)|1\rangle$:
$$i\hbar\partial_t|\psi_1(t)\rangle=H|\psi_1(t)\rangle\implies f_1(t)=\exp\bigg(\frac{1}{i\hbar}\int_0^tE_1(t')dt'\bigg)=\exp\bigg(-\frac{i\alpha t^2}{4\hbar}\bigg)$$
Similarly, $|\psi_2(t)\rangle=e^{i\alpha t^2/4\hbar}|2\rangle$. The general solution has the form $A(t)f_1(t)|1\rangle+B(t)f_2(t)|2\rangle$. If we start with $|1\rangle$ at $t\rightarrow-\infty$, the amplitude $A$ at $t\rightarrow\infty$ is the amplitude for a non-adiabatic transition. Schr\"{o}dinger's equation will give
$$\dot{A}=-\frac{i}{\hbar}H_{12}B(t)e^{i\theta_{12}(t)},\quad\dot{B}=-\frac{i}{\hbar}H_{12}^*A(t)e^{-i\theta_{12}(t)},\quad\theta_{12}(t)=\frac{1}{\hbar}\int_0^tE_{12}(t')dt'$$
One can hence show that, in the limit as $|H_{12}|\rightarrow 0$, the probability for a non-adiabatic transition is $e^{-2\pi\frac{|H_{12}|^2}{\hbar\alpha}}$.
\end{remarks}
\begin{eg}
When $\frac{\hbar|\alpha|}{|H_{12}|^2}>>1$, we expect a system that starts out at in the lower state at $t\rightarrow-\infty$ to end up in the upper state, with only a small probability of remaining in the lower state. This can be treated using time-dependent perturbation theory with $|H_{12}|$ as the perturbation and $H_0=0.5\diag(\alpha t,-\alpha t)$ as the unperturbed Hamiltonian. Since the unperturbed levels pass through each other, staying in the lower state corresponds to making a transition between the unperturbed states. From time-dependent perturbation theory and $\sqrt{0.25(\alpha t)^2+|H_{12}|^2}\approx0.5\alpha t$
$$c_+^{(1)}=\frac{-i}{\hbar}\int_{-\infty}^\infty\exp\bigg(i\int_0^{t'}E_0^+(t'')-E_0^-(t'')dt''\bigg)|H_{12}|dt'=\frac{i|H_{12}|}{-\hbar}\int_{-\infty}^\infty e^{0.5i(\alpha-(-\alpha))t'^2/2}dt'=-\frac{i|H_{12}|}{\hbar}\sqrt{\frac{2\pi\hbar}{i\alpha}}$$
The transition probability is $|c_+|^2=\frac{2\pi|H_{12}|^2}{\alpha\hbar}$ which is an approximation to $1-e^{-2\pi|H_{12}|^2/\hbar\alpha}$ when $\frac{\hbar|\alpha|}{|H_{12}|^2}>>1$. 
\end{eg}
\subsection{Berry's phase}
We will now explain why the geometrical phase (or Berry's phase) is geometrical. Assume we have eigenstates for all values of the coordinates $H(\mathbf{R})|\psi_n(\mathbf{R})\rangle=E(\mathbf{R})|\psi_n(\mathbf{R})\rangle$, then for time-dependent coordinates $\mathbf{R}(t)$, we have instantaneous eigenstates $H(\mathbf{R}(t))|\psi_n(\mathbf{R}(t))\rangle=E(\mathbf{R}(t))|\psi_n(\mathbf{R}(t))\rangle$. The time evolution of the Hamiltonian can be thought as a path in the configuration space, parametrized by time.
\begin{prop}[Berry's phase]
The geometric phase accumulated from some initial state $t_i$ up to some final state $t_f$ (as the path evolves in configuration space) is
\begin{equation}
    \gamma_n(t_i,t_f)=\int_{\mathbf{R_i}}^{\mathbf{R_f}}\langle\psi_n(\mathbf{R}(t))|\boldsymbol{\nabla_R}|\psi_n(\mathbf{R}(t))\rangle\cdot d\mathbf{R}\label{Berry}
\end{equation}
\end{prop}
\begin{proof}
The integrand in the geometric phase is
$$\nu_n(t)=i\langle\psi_n(\mathbf{R}(t))|\frac{d}{dt}|\psi_n(\mathbf{R}(t))\rangle=i\langle\psi_n(\mathbf{R}(t))|\sum_{i=1}^N\frac{\partial}{\partial R_i}|\psi_n(\mathbf{R}(t))\rangle\frac{dR_i}{dt}=i\langle\psi_n(\mathbf{R}(t))|\boldsymbol{\nabla_R}|\psi_n(\mathbf{R}(t))\rangle\cdot\frac{d\mathbf{R}(t)}{dt}$$
The geometric phase accumulated from some initial time $t_i$ up to some final time $t_f$ is $\gamma_n(t_i,t_f)=\int_{t_i}^{t_f}\nu_n(t)dt$. The result follows.
\end{proof}
\begin{remarks}
The integral depends on the path $\Gamma_{if}$ in configuration space but does not depend on time, hence a geometric phase. It does not depend at all on the parameterization of the path by the time parameter.
\end{remarks}
\begin{defi}[Berry connection]
\begin{equation}
    \mathbf{A_n}(\mathbf{R})=i\langle\psi_n(\mathbf{R})|\boldsymbol{\nabla_R}|\psi_n(\mathbf{R})\rangle\label{connection}
\end{equation}
where the Berry's phase is $\gamma_n(\Gamma_{if})=\int_{\Gamma_{if}}\mathbf{A_n}(\mathbf{R})\cdot d\mathbf{R}$ for the path $\Gamma_{if}$.
\end{defi}
\begin{cor}\leavevmode
\begin{enumerate}
    \item The geometrical phase is unambigiously defined for closed paths $\Gamma$ in parameter space, i.e. $\gamma_n(\Gamma)$ is an observable if $\Gamma$ is a closed path. Conversely, it is ambiguous for an open path.
    \item If $|\psi_n(t)\rangle$ are real, Berry's phase vanishes. 
    \item If the configuration space is one-dimensional, Berry's phase vanishes for any loop $\Gamma$.
    \item If the configuration space is three-dimensional, we can define the Berry's curvature $\mathbf{B_n}=\boldsymbol{\nabla_R}\times\mathbf{A_n}(\mathbf{R})$, with a magnetic field analog.
\end{enumerate}
\end{cor}
\begin{proof}\leavevmode
\begin{enumerate}
    \item The gauge transformations arise from redefinitions of the instantaneous eigenstates by phases reflecting the ambiguity of these states.
    $$|\psi_n(\mathbf{R})\rangle\rightarrow|\tilde{\psi}_n(\mathbf{R})\rangle=e^{-i\beta(\mathbf{R})}|\psi_n(\mathbf{R})\rangle\implies\mathbf{\tilde{A}_n}(\mathbf{R})=i(-i\boldsymbol{\nabla_R}\beta(\mathbf{R}))\langle\psi_n(\mathbf{R})|\psi_n(\mathbf{R})\rangle+\mathbf{A_n}(\mathbf{R})$$
    where $\beta(\mathbf{R})$ is an arbitrary real function. The corresponding Berry's phase is
    $$\tilde{\gamma}_n(\Gamma_{if})=\int_{\Gamma_{if}}(\mathbf{A_n}(\mathbf{R})+\boldsymbol{\nabla_R}\beta(\mathbf{R}))\cdot d\mathbf{R}=\gamma_n(\Gamma_{if})+\int_{\Gamma_{if}}d\beta(\mathbf{R})$$
    The second term is an integrand of a total derivative and thus picks the values of the integrand at the final and initial points of the path. This leads to ambiguity for open paths since $\beta(\mathbf{R})$ may be suitably chosen to lead to vanishing geometric phase. But if $\mathbf{R_f}=\mathbf{R_i}\implies\beta(\mathbf{R_f})=\beta(\mathbf{R_i})$, we have a gauge-invariant phase.
    \item $\langle\psi_n(t)|\dot{\psi}_n(t)\rangle$ is purely imaginary, so the overlap with a real vector will lead to zero. To show that, take $\frac{d}{dt}\langle\psi(t)|\psi(t)\rangle=0\implies\langle\psi(t)|\dot{\psi}(t)\rangle^*+\langle\psi(t)|\dot{\psi}(t)\rangle=0$, i.e. the real part of $\langle\dot{\psi}(t)|\psi(t)\rangle$ vanishes.
    \item $\oint_\Gamma\mathbf{A_n}(R)dR=0$. For a loop stretching from $R_i$ to $R_f$ and then back the contribution to the phase from $R_i$ to $R_f$ is cancelled by the contribution from $R_f$ to $R_i$.
    \item Consider Berry's phase for a closed loop $\Gamma$ in $\mathbb{R}^3$ configuration space. By Stokes' theorem, i.e.     $\gamma_n(\Gamma)=\oint_\Gamma\mathbf{A_n}(\mathbf{R})\cdot d\mathbf{R}=\int\int_S(\boldsymbol{\nabla_R}\times\mathbf{A_n})\cdot d\mathbf{S}$. The Berry's curvature is gauge invariant under the gauge transformation (gauge $\beta$ as before). Unlike magnetic field, the Berry's curvature can have monopoles - sources and sinks.
\end{enumerate}
\end{proof}
\begin{remarks}
While the phase of a single state is not observable, Berry's phase is observable in a setting with more than one state. The relative difference between the phases acquired by any two states at the end of an evolution can be observed.
\end{remarks}
\begin{eg}[Spin-1/2 in magnetic field]
Here, the configuration space is $(B_0,\theta,\phi)$ which are time-dependent. Suppose $\mathbf{B}(t)=B_0(t)\mathbf{n}(t)$ changes in a way that it traces a closed loop $\Gamma$ in configuration space, then the tip of the unit vector $\mathbf{n}$ traces a closed path on the unit sphere, and that the magnitude of $B_0$ of the magnetic field changes but returns to its initial value. We want to compute Berry's phase for this loop. The instantaneous eigenstates of $H(\mathbf{n})=\mu_BB_0\mathbf{n}\cdot\boldsymbol{\sigma}$ are
$$H(B_0,\mathbf{n})|\pm\mathbf{n}\rangle=\pm\mu_BB_0|\pm\mathbf{n}\rangle,\quad|+\mathbf{n}\rangle=\begin{pmatrix}\cos0.5\theta\\e^{i\phi}\sin0.5\theta\\\end{pmatrix},~|-\mathbf{n}\rangle=\begin{pmatrix}\sin0.5\theta\\-e^{i\phi}\cos0.5\theta\\\end{pmatrix}$$
The Berry's phase $\gamma$ for $|+\mathbf{n}\rangle$ gives
$$\mathbf{A_+}(\mathbf{R})\cdot d\mathbf{R}=i\langle+\mathbf{n}|\boldsymbol{\nabla_R}|+\mathbf{n}\rangle\cdot d\mathbf{R}=i\langle+\mathbf{n}|\bigg(\frac{\partial}{\partial B_0}|+\mathbf{n}\rangle dB_0+\frac{\partial}{\partial \theta}|+\mathbf{n}\rangle d\theta+\frac{\partial}{\partial \phi}|+\mathbf{n}\rangle d\phi\bigg)$$
The first two contributions evaluate to 0 while the third  gives $\langle+\mathbf{n}|\partial_\phi|+\mathbf{n}\rangle=i\sin^20.5\theta$. Hence, $\gamma_+(\Gamma)=-\frac{1}{2}\int_\Gamma(1-\cos\theta)d\phi$. Here, $\Gamma$ is the curve traced by the unit vector $\mathbf{n}$ in the adiabatic process. Let $S_\Gamma$ denote the surface on the unit sphere whose boundary is $\Gamma$. Consider a tiny rectangle $d\Gamma$ with infinitesimal lengths $d\theta$, $d\phi$, with diagonal corners at $(\theta,\phi)$ and $(\theta+d\theta,\phi+d\phi)$.
$$\oint_{d\Gamma}(1-\cos\theta)d\phi=(\cos\theta-\cos(\theta+d\theta))d\phi=\sin\theta d\theta d\phi=d\Omega$$
where the area element on the unit sphere equals to the solid angle element. For the surface $S_\Gamma$ with boundary $\Gamma$, we can show in general that $\int_\Gamma(1-\cos\theta)d\phi=\Omega(S_\Gamma)$.
\end{eg}
\newpage
\section{Path Integrals}
\subsection{Propagators}
\begin{defi}[Propagator]
The propagator is the position representation of the time evolution operator.
\begin{equation}
K(\mathbf{r},t|\mathbf{r'},t')=\Theta(t-t')\langle\mathbf{r}|U(t-t')|\mathbf{r'}\rangle\label{propagator}
\end{equation}
where $\Theta(t-t')$ is a Heaviside step function that is 1 if $t\geq t'$ and 0 otherwise, and $|\mathbf{r}\rangle$ is the position eigenstate.
\end{defi}
\begin{prop}
The propagator propagates the wavefunction forward in time.
\end{prop}
\begin{proof}
Working in the position representation,
$$\psi(\mathbf{r},t)=\langle\mathbf{r}|\psi(t)\rangle=\langle\mathbf{r}|U(t-t')|\psi(t')\rangle=\int\langle\mathbf{r}|U(t-t')|\mathbf{r'}\rangle\langle\mathbf{r'}|\psi(t')\rangle d\mathbf{r'}$$
giving $\psi(\mathbf{r},t)=\int K(\mathbf{r},t|\mathbf{r'},t')\psi(\mathbf{r'},t)d\mathbf{r'}$ for $t\geq t'$.
\end{proof}
\begin{remarks}\leavevmode
\begin{enumerate}
\item We can always split the time evolution into steps: $U(t-t')=U(t-t'')U(t''-t')$, for $t\geq t''\geq t'$, which gives
\begin{equation}
    K(\mathbf{r},t|\mathbf{r'},t')=\int K(\mathbf{r},t|\mathbf{r''},t'')K(\mathbf{r''},t''|\mathbf{r'},t')d\mathbf{r'}\label{composition}
\end{equation}
\item The propagator is a Green's function solution, i.e.
$$\bigg(i\hbar\frac{\partial}{\partial t}-H\bigg)K(\mathbf{r},t|\mathbf{r'},t')=i\hbar\delta(\mathbf{r}-\mathbf{r'})\delta(t-t')$$
and 0 if $t<t'$. This is equivalent to Eqn.~\ref{propagator}. Integrate this through $t\in(t'-\xi,t'+\xi)$:
\begin{align}
    i\hbar\delta(\mathbf{r}-\mathbf{r'})&=\int^{t'+\xi}_{t'-\xi}i\hbar\frac{\partial}{\partial t}K(\mathbf{r},t|\mathbf{r'},t')-HK(\mathbf{r},t|\mathbf{r'},t')dt\nonumber\\&=i\hbar K(\mathbf{r},t'+\xi|\mathbf{r'},t')-\int_{t'-\xi}^{t'+\xi}HK(\mathbf{r},t|\mathbf{r'},t')dt\nonumber\\&\approx i\hbar K(\mathbf{r},t'+\xi|\mathbf{r'},t')-\xi HK(\mathbf{r},t'|\mathbf{r'},t')\nonumber
\end{align}
In comparison, set $\langle\mathbf{r}|U(0)|\mathbf{r'}\rangle=K(\mathbf{r},t'|\mathbf{r'},t')$ and Taylor expand Eqn.~\ref{propagator} (note $U(0)=1$, $\xi=t-t'$):
\begin{align}
\langle\mathbf{r}|U(\xi)|\mathbf{r'}\rangle\Theta(\xi)&\approx\langle\mathbf{r}|U(0)|\mathbf{r'}\rangle+\xi\langle\mathbf{r}|\partial_tU(0)|\mathbf{r'}\rangle\nonumber\\&=\delta(\mathbf{r}-\mathbf{r'})U(0)-\xi\frac{i}{\hbar}\langle\mathbf{r}|HU(0)|\mathbf{r'}\rangle\nonumber\\&=\delta(\mathbf{r}-\mathbf{r'})-\xi\frac{i}{\hbar}\int\langle\mathbf{r}|H|\mathbf{r''}\rangle\langle\mathbf{r''}|U(0)|\mathbf{r'}\rangle d\mathbf{r''}\nonumber\\&=\delta(\mathbf{r}-\mathbf{r'})-\xi\frac{i}{\hbar}\int H\delta(\mathbf{r}-\mathbf{r''})\langle\mathbf{r''}|U(0)|\mathbf{r'}\rangle d\mathbf{r''}\nonumber\\&=\delta(\mathbf{r}-\mathbf{r'})-\frac{i\xi}{\hbar}H\langle\mathbf{r}|U(0)|\mathbf{r'}\rangle\nonumber
\end{align}
but $\langle\mathbf{r}|U(\xi)|\mathbf{r'}\rangle=K(\mathbf{r},t'+\xi|\mathbf{r'},t'\rangle$.
\end{enumerate}
\end{remarks}
\begin{prop}
The free particle propagator is
\begin{equation}
    K_{\text{free}}(\mathbf{r},t|\mathbf{r'},t')=\frac{\Theta(t-t')}{(4\pi\frac{i\hbar}{2m}(t-t'))^{3/2}}\exp\bigg(-\frac{|\mathbf{r}-\mathbf{r'}|^2}{4(i\hbar/2m)(t-t')}\bigg)\label{freeparticlepropagator}
\end{equation}
\end{prop}
\begin{proof}
To solve the differential equation $(i\hbar\partial_t+\frac{\hbar^2}{2m}\nabla^2)K_{\text{free}}=i\hbar\delta(t-t')\delta(\mathbf{r}-\mathbf{r'})$, we solve the heat diffusion equation for the local temperature $\theta(\mathbf{r},t)$. Consider
$$c\dot{\theta}=-\boldsymbol{\nabla}\cdot\mathbf{J_Q}+cS(\mathbf{r},t),\quad\mathbf{J_Q}=-\kappa\boldsymbol{\nabla}\theta$$
where $c$ is the specific heat capacity, $\mathbf{J_Q}$ is the heat current density, $\kappa$ is the thermal conductivity and $S$ is the heat source. The corresponding Green's function (heat kernel) satisfy
$$\bigg(\frac{\partial}{\partial t}-D\nabla^2\bigg)K_{\text{heat}}(\mathbf{r},t|\mathbf{r'},t')=\delta(\mathbf{r}-\mathbf{r'})\delta(t-t')$$
where $K_{\text{heat}}(\mathbf{r},t|\mathbf{r'},t')=0$ for $t<t'$. The general solution will then be $\theta(\mathbf{r},t)=\int K_{\text{heat}}(\mathbf{r},t|\mathbf{r'},t'')S(\mathbf{r'},t'')d\mathbf{r'}dt''$. This is a first order equation in time, so use $S$ to set initial conditions at time $t'$ - suppose $S(\mathbf{r'},t'')=\delta(t''-t')\theta(\mathbf{r'},t')$. Hence, $\theta(\mathbf{r},t)=\int K_{\text{heat}}(\mathbf{r},t|\mathbf{r'},t')\theta(\mathbf{r'},t')d\mathbf{r'}$. One can then verify the heat kernel is
$$K_{\text{heat}}(\mathbf{r},t|\mathbf{r'},t')=\frac{\Theta(t-t')}{(4\pi D(t-t'))^{3/2}}\exp\bigg(-\frac{|\mathbf{r}-\mathbf{r'}|^2}{4D(t-t')}\bigg)$$
By identifying $D$ as $i\hbar/2m$, we obtain our solution.
\end{proof}
\begin{remarks}[Momentum representation]
We could also express the free particle propagator in the momentum representation.
$$K_{\text{free}}(\mathbf{p},t|\mathbf{p'},t')=\Theta(t-t')\langle\mathbf{p}|U(t-t')|\mathbf{p'}\rangle=\Theta(t-t')\exp\bigg(-\frac{i}{\hbar}\frac{p^2}{2m}(t-t')\bigg)\langle\mathbf{p}|\mathbf{p'}\rangle$$
where we have
\begin{align}
K(\mathbf{r},t|\mathbf{r'},t')&=\Theta(t-t')\langle \mathbf{r}|U(t-t')|\mathbf{r'}\rangle\nonumber\\&=\Theta(t-t')\int\int\langle\mathbf{r}|\mathbf{p}\rangle\langle\mathbf{p}|U(t-t')|\mathbf{p'}\rangle\langle\mathbf{p'}|\mathbf{r'}\rangle d^3\mathbf{p}d^3\mathbf{p'}\nonumber\\&=
\int\int\frac{1}{(2\pi\hbar)^3}e^{i(\mathbf{p}\cdot\mathbf{r}-\mathbf{p'}\cdot\mathbf{r'})}K_{\text{free}} d^3\mathbf{p} d^3\mathbf{p'}=\frac{\Theta(t-t')}{(2\pi\hbar)^3}\int\exp\bigg(\frac{i\mathbf{p}\cdot(\mathbf{r}-\mathbf{r'})}{\hbar}-i\frac{p^2}{2m\hbar}(t-t')\bigg)d^3\mathbf{p}\nonumber\\&=\frac{\Theta(t-t')}{(2\pi\hbar)^3}\exp\bigg(-\frac{m(\mathbf{r}-\mathbf{r'})^2}{2i\hbar(t-t')}\bigg)\int\exp\bigg[\frac{-i(t-t')}{2m\hbar}\bigg(\mathbf{p}-m\frac{\mathbf{r}-\mathbf{r'}}{t-t'}\bigg)^2\bigg]d^3\mathbf{p}\nonumber\\&=\frac{\Theta(t-t')}{(2\pi\hbar)^3}\bigg(\frac{m\hbar}{i(t-t')}\bigg)^{3/2}(2\pi)^{3/2}\exp\bigg(-\frac{m(\mathbf{r}-\mathbf{r'})^2}{2i\hbar(t-t')}\bigg)\nonumber\\&=\Theta(t-t')\bigg(\frac{m}{2i\pi\hbar(t-t')}\bigg)^{3/2}\exp\bigg(-\frac{m(\mathbf{r}-\mathbf{r'})^2}{2i\hbar(t-t')}\bigg)\nonumber
\end{align}
where we completed the square, used $\langle\mathbf{r}|\mathbf{p}\rangle=\frac{1}{(2\pi\hbar)^{3/2}}e^{i\mathbf{p}\cdot\mathbf{r}/\hbar}$ and the gaussian integral $\int e^{-ax^2}dx=\sqrt{\frac{\pi}{a}}$. This provides a route to deriving the free particle propagator in the position representation.
\end{remarks}
\subsection{Path integral}
\begin{defi}[Path integral]
By using the property of the kernel (Eqn.~\ref{composition}), we can subdivide the evolution from time $t_i$ to $t_f$ into $N$ smaller intervals of length $\Delta t=(t_f-t_i)/N$, each characterized by its own propagator
$$K(\mathbf{r_f},t_f|\mathbf{r_i},t_i)=\int K(\mathbf{r_f},t_f|\mathbf{r_{N-1}},t_{N-1})\dots K(\mathbf{r_1},t_1|\mathbf{r_i},t_i)d\mathbf{r_1}\dots d\mathbf{r_{N-1}}$$
In the limit of small propagation intervals, the integration over the variables $\{\mathbf{r_i}\}$ becomes an integral over paths $\mathbf{r}(t)$ with a continuous index - time - rather than a discrete one. This is the path integral.
\end{defi}
\begin{prop}[Path integral]
\begin{equation}
    K(\mathbf{r_f},t_f|\mathbf{r_i},t_i)=\int_{\mathbf{r}(t_f)=\mathbf{r_f},\mathbf{r}(t_i)=\mathbf{r_i}}\exp\bigg(\frac{i}{\hbar}\int_{t_i}^{t_f}\frac{1}{2}m\mathbf{\dot{r}}^2-V(\mathbf{r}(t))dt\bigg)\mathcal{D}\mathbf{r}(t)\label{path}
\end{equation}
where $\mathcal{D}\mathbf{r}(t)=\prod_{n=1}^{N-1}(\frac{m}{2\pi i\hbar\Delta t})^{3/2}d\mathbf{r_n}$ is the volume element in the space of paths.
\end{prop}
\begin{proof}
With a constant potential $V(\mathbf{r},t)=V_0$, the free propagator Eqn.~\ref{freeparticlepropagator} becomes
$$K_{\text{free}}(\mathbf{r},t|\mathbf{r'},t')=\Theta(t-t')\bigg(\frac{m}{2\pi i\hbar(t-t')}\bigg)^{3/2}\exp\bigg[-\frac{m}{2i\hbar}\frac{(\mathbf{r}-\mathbf{r'})^2}{t-t'}-i\frac{V_0(t-t')}{\hbar}\bigg]$$
As $\Delta t\rightarrow 0$, $K$ is a localized function $\delta(\mathbf{r}-\mathbf{r'})$. The potential can be approximately evaluated at the midpoint $V(\mathbf{r_n})\approx V(\mathbf{r_{n-1}})\approx V(0.5(\mathbf{r_n}+\mathbf{r_{n-1}}))$. The kernel is
$$K(\mathbf{r_{n+1}},t_{n+1}|\mathbf{r_n},t_n)\approx\bigg(\frac{m}{2\pi i\hbar\Delta t}\bigg)^{3/2}\exp\bigg(-\frac{m|\mathbf{r_{n+1}}-\mathbf{r_n}|^2}{2i\hbar\Delta t}-\frac{i}{\hbar}V(0.5(\mathbf{r_{n+1}}-\mathbf{r_n}))\Delta t\bigg)$$
Taking the limit $\delta t\rightarrow 0$, the term $\frac{|\mathbf{r_{n+1}}-\mathbf{r_n}|^2}{\Delta t}\rightarrow|\mathbf{\dot{r}}|^2\Delta t$. After multiplying the $K$'s together, we have the sum of exponents to be $\frac{i}{\hbar}\int_{t_i}^{t_f}\frac{1}{2}m|\mathbf{\dot{r}}|^2-V(\mathbf{r})dt$.
\end{proof}
\begin{defi}[Classical action]
\begin{equation}
    S[\mathbf{r},t]=\int_{t_i}^{t_f}\mathcal{L}(\mathbf{r},\mathbf{\dot{r}})dt\label{action}
\end{equation}
For a free particle, the Lagrangian is $\mathcal{L}=\frac{1}{2}m|\mathbf{\dot{r}}|^2-V(\mathbf{r})$. We could also extend this to $V(\mathbf{r},t)$.
\end{defi}
\begin{remarks}\leavevmode
\begin{enumerate}
    \item By using position representation, no operators are involved, and we do not need to worry about operator ordering.
    \item This formalism is not useful for one-particle problems, but powerful way to describe quantum fields - functional integral over field configurations.
\end{enumerate}
\end{remarks}
\begin{eg}[Harmonic oscillator]
To find the classical action, we first obtain the trajectory that extremizes the Lagrangian:
$$\mathcal{L}=\frac{1}{2}m\dot{x}^2-\frac{1}{2}m\omega^2x^2\implies\frac{d}{dt}\frac{\partial\mathcal{L}}{\partial\dot{x}}=\frac{\partial\mathcal{L}}{\partial x}\implies\ddot{x}=-\omega^2x$$
This gives the general solution to be $x(t)\propto\sin(\omega t)$. To get from the initial point $(x_i,t_i)$ to the final point $(x_f,t_f)$, we have
$$x(t)=x_i\frac{\sin[\omega(t-t_f)]}{\sin[\omega(t_i-t_f)]}+x_f\frac{\sin[\omega(t-t_i)]}{\sin[\omega(t_f-t_i)]}=\frac{-x_i\sin[\omega(t-t_f)]+x_f\sin[\omega(t-t_i)]}{\sin[\omega(t_f-t_i)]}$$
This gives the Lagrangian to be
$$\mathcal{L}_{\text{SHO}}=\frac{m\omega^2}{2\sin^2[\omega(t_f-t_i)]}\bigg[x_f^2\cos[2\omega(t-t_i)]-x_i^2\cos[2\omega(t-t_f)]-2x_ix_f\cos[\omega(2t-t_i-t_f)]\bigg]$$
hence the action is
$$S_{\text{SHO}}[x_0(t)]=\frac{m\omega}{2\sin[\omega(t_f-t_i)]}\bigg[(x_i^2+x_f^2)\cos[\omega(t_f-t_i)]-2x_ix_f\bigg]$$
We guess the propagator has the form $K_{\text{SHO}}(x,t|x',t')=f(t)e^{iS_{\text{SHO}}(t)/\hbar}$. This must be a solution of
$$0=\bigg(i\hbar\frac{\partial}{\partial t}+\frac{\hbar^2}{2m}\frac{\partial^2}{\partial x^2}-\frac{1}{2}m\omega^2x^2\bigg)f(t)e^{i\frac{m\omega}{2\hbar}[(x_f^2+x_i^2)\cot[\omega(t_f-t_i)]-\frac{2x_ix_f}{\sin[\omega(t_f-t_i)]}}\implies \frac{\partial f(t)}{\partial t}=-\frac{\omega}{2}f(t)\frac{\cos\omega t}{\sin\omega t}$$
which gives $f(t)=\frac{C}{\sqrt{\sin\omega t}}$. Since $K_{\text{free}}\lim_{\omega\rightarrow0}K_{\text{SHO}}=\frac{C}{\sqrt{\omega t}}e^{i(m/2\hbar^2)(x_f^2-x_i^2)}\implies C=\sqrt{\frac{m\omega}{2\pi i\hbar}}$.
\end{eg}
\begin{remarks}[Semiclassical limit]
Essentially, all possible paths contribute to the propagator with some amplitude but phases determined by the integral of the action along the path. In the semiclassical limit ($\hbar\rightarrow0$), the dominant contributions to the propagator are paths close to the classical paths, which extremizes the action, i.e. $\delta\mathcal{L}=0$, which will add coherently, whereas those far from the classical path will destructively interfere.
\end{remarks}
\newpage
\subsection{JWKB semiclassical}
\begin{defi}[JWKB method]
For slowly varying potentials, the time-independent Schr\"{o}dinger equation can be solved in the JWKB approximation. Formally, it is accurate at the $\hbar\rightarrow 0$ limit, and when the local version of the de Broglie wavelength of the particle is small compared to the relevant length scales in the problem. 
\end{defi}
\begin{defi}[Local de Broglie wavelength and momentum]
At local position $x$, the local momentum is $p^2(x)=2m(E-V(x))$ where the local de Broglie wavelength is $\lambda(x)=\frac{h}{p(x)}$.
\end{defi}
\begin{remarks}
Take the operator $p^2=-\hbar^2\frac{d^2}{dx^2}$ on $\psi(x)$, we obtain $p^2\psi(x)=p^2(x)\psi$. This resembles an eigenvalue equation but it is not one.
\end{remarks}
\begin{defi}[Classically allowed and forbidden]
If we are in the classically allowed region, $E>V(x)$, $p^2(x)$ is positive:
\begin{equation}
    p^2(x)=2m(E-V(x))=\hbar^2k^2(x)\label{classicallyallowed}
\end{equation}
If we are in the classically forbidden region, $V(x)>E$, $p^2(x)$ is negative:
\begin{equation}
    -p^2(x)=2m(E-V(x))=\hbar^2\kappa^2(x)\label{classicallyforbidden}
\end{equation}
where $\kappa(x)\in\mathbb{R}^+$.
\end{defi}
\begin{defi}[Local phase]
Frequently, we write the wavefunctions in the polar form:
\begin{equation}
    \Psi(\mathbf{x},t)=\sqrt{\rho(\mathbf{x},t)}\exp\bigg(\frac{i}{\hbar}S(\mathbf{x},t)\bigg)\label{phase}
\end{equation}
where $\rho(\mathbf{x},t)=|\Psi(\mathbf{x},t)|^2$ and $S(\mathbf{x},t)$ are real.
\end{defi}
\begin{prop}
The current density $\mathbf{j}$ is perpendicular to the surfaces of constant phase.
\end{prop}
\begin{proof}
The current density is
$$\mathbf{j}=\frac{\hbar}{m}
\text{Im}[\Psi^*\boldsymbol{\nabla}\Psi]=\frac{\hbar}{m}\text{Im}\bigg[\frac{1}{2}\boldsymbol{\nabla}\rho+\frac{i}{\hbar}\rho\boldsymbol{\nabla}S\bigg]=\frac{\rho\hbar\boldsymbol\nabla S}{\hbar m}$$
where $\boldsymbol{\nabla}\Psi=\frac{\boldsymbol{\nabla}\rho}{2\sqrt{\rho}}e^{iS/\hbar}+\frac{i}{\hbar}\boldsymbol{\nabla}S~\Psi$.
\end{proof}
\begin{remarks}
By comparing $\mathbf{j}=\rho\frac{\boldsymbol{\nabla}S}{m}$ with $\mathbf{j}=\rho\mathbf{v}=\rho\frac{\mathbf{p}}{m}$ (classical fluid), we have $\mathbf{p}(\mathbf{x})\approx\boldsymbol{\nabla}S$, i.e. the gradient of the phase is like the classical local momentum.
\end{remarks}
\begin{eg}
For $\Psi(\mathbf{x},t)=e^{i(\mathbf{p}\cdot\mathbf{x}-Et)/\hbar}$, we can identify $S=\mathbf{p}\cdot\mathbf{x}-Et\implies\boldsymbol{\nabla}S=\mathbf{p}$.
\end{eg}
The goal is to solve the time-independent Schr\"{o}dinger's equation. But taking $\hbar\rightarrow 0$ directly is not useful. The starting solutions we hope to have, as the de Broglie wavelength goes to zero, are those representing plane waves (good approximations for very slowly varying potentials) $\psi\sim e^{ipx/\hbar}$.
\begin{thm}
The basic solution in the WKB approximation is
\begin{equation}
    \psi(x)=\frac{A}{\sqrt{p(x)}}\exp\bigg[\pm\frac{i}{\hbar}\int_{x_0}^xp(x')dx'\bigg]\label{WKB}
\end{equation}
\end{thm}
\begin{proof}
Consider $\psi(x)=\exp(iS(x)/\hbar)$ with $S(x)\in\mathbb{C}$. Plug into the time-independent Schr\"{o}dinger's equation:
$$p^2(x)e^{i S(x)/\hbar}=-\hbar^2\frac{d^2}{dx^2}(e^{iS(x)/\hbar})=-\hbar^2\bigg(\frac{iS''}{\hbar}-\frac{(S')^2}{\hbar^2}\bigg)e^{iS(x)/\hbar}$$
giving a non-linear equation for $S$: $(S'(x))^2-i\hbar S''(x)=p^2(x)$ (since $\psi$ was nonlinear). We will show if $V(x)$ is slowly varying, then the term $i\hbar S''$ is small: if $V(x)=V_0$ is a constant (effectively), then the local momentum $p(x)$ is equal to a constant $p_0$: $S'=p_0$, $S''=0$ (vanish identically). Consider the systematic expansion of $S(x)$:
$$S(x)=S_0(x)+\hbar S_1(x)+\hbar^2S_2(x)+O(\hbar^3)$$
Here, $S_0$ has units of $\hbar$, with the next successive correction having one less unit of $\hbar$. Plug this into the non-linear equation:
$$(S_0'+\hbar S_1'+\hbar^2S_2'+\dots)^2-i\hbar(S_0''+\hbar S_1''+\hbar^2S_2''+\dots)-p^2(x)=0$$
We want the LHS to vanish $\forall\hbar$, so the coefficient of each power of $\hbar$ vanishes. We have
$$(S_0')^2-p^2(x)+\hbar(2S_0'S_1'-iS_0'')+O(\hbar^2)=0$$
which gives $S_0'=\pm p(x)\implies S_0(x)=\pm\int_{x_0}^xp(x')dx'$, and $S_1'=\frac{iS_0''}{2S_0'}=\frac{ip'}{2p}\implies iS_1(x)=-0.5\ln p(x)+C'$. Hence, the wavefunction is
$$\psi(x)=\exp\bigg[\frac{i}{\hbar}(S_0+\hbar S_1+O(\hbar^2))\bigg]\approx e^{iS_0/\hbar}e^{iS_1}=\exp\bigg[\pm \frac{i}{\hbar}\int_{x_0}^xp(x')dx'\bigg]e^{-0.5\ln p(x)+C'}$$
The result follows.
\end{proof}
\begin{remarks}\leavevmode
\begin{enumerate}
    \item The probability density is $\rho=\psi^*\psi=\frac{|A|^2}{p(x)}$. The particle lingers in those regions where $p(x)$ is small and hence more likely to be found there.
    \item The probability current is    $j(x)=\frac{\rho}{m}\frac{\partial S}{\partial x}=\frac{|A|^2p(x)}{p(x)m}=\frac{|A|^2}{m}$. It is not possible for $j(x)$ to be position-dependent, otherwise it violates the current conservation equation.
    \item In the allowed region, the WKB solution is waves that propagate in opposite directions. On the other hand, in the forbidden region, the WKB solution is a superposition of a growing and decaying exponential.
    $$\psi|_{E>V(x)}=\frac{A}{\sqrt{k(x)}}e^{i\int_{x_0}^xk(x')dx'}+\frac{B}{\sqrt{k(x)}}e^{-i\int_{x_0}^xk(x')dx'},\quad \psi|_{E<V(x)}=\frac{C}{\sqrt{\kappa(x)}}e^{-\int_{x_0}^x\kappa(x')dx'}+\frac{D}{\sqrt{\kappa(x)}}e^{+\int_{x_0}^x\kappa(x')dx'}$$
\end{enumerate}
\end{remarks}
\begin{eg}[Linear potential]
Consider the linear potential $V(x)=gx$ with $g>0$, i.e. increasing strictly $\forall x$ and unbounded from below. We study non-normalizable energy eigenstates with energy $E$. At the classical turning point $x=a$, $V(x)=E=g(x-a)\implies a=E/g$. The dimensionless Schr\"{o}dinger's equation becomes ($x=L\tilde{u}$) the Airy differential equation:
$$-\frac{\hbar^2}{2m}\frac{1}{gL^3}\frac{d^2\psi}{d\tilde{u}^2}+\bigg(\tilde{u}-\frac{a}{L}\bigg)\psi=0\implies\frac{d^2\psi}{du^2}=u\psi$$
where we set $L^3=\frac{\hbar^2}{2mg}$ and $u=\tilde{u}-\frac{a}{L}=\frac{1}{L}(x-a)$. The WKB solutions are not expected to hold near the turning point $u=0$ (equivalently, $x=a$) because near turning points, the local momentum goes to zero and the de Broglie wavelength goes to infinity. For $u>>1$, we set $\kappa=\sqrt{u}$:
\begin{equation}
\psi(u>>1)=\frac{C}{u^{1/4}}e^{-\int_{u_0}^u\sqrt{u'}du'}+\frac{D}{u^{1/4}}e^{\int_{u_0}^u\sqrt{u'}du'}=\frac{C}{u^{1/4}}e^{-2u^{3/2}/3}+\frac{D}{u^{1/4}}e^{2u^{3/2}/3}\label{result1}
\end{equation}
where w.l.o.g, we choose $u_0=0$. Similarly, for $u<<-1$, set $k=\sqrt{-u}=|u|^{1/2}$. Choose the limits of integration so that the value of the lower limit is smaller than the value of the upper limit.
\begin{equation}
\psi(u<<-1)=\frac{A}{|u|^{1/4}}e^{i2|u|^{3/2}/3}+\frac{B}{|u|^{1/4}}e^{-i2|u|^{3/2}/3}\label{result2}
\end{equation}
We will see later that we need to connect the two solutions $u>>1$ and $u<<-1$.
\end{eg}
\begin{remarks}[Validity]
We require the $O(\hbar)$ terms in the non-linear differential equation of $S$ to be much smaller, in magnitude, than the $O(1)$ terms. Since at each of these orders, we have two terms that are set equal to each other by the differential equations. It therefore suffices to check that one of the $O(\hbar)$ terms is much smaller than one of the $O(1)$ terms, e.g. $|\hbar S_0'S_1'|<<|S_0'|^2\implies|p|>>|\hbar S_1'|=|\hbar p'/p|$, with $|S_0'|=|p|$. In another words, the change in the local momentum over a distance equal to the de Broglie wavelength are small compared to the momentum. Equivalently,
\begin{equation}
    \bigg|\frac{d\lambda}{dx}\bigg|<<1\label{validityWKB}
\end{equation}
Alternatively, with $p^2=2m(E-V(x))$, we have $|\lambda(x)V'(x)|<<p^2/2m$, i.e. the precise meaning of a slowly changing potential.
\end{remarks}
\begin{eg}
Under general conditions, sufficiently near a turning point the potential $V(x)$ is approximately linear, i.e. $V(x)-E=g(x-a)$, $g>0$. In the allowed region $x<a$, the local momentum is $p^2(x)=2m(E-V(x))=2mg(a-x)$. The rate of change of de Broglie wavelength is
$$\bigg|\frac{d\lambda}{dx}\bigg|=\bigg|\frac{d}{dx}\frac{2\pi\hbar}{\sqrt{2mg}\sqrt{a-x}}\bigg|=\frac{\pi\hbar}{\sqrt{2mg}}\frac{1}{(a-x)^{3/2}}$$
which goes to infinity as $x\rightarrow a$, violating the key condition for WKB as we approach the turning points. Our basic WKB solutions Eqn.~\ref{WKB} can be valid only as we remain away from the turning points.
\end{eg}
We thus require the connection formula to form a single solution.
\begin{thm}[Connection formula]
Suppose the turning point $x=a$ separates a classically allowed region to the left and a classically forbidden region to the right. WKB solutions to the right are exponentials while to the left are oscillatory functions.
\begin{equation}
    \frac{2}{\sqrt{k(x)}}\cos\bigg(\int_x^ak(x')dx'-\frac{\pi}{4}\bigg)\leftarrow\frac{1}{\sqrt{\kappa(x)}}\exp\bigg(-\int_a^x\kappa(x')dx'\bigg)\label{connection1}
\end{equation}
i.e. a pure decaying exponential to the right of $x=a$ will reliably suggest the solution to the left of $x=a$ is a phase-shifted cosine function.
\begin{equation}
    \frac{-1}{\sqrt{k(x)}}\sin\bigg(\int_x^ak(x')dx'-\frac{\pi}{4}\bigg)\rightarrow\frac{1}{\sqrt{\kappa(x)}}\exp\bigg(\int_a^x\kappa(x')dx'\bigg)\label{connection2}
\end{equation}
i.e. a phase-shifted sine function to the left of $x=a$ will reliably suggest the solution to the right of $x=a$ is a growing exponential.
\end{thm}
To derive the connection formula, we require the asymptotic expansion of the Airy functions. Before that, we discuss an example.
\begin{eg}
Consider a monotonically increasing potential $V(x)$ (increases without bound) that includes a hard wall at $x=0$. Let $E$ denote the energy of our searched-for eigenstate, which determines the turning point $x=a$. The solution for $x>a$ must only have a decaying exponential since the forbidden region extends forever of the right of $x=a$. The left of $x=a$ will be
$$\psi(0\leq x<<a)=\frac{1}{\sqrt{k(x)}}\cos\bigg(\int_x^a k(x')dx'-\frac{\pi}{4}\bigg)$$
The hard wall condition gives $\psi(0)=0\implies\int_{x=0}^ak(x')dx'-\frac{\pi}{4}=\frac{\pi}{2}+n\pi$ with $n\in\mathbb{Z}$, i.e. a quantization condition. Negative integers are not allowed since the LHS is manifestly positive. The wavefunction is thus
$$\psi(x)=\frac{1}{\sqrt{k(x)}}\cos\bigg(\int_0^ak(x')dx'-\frac{\pi}{4}-\int_0^xk(x')dx'\bigg)=\frac{\sin(\int_0^ak(x')dx'-\frac{\pi}{4})}{\sqrt{k(x)}}\sin\bigg(\int_0^xk(x')dx'\bigg)$$
where we used the hard wall condition when we expanded the cosine of a sum of angles. The quantization condition indicates that the excursion of the phase in the solution from 0 to $a$ is a bit higher than $n\pi$ but less than $(n+1)\pi$, producing the $n$ nodes in an $n$th excited state, as expected.
\end{eg}
\begin{lemma}
The asymptotic expansion for the Airy functions (solutions to the Airy differential equation) are
\begin{equation}
    \text{Ai}[u>>1]\approx\frac{1}{2\sqrt{\pi}u^{1/4}}e^{-2u^{3/2}/3},\quad    \text{Ai}[u<<-1]\approx\frac{1}{\sqrt{\pi}|u|^{1/4}}\cos\bigg(\frac{2}{3}|u|^{3/2}-\frac{\pi}{4}\bigg)\label{Airy1}
\end{equation}
\begin{equation}
    \text{Bi}[u>>1]\approx\frac{1}{\sqrt{\pi}u^{1/4}}e^{2u^{3/2}/3},\quad    \text{Bi}[u<<-1]\approx\frac{-1}{\sqrt{\pi}|u|^{1/4}}\sin\bigg(\frac{2}{3}|u|^{3/2}-\frac{\pi}{4}\bigg)\label{Airy2}
\end{equation}
\end{lemma}
\begin{proof}
We solve for the Airy differential equation $\frac{d^2\psi}{du^2}=u\psi$ via an integral solution $\psi(u)=\int_\Gamma\tilde{\psi}(k)e^{iku}\frac{dk}{2\pi}$. This is a Fourier transform only if $\Gamma$ is the real full line. Both $k$ and $u$ here are unit free ($k$ is not the local wavevector). This gives separately
$$\frac{d^2\psi}{du^2}=\int_\Gamma(-k^2\tilde{\psi}(k))e^{iku}\frac{dk}{2\pi},\quad u\psi=\int_\Gamma\tilde{\psi}(k)\frac{1}{i}\frac{d}{dk}e^{iku}\frac{dk}{2\pi}=\frac{1}{2\pi i}[\tilde{\psi}(k)e^{iku}]_{k_-}^{k_+}-\int_\Gamma\frac{1}{i}\frac{d\tilde{\psi}}{dk}e^{iku}\frac{dk}{2\pi}$$
If $\psi(u)$ is indeed a solution of the Airy differential equation, then we require
$$-k^2\tilde{\psi}(k)+\frac{1}{i}\frac{d\tilde{\psi}}{dk}=0\implies\tilde{\psi}(k)=e^{ik^3/3},\quad\tilde{\psi}(k_+)e^{ik_+u}=0,~\tilde{\psi}(k_-)e^{ik_-u}=0$$
where the boundary terms must separately vanish (they cannot cancel each other otherwise we must have $k_-=k_+$ in order for $u$ to be independent, making the contour closed and can be deformed/shrinked to a point since there are no poles). The term $e^{ik^3/3}$ will vanish as $|k|\rightarrow\infty$ if $k^3$ has a positive imaginary part $\text{Im}[k^3]>0$, contributing a large negative number in the exponent of $e^{ik^3/3}$. Equivalently, by writing $k=|k|e^{i\theta_k}$:
$$\text{Im}[k^3]>0\implies 0<3\theta_k<\pi$$
Further, if $\text{Im}[k^3]>0$ for some $\theta_k$, it is also positive for $\theta_k+\frac{2\pi}{3}$. This demarcates three regions in complex $k$-space. If $k_+$ or $k_-$ approach infinity in these regions, the boundary terms will vanish and we have a solution to the differential equation. The boundary lines of these sectors corresponds to $\text{Im}[k^3]=0$ and are marginally acceptable for $k_\pm$. To obtain the first solution, take $\Gamma$ to be the contour $C_1$ running all over the real line:
$$\text{Ai}[u]=\int_{-\infty}^\infty e^{ik^3/3}e^{iku}\frac{dk}{2\pi}=\frac{1}{\pi}\int_0^\infty\cos(k^3/3+ku)dk$$
To obtain another solution, we need another contour that cannot be deformed into $C_1$. Let $C_2$ goes from $-\infty$ to zero, right above the real axis, and then goes down from zero along the negative imaginary axis. 
$$\text{Bi}[u]=-i\int_{C_1}e^{ik^3/3}e^{iku}\frac{dk}{2\pi}+2i\int_{C_2}e^{ik^3/3}e^{iku}\frac{dk}{2\pi}=\frac{1}{\pi}\int_0^\infty (e^{-k^3/3}e^{ku}+\sin(k^3/3+ku))dk$$
To find the asymptotic expansion, we do a stationary phase approximation to the integral representation. For $\text{Ai}[u]$, the phase $\phi(k)=(k^3/3)+ku$ is minimized at $k=\pm i\sqrt{u}$ for $u>0$. It is convenient to move the contour of integration so that it goes through this stationary phase point (this shift/deformation is possible since the integrand has no poles in the upper half plane and vanishes at the endpoints as long as we stay in the demarcated regions). The new contour is parametrized with a new real variable $\tilde{k}=k-i\sqrt{u}\in(-\infty,\infty)$. On this contour, the integral becomes
$$\psi(u)=\int_{-\infty}^\infty e^{-(2/3)u^{3/2}-\sqrt{u}\tilde{k}^2+i\tilde{k}^3/3}\frac{d\tilde{k}}{2\pi}=e^{-2u^{3/2}/3}\int_{-\infty}^\infty e^{-\sqrt{u}\tilde{k}^2+i\tilde{k}^3/3}\frac{d\tilde{k}}{2\pi}$$
where $\phi(k)=\frac{2}{3}iu^{3/2}+i\sqrt{u}\tilde{k}^2+\frac{1}{3}\tilde{k}^3$. For large $u$, the suppression created by $-\sqrt{u}\tilde{k}^2$ term implies we can ignore the $i\tilde{k}^3$ term, giving us the Gaussian integral $\psi(u)\approx\frac{1}{2\pi}e^{-2u^{3/2}/3}\frac{\sqrt{\pi}}{u^{1/4}}$.\\[5pt]
Similarly, when $u$ is negative and large, the stationary phase condition gives two real roots, hence the contour can stay along the real axis. Doing the integration directly gives the result. Similar for $\text{Bi}[u]$.
\end{proof}
\begin{remarks}
The integrand in Ai$[u]$ oscillates faster and faster and fails to contribute, giving a finite result as $k\rightarrow\infty$. $\text{Ai}[u]$ is oscillatory for negative $u$ and decays to zero rapidly for $u>0$. $\text{Bi}[u]$ is defined so that it is oscillatory for $u<0$, but rather than decay rapidly for $u>0$, it grows without bound.
\end{remarks}
The above asymptotic expansion of the Airy functions was necessary to prove the connection formulae.
\begin{proof}
Sufficiently near $x=a$, any potential is approximately linear, $V\approx gx$ and again set $x>>a$ to be classically forbidden and $x<<a$ to be classically allowed. Using the result Eqns.~\ref{result1},~\ref{result2}, we have
$$\psi_R(u>>1)=\frac{C}{u^{1/4}}e^{-2u^{3/2}/3}+\frac{D}{u^{1/4}}e^{2u^{3/2}/3},\quad\psi_L(u<<-1)=\frac{A}{|u|^{1/4}}\cos\bigg(\frac{2}{3}|u|^{3/2}-\frac{\pi}{4}\bigg)+\frac{B}{|u|^{1/4}}\sin\bigg(\frac{2}{3}|u|^{3/2}-\frac{\pi}{4}\bigg)$$
where we wrote the exponentials in sines and cosines for the left. The shift of $\pi/4$ is for convenience. From Eqn.~\ref{Airy1}, we match $C$ to $A$ via $C=0.5A$. Similarly, from Eqn.~\ref{Airy2}, we match $B$ to $D$ via $D=-B$. Put all together, replacing $C$ by $A$ and $A$ by $2A$, as well as $B$ by $-B$.\\[5pt]
Suppose for $x>>a$, we have a decaying exponential, then the matching gives a cosine curve to the left. On the other hand, suppose we have the object (cosine) to the left, there is always a possibility for a sine wave with a tiny coefficient, but this leads incorrectly to an overwhelming growing exponential on the right $x>>a$. Similar for the second relation. A sine can only lead to a growing exponential.
\end{proof}
\begin{eg}[Tunnelling through a barrier]
We apply the WKB approximation to the tunnelling probability $T$ for a wave of energy $E$ incident on a smoothly varying wide barrier. Here, $E$ is smaller than the height of the potential so we expect $T$ to be small, giving the reflected wave to be as large as the incoming wave, i.e. $|B|\approx |A|$. For $x>>b$, there is just an outgoing wave with amplitude controlled by $F$. Within the barrier, the component that decays as $x$ grows is more relevant than the component that grows. The transmitted right-moving wave is
$$\psi_{\text{tr}}(x>>b)=\frac{F}{\sqrt{k(x)}}e^{i\int_b^xk(x')dx'-i\pi/4}=\frac{F}{\sqrt{k(x)}}\cos\bigg(\int_b^xk(x')dx'-\frac{\pi}{4}\bigg)+\frac{iF}{\sqrt{k(x)}}\sin\bigg(\int_b^xk(x')dx'-\frac{\pi}{4}\bigg)$$
Match the second term to an exponential that grows as we move to the left of $x=b$:
$$\psi(x)_{\text{barr}}(a<<x<<b)=\frac{-iF}{\sqrt{\kappa(x)}}\exp\bigg(\int_x^b\kappa(x')dx'\bigg)$$
If we attempted to match instead the first term, we would have gotten an exponential decay as we move to the left of $x=b$, which is overwhelmed by the growing exponential. We can now refer this solution to the point $x=a$:
$$\psi(a<<x<<b)_{\text{barr}}=\frac{-iF}{\sqrt{\kappa(x)}}\exp\bigg(\int_a^b\kappa(x')dx'-\int_a^x\kappa(x')dx'\bigg)=\frac{-iFe^\theta}{\sqrt{\kappa(x)}}\exp\bigg(-\int_a^x\kappa(x')dx'\bigg)$$
where $\theta:=\int_a^b\kappa(x')dx'$. Since this is a decaying exponential to the right of $x>a$, we can connect it to a solution to the left of $x<a$ using
$$\psi(x)=-\frac{2iFe^\theta}{\sqrt{k(x)}}\cos\bigg(\int_x^ak(x')dx'-\frac{\pi}{4}\bigg)$$
which is a superposition of a wave incident $\psi_{\text{inc}}(x)=-\frac{iFe^{\theta}}{\sqrt{k(x)}}\exp(-i\int_x^ak(x')dx'+i0.25\pi)$ and a wave reflected from the barrier. The transmission coefficient is then $T=\frac{|F|^2}{|-iFe^\theta|^2}=e^{-2\theta}$ giving exponential suppression, as expected.
\end{eg}
\begin{eg}
For a rectangular barrier of $V_0$ large for $|x|<a$ and zero otherwise, we have
$$T_{\text{wkb}}=\exp\bigg(-2\int_{-a}^a\sqrt{\frac{2m}{\hbar^2}(V_0-E)}dx'\bigg)=e^{-4(a/\hbar)\sqrt{2m(V-0-E)}}\approx e^{-4z_0\sqrt{1-(E/V_0)}}$$
where $z_0^2=\frac{2mV_0a^2}{\hbar^2}>>1$.
\end{eg}
\newpage
\section{Scattering Theory}
\subsection{Scattering in 1D}
Consider scattering from the left (rightward-moving state): look for solutions of the asymptotic form
$$\psi_R(x)\sim
\left\{
        \begin{array}{ll}
      e^{ikx}+re^{-ikx} & x\rightarrow-\infty \\
      te^{ikx} & x\rightarrow+\infty
        \end{array}
    \right.$$
    where $r,t\in\mathbb{C}$. The probabilities of reflection and transmission are $R=|r|^2$ and $T=|t|^2$ respectively. $J(x)=-\frac{i\hbar}{2m}(\psi^*\partial_x\psi-\psi\partial_x\psi^*)$ obeys the conservation law $\frac{d}{dx}J(x)=0$. At $x\rightarrow-\infty$ and $x\rightarrow+\infty$, $J(x)=\frac{\hbar k}{2m}(1-|r|^2)$ and $J(x)=\frac{\hbar k}{2m}|t|^2$ respectively. Conservation of $J(x)$ require $|r|^2+|t|^2=1$. Now consider scattering from the right. Look for asymptotic solutions of the form
$$\psi_L(x)\sim
\left\{
        \begin{array}{ll}
      e^{-ikx}+r'e^{ikx} & x\rightarrow+\infty \\
      t'e^{-ikx} & x\rightarrow-\infty
        \end{array}
    \right.$$
    where $r',t'\in\mathbb{C}$. Consider
$$\psi_R^*(x)-r^*\psi_R\sim
\left\{
        \begin{array}{ll}
      (1-|r|^2)e^{-ikx} & x\rightarrow-\infty \\
      t^*e^{-ikx}-r^*te^{ikx} & x\rightarrow+\infty
        \end{array}
    \right.$$    
    This is also a solution of the form $\psi_L$. Comparing coefficients gives $r'=r^*\frac{t}{t^*}$ and $t'=t$. Here, $R=|r|^2=|r'|^2$.
\begin{eg}
Consider the square potential with $V(-a/2<x<a/2)=-V_0<0$ and 0 everywhere else. For $|x|>\frac{a}{2}$, $\phi=e^{\pm iKx}$ and for $|x|<\frac{a}{2}$, $\phi=Ae^{iqx}+Be^{-iqx}$ where $q^2=\frac{2mV_0}{\hbar^2}+K^2$. $r=0$ for some incoming momentum $q$. This requires $q$ to be quantized, i.e. $qa=n\pi$, $n\in\mathbb{Z}$. Look for solutions of the form $\psi_R$. Match by demanding continuity of $\psi$ and $\psi'$ at $\pm\frac{a}{2}$.
$$e^{-ika/2}+re^{ika/2}=Ae^{-iqa/2}+Be^{iqa/2},\quad te^{ika/2}=Ae^{iqa/2}+Be^{-iqa/2}$$
$$\frac{k}{q}(e^{-ika/2}-re^{ika/2})=Ae^{-iqa/2}+Be^{iqa/2},\quad\frac{k}{q}te^{ika/2}=Ae^{iqa/2}-Be^{-iqa/2}$$
Eliminate $A$ and $B$ and we obtain
$$r=\frac{(k^2-q^2)\sin(qa)e^{-ika}}{(q^2+k^2)\sin(qa)+2iqk\cos(qa)},\quad t=\frac{2iqke^{-ika}}{(q^2+k^2)\sin(qa)+2iqk\cos(qa)}$$
where $\lim_{k\rightarrow\infty}r=0$ and $\lim_{k\rightarrow0}r=-1$.
\end{eg}
\begin{defi}[S-matrix]
Define ingoing right (left) moving wave to be $I_R=e^{ikx}$ for $x\rightarrow-\infty$ ($I_L=e^{-ikx}$ for $x\rightarrow+\infty$) and outgoing right (left) moving wave to be $O_R=e^{ikx}$ for $x\rightarrow+\infty$ ($O_L=e^{-ikx}$ for $x\rightarrow-\infty$).  Asymptotically, the solutions take the form
$$\begin{pmatrix}\psi_R\\\psi_L\\\end{pmatrix}=\begin{pmatrix}I_R\\I_L\\\end{pmatrix}+S\begin{pmatrix}O_R\\O_L\\\end{pmatrix},\quad S=\begin{pmatrix}t&r\\r'&t'\\\end{pmatrix}$$
\end{defi}
\begin{prop}
$S$-matrix is unitary.
\end{prop}
\begin{proof}
Follows from flux conservation, i.e. $SS^\dag=\begin{pmatrix}|t|^2+|r|^2&tr'^*+rt'^*\\r't^*+t'r^*&|t'|^2+|r'|^2\\\end{pmatrix}=\Id$.
\end{proof}
\begin{eg}
For $V(x)=g\delta(x)$, $\psi$ and $\psi'$ are continuous and discontinuous respectively.
$$\int_{-\varepsilon}^\varepsilon-\frac{\hbar^2}{2m}\frac{\partial^2\psi}{\partial x^2}+g\delta(x)\psi dx=\int_{-\varepsilon}^\varepsilon E\psi dx\implies-\frac{\hbar^2}{2m}\bigg[\frac{\partial\psi}{\partial x}\bigg|_\varepsilon-\frac{\partial\psi}{\partial x}\bigg|_{-\varepsilon}\bigg]+g\psi(0)=0$$
for the latter. One can show the S-matrix has $t=t'=\frac{g}{i\hbar^2k/m-g}$ and $r=r'=\frac{i\hbar^2k/m}{i\hbar^2k/m-g}$.
\end{eg}
\begin{defi}[Parity]
For symmetric potentials $V(x)=V(-x)$, $\psi_L(x)=\psi_R(-x)$. The asymptotic eigenfunctions of $\mathcal{P}$:
$$\psi_\pm(x)=\psi_R(x)\pm\psi_L(x)=\psi_R(x)\pm\psi_R(-x)\implies\mathcal{P}\psi_\pm(x)=\psi_R(-x)\pm\psi_R(x)=\pm\psi_\pm(x)$$
\end{defi}
\begin{eg}
Consider $V(x)$ in Example 3.1. The even (+) and odd (-) parity eigenfunctions are
$$\psi_\pm(x)=A_\pm(e^{iqx}\pm e^{-iqx})$$
for $x\in\{-\frac{a}{2},\frac{a}{2}\}$ and recall $q^2=k^2+\frac{2mV_0}{\hbar^2}$. For the even parity, match with asymptotics at $x=\pm\frac{a}{2}$:
$$e^{-ika/2}+(r+t)e^{ika/2}=A_+(e^{iqa/2}+e^{-iqa/2}),\quad-e^{-ika/2}+(r+t)e^{ika/2}=\frac{q}{k}A_+(e^{iqa/2}-e^{-iqa/2})$$ 
This results in $r+t=-e^{-ika}\frac{q\tan(qa/2)-ik}{q\tan(qa/2)+ik}$.  As for odd parity,
$$e^{-ika/2}+(r-t)e^{ika/2}=A_-(e^{iqa/2}-e^{-iqa/2}),\quad-e^{-ika/2}+(r-t)e^{ika/2}=\frac{q}{k}A_-(e^{iqa/2}+e^{-iqa/2})$$ 
This results in $r-t=e^{-ika}\frac{q+ik\tan(qa/2)}{q-ik\tan(qa/2)}$.
\end{eg}
\begin{defi}[Parity states]
In general, define new incoming and outgoing asymptotic states:
$$\text{incoming: }
\left\{
        \begin{array}{ll}
      \mathcal{I}_+(x)=e^{-ik|x|} & \text{ even } \\
      \mathcal{I}_-(x)=\sgn(x)e^{-ik|x|} & \text{ odd } 
        \end{array}
    \right.\quad\text{outgoing: }
\left\{
        \begin{array}{ll}
      \mathcal{O}_+(x)=e^{ik|x|} & \text{ even } \\
      \mathcal{O}_-(x)=-\sgn(x)e^{ik|x|} & \text{ odd } 
        \end{array}
    \right.$$
These are related to other states as
$$\begin{pmatrix}\mathcal{I}_+(x)\\\mathcal{I}_-(x)\\\end{pmatrix}=M\begin{pmatrix}I_R\\I_L\\\end{pmatrix},\quad\begin{pmatrix}\mathcal{O}_+(x)\\\mathcal{O}_-(x)\\\end{pmatrix}=M\begin{pmatrix}O_R\\O_L\\\end{pmatrix},\quad M=\begin{pmatrix}1&1\\-1&1\\\end{pmatrix}$$
The S-matrix with respect to the parity basis $S^P$ satisfies
$$\begin{pmatrix}\psi_+\\\psi_-\\\end{pmatrix}=\begin{pmatrix}\mathcal{I}_+\\\mathcal{I}_-\\\end{pmatrix}+S^p\begin{pmatrix}\mathcal{O}_+\\\mathcal{O}_-\\\end{pmatrix}\implies S^P=MSM^{-1}=\begin{pmatrix}t+0.5(r+r')&0.5(-r'+r)\\0.5(r'-r)&t-0.5(r+r')\\\end{pmatrix}$$
\end{defi}
\begin{prop}
For symmetric potentials, the S-matrix is diagonal. 
\end{prop} 
\begin{proof}
$V(x)=V(-x)\implies r=r'$ and so $S_{++}=r+t$, $S_{--}=-r+t$ and $S_{+-}=S_{-+}=0$.
\end{proof}
\begin{remarks}
Observe $|S_{++}|^2=|S_{--}|^2=1$, then one can write these as relative phases $S_{\pm\pm}=e^{2i\delta_\pm(k)}$, where $\delta_\pm\in\mathbb{R}$ is the phase shift. The S-matrix encodes the properties of the symmetric potential. There are thus two scattering channels (even/odd) which do not intermix.
\end{remarks}
\begin{prop}
The poles of the $S$-matrix in the complex $k$ plane, that lie on the positive imaginary axis, correspond to the bound states.
\end{prop}
\begin{proof}
The trick is to analytically continue $S(k)$ to $k\in\mathbb{C}$. Assume symmetric potentials $V(x)=V(-x)$. Consider the even parity states $\psi_+(x)=\mathcal{I}_++S_{++}\mathcal{O}_+$. Set $k=i\lambda$ for $\lambda>0$. We get a solution with asymptotes 
$$\psi_+(x)=
\left\{
        \begin{array}{ll}
      e^{ikx}+S_{++}e^{-ikx}=e^{-\lambda x}+S_{++}e^{+\lambda x}& x\rightarrow-\infty \\
      e^{-ikx}+S_{++}e^{ikx}=e^{\lambda x}+S_{++}e^{-\lambda x}& x\rightarrow+\infty  
        \end{array}
    \right.$$
We can multiply $S_{++}^{-1}$ to $\psi_+(x)$, then up to a multiplicative phase, $\psi_x=S_{++}^{-1}e^{\lambda x}+e^{-\lambda x}$ for $x\rightarrow\infty$. If $S_{++}^{-1}$ decays as quickly, the term might be able to negate the divergent effects of $e^{\lambda x}$. The wavefunction is normalizable when $S_{++}(k)\rightarrow+\infty$ for $k\rightarrow i\lambda$, i.e. the poles of the S-matrix in complex momentum plane that lie on the positive imaginary axis corresponds to the bound states. Their energy is $E=\frac{\hbar^2k^2}{2m}=-\frac{\hbar^2\lambda^2}{2m}<0$. If $k=-i\lambda$, $\lambda>0$ instead, then for the wavefunction to be normalizable, $S_{++}(k=-i\lambda)=0$. This gives the same bound states. Similar for $S_{--}(k)$.
\end{proof}
\begin{eg}
Consider Example 3.1: set $k=i\lambda$, $S_{++}(k=i\lambda)$ has a pole if $\lambda=q\tan(a/2)$ with $\lambda^2+q^2=\frac{2mVa}{\hbar^2}$. There is always at least one bound state, of even parity. 
\end{eg}
\newpage
\begin{eg}
Consider the potential:
\begin{center}
\begin{tikzpicture}
      \draw[->] (-4,0) -- (4,0) node[right] {$x$};
      \draw[->] (0,0) -- (0,2) node[left] {$V(x)$};
      \draw[domain=-4:4,smooth,variable=\x,black] plot ({\x},{exp(-(abs(\x)-1.4)^2)});
      \draw (0,0) node[below]{0};
\end{tikzpicture}
\end{center}
There are no bound states here because $E>0$. Instead, a quantum particle localized around $x=0$ will evolve such that the wavefunction extends beyond the barrier. This is quantum tunnelling. This occurs when the S-matrix has a pole at $k=k_0-i\gamma$ with $\gamma>0$. The corresponding energy is
$$E=\frac{\hbar^2k^2}{2m}=E_0-i\frac{\Gamma}{2},\quad E_0=\frac{\hbar^2}{2m}(k_0^2-\gamma^2),~\Gamma=\frac{2\hbar\gamma k_0}{m}$$
The time dependence of wavefunction $e^{-iEt/\hbar}=e^{-iE_0t/\hbar}e^{-\Gamma t/2\hbar}$ has an exponentially decaying envelope - characteristic of an unstable state, resonance. $\Gamma$ is the width of the state and $\tau=\frac{\hbar}{\Gamma}$ is the half-life.
\end{eg}
\begin{remarks}\leavevmode
\begin{enumerate}
\item Suppose there is a pole in $S_{++}(k)$. The same argument as for the bound state and find the asymptotic form
$$\psi_+(x,t)\sim
\left\{
        \begin{array}{ll}
      e^{-ikx}e^{-iEt/\hbar}=e^{-iE_0t/\hbar}e^{-ik_0x}e^{-\gamma x-\frac{\Gamma t}{2\hbar}} & x\rightarrow-\infty \\
      e^{+ikx}e^{-iEt/\hbar}=e^{-iE_0t/\hbar}e^{ik_0x}e^{\gamma x-\frac{\Gamma t}{2\hbar}} & x\rightarrow+\infty
        \end{array}
    \right.$$
The non-oscillating factor is $e^{\pm\gamma(x\mp vt)}$ with $v=\frac{\Gamma}{2\hbar\gamma}=\frac{\hbar k_0}{m}$. This has the interpretation of an escaped particle moving with velocity $v$.
\item 
$\psi$ is an eigenstate of $H$ with complex eigenvalue. It's because $\psi$ is not normalizable. It is not a stationary state in the Hilbert space. We can view $S_{++}$ as a function of energy $E=\frac{\hbar^2k^2}{2m}$. At resonance,
$$S_{++}(E)=\frac{E-(E_0+i\Gamma/2)}{E-(E_0-i\Gamma/2)}$$
\end{enumerate}
\end{remarks}
\subsection{Scattering in 3D}
For simplicity, we will consider elastic scattering between spinless non-relativistic particles where none of the particles internal states change in the collision. The interaction potential is translational invariant, and taken to have finite range (vanishes for $r>a$ for some constant $a$, otherwise vanish as $r\rightarrow\infty$ faster than $1/r$).
\subsubsection{Classical scattering}
\begin{defi}[Differential and total cross-section]
Let $\theta$ be the scattering angle, a function of impact parameter $b$ and energy $E$. Now consider a uniform beam of particles. Particles in a cross-section $d\sigma=bdbd\phi$ will scatter with a solid angle $d\Omega=\sin\theta d\theta d\phi$.
The differential cross-section is
$$\frac{d\sigma}{d\Omega}=\frac{bdbd\phi}{\sin\theta d\phi d\theta}$$
which is also the ratio of scattered flux per solid angle (number of particles scattered per unit time per solid angle $d\Omega$ about $(\theta,\phi)$) to incident flux (number of particles per unit area and per unit time). This depends on $E$ and $\theta$. The total cross-section is $\sigma_T=\int\frac{d\sigma}{d\Omega}d\Omega =\int\frac{d\sigma}{d\Omega}d\Omega$. Both have dimensions of area.
\end{defi}
\begin{eg}[Hard Sphere]
$$V(r)\sim
\left\{
        \begin{array}{ll}
      \infty & r\leq R \\
      0 & r>R
        \end{array}
    \right.$$
We have $\theta=\pi-2\alpha$, $b=R\sin\alpha=R\cos\frac{\theta}{2}$ where $b<R$. Then, the differential cross-section is $\frac{d\sigma}{d\Omega}=\frac{b}{\sin\theta}\frac{db}{d\theta}=\frac{R^2\cos(\theta/2)\sin(\theta/2)}{2\sin\theta}=\frac{R^2}{4}$. It is independent of $\theta$ and $E$. The total cross-section is  $\sigma_T=\int\frac{d\sigma}{d\Omega}d\Omega=\pi R^2$, which is the geometrical cross-sectional area the incoming beam sees.
\end{eg}
\subsubsection{Quantum Scattering}
The time-independent Schr\"{o}dinger's equation gives $(-\frac{\hbar^2}{2m}(\nabla^2+k^2)+V(\mathbf{r}))\psi(\mathbf{r})=0$. A spherical wave of the form $e^{ikr}$ is not a solution, but $e^{ikr}/r$ is a solution except at $r=0$. We look for solutions of asymptotic form ($r\rightarrow\infty$)
\begin{equation}
\psi(r)=e^{ikz}+f(\theta,\phi)\frac{e^{ikr}}{r}\label{asymptotic}
\end{equation}
where $f(\theta,\phi)$ is the scattering amplitude. This is a superposition of an incident wave $e^{ikz}$ and scattered wave $f(\theta,\phi)\frac{e^{ikr}}{r}$. For central potentials ($V=V(r)$), we have $f=f(\theta)$, which follows from azimuthal symmetry about the scattering direction. 
\begin{prop}
The differential cross-section and total cross-section are respectively
\begin{equation}
    \frac{d\sigma}{d\Omega}=|f(\theta)|^2,\quad\sigma_T=\int|f(\theta)|^2d\Omega\label{crossection}
\end{equation}
\end{prop}
\begin{proof}
The probability current is $J=-\frac{i\hbar}{m}(\psi^*\boldsymbol{\nabla}\psi-\psi\boldsymbol{\nabla}\psi^*)$ and is conserved, i.e. satisfy the continuity equation $\boldsymbol{\nabla}\cdot\mathbf{J}=0$.
$$\psi_{\text{incident}}=e^{ikz}\implies\mathbf{J_{\text{incident}}}=\frac{\hbar k}{m}\mathbf{\hat{z}},\quad\psi_{\text{scattered}}=f(\theta)\frac{e^{ikr}}{r}\implies\mathbf{J_{\text{scattered}}}=\frac{\hbar k}{m}\frac{|f(\theta)|^2}{r^2}\mathbf{\hat{r}}+O(r^{-3})$$
The flux of outgoing particles through a cross-sectional area $dA=r^2d\Omega$, subtended by an infinitesimal solid angle $d\Omega$, then $\mathbf{J_{scattered}}\cdot\mathbf{\hat{r}}dA=\frac{\hbar k}{m}|f(\theta)|^2d\Omega$. Using the definition of differential cross-section (ratio of scattered flux to incident flux), $\frac{d\sigma}{d\Omega}=\frac{\hbar k|f|^2/m}{\hbar k/m}=|f(\theta)|^2$. The total cross-section is $\sigma_T=\int|f(\theta)|^2d\Omega$.
\end{proof}
\begin{remarks}
The total cross-section represents the area that removes from the incoming beam a flux of particles, equal to the flux of scattered particles.
\end{remarks}
\subsubsection{Partial wave expansion}
Still considering central potential (invariant under rotations, can thus label wavefunctions via the angular momentum number $\ell$), we will now calculate $f_k(\theta)$ in terms of quantities $\delta_\ell$ called phase shifts. Physically, the scattering takes place on each of the `partial waves' that, by superposition, represent the full wavefunction. Each partial wave is a wave of different (quantized) angular momentum $\ell$ and has an incoming and an outgoing component. 
\begin{defi}[Partial wave expansion]
For $\psi(r,\theta,\phi)$ independent of $\phi$, we can write it as an expansion of partial waves:
\begin{equation}
\psi(r,\theta)=\sum_{\ell=0}R_\ell(r)P_\ell(\cos\theta)\label{partialwaves}
\end{equation}
where the Legendre polynomials are eigenstates of the angular momentum operator, i.e. $\hat{L}^2P_\ell(\cos\theta)=\hbar^2\ell(\ell+1)P_\ell(\cos\theta)$ and satisfy the Legendre equation.
\end{defi}
\begin{prop}
The full partial wave expansion of the sum of incoming and outgoing waves
\begin{equation}
\psi(r)=\sum_{\ell=0}^\infty\frac{2\ell+1}{2ik}\bigg[(-1)^{\ell+1}\frac{e^{-ikr}}{r}+(1+2if_\ell)\frac{e^{ikr}}{r}\bigg]P_\ell(\cos\theta)\label{fullexpansion}
\end{equation}
\end{prop}
\begin{proof}
We expand asymptotic form of $\psi$, where we assumed the potential dies off quickly as $r\rightarrow\infty$. 
$$\psi_{\text{incident}}(\rho,\theta)=e^{i\rho\cos\theta}=\sum_\ell(2\ell+1)u_\ell(\rho)P_\ell(\cos\theta),\quad\rho = kr,~u_\ell(\rho)=\rho R_\ell(\rho)$$
We use the orthogonality relation for the Legendre polynomials:
$$\int_{-1}^1P_\ell(\omega)P_{\ell'}(\omega)d\omega=\frac{2}{2\ell+1}\delta_{\ell,\ell'}\implies u_\ell(\rho)=\frac{1}{2}\int_{-1}^1e^{i\rho\omega}P_\ell(\omega)d\omega=\frac{1}{2i\rho}[e^{i\rho}-(-1)^\ell e^{-i\rho}]+O(\rho^{-2})$$
where we did integration by parts twice and use $P_\ell(1)=1$ and $P_\ell(-1)=(-1)^\ell$. We expand the asymptotic incident plane waves in terms of the spherical waves
\begin{equation}
\psi_{\text{incident}}=\sum_{l=0}^\infty\frac{2\ell+1}{2ik}\bigg[\frac{e^{ikr}}{r}-(-1)^\ell\frac{e^{-ikr}}{r}\bigg]P_\ell(\cos\theta)\label{incident}
\end{equation}
But since the scattered wave has the form 
\begin{equation}
\psi_{\text{scattered}}=f(\theta)\frac{e^{ikr}}{r},\quad f(\theta)=\sum_{\ell=0}^\infty\frac{2l+1}{k}f_\ell P_\ell(\cos\theta)\label{scattered}
\end{equation}
where we can similarly expand the scattering amplitude. Result follows by adding.
\end{proof}
\begin{defi}[S-matrix for rotationally invariant potential]
We define the S-matrix elements to be $S_\ell:=1+2if_\ell$ with $\ell=0,1,2...$. S-matrix will be diagonal in the angular momentum basis for rotationally invariant potentials.
\end{defi}
\begin{thm}[Optical Theorem]
\begin{equation}
    \sigma_T=\frac{4\pi}{k}\text{Im}[f(0)]\label{optical}
\end{equation}
\end{thm}
\begin{proof}
The unitarity of the S-matrix means we can write $S_\ell$ as a pure phase, i.e. $S_\ell=e^{2i\delta_\ell}$, then $f_\ell=\frac{1}{2i}(e^{2i\delta_\ell}-1)=e^{i\delta_\ell}\sin\delta_\ell$, where the phase shifts are only defined modulo $\pi$, i.e. $\delta_\ell\sim\delta_\ell+n\pi$ for $n\in\mathbb{Z}$. The phase shift and scattering amplitude are related by
$$f(\theta)=\frac{1}{2ik}\sum_{\ell=0}^\infty (2\ell+1)(e^{2i\delta_\ell}-1)P_\ell(\cos\theta)=\frac{1}{k}\sum_{\ell=0}^\infty(2\ell+1)e^{i\delta_\ell}\sin\delta_\ell P_\ell(\cos\theta)\implies \text{Im}[f(0)]=\frac{1}{k}\sum_\ell(2\ell+1)\sin^2\delta_\ell$$
where $P(1)=1$. On the other hand, the total cross-section $\sigma_T=\int\frac{d\sigma}{d\Omega}d\Omega=2\pi\int\frac{d\sigma}{d\Omega}~d\cos\theta$ is
$$\frac{d\sigma}{d\Omega}=|f(\theta)|^2=\frac{1}{k^2}\sum_{\ell,\ell'}(2\ell+1)(2\ell'+1)f_\ell f_{\ell'}^*P_\ell(\cos\theta)P_{\ell'}(\cos\theta)\implies\sigma_T=\frac{4\pi}{k^2}\sum_\ell(2\ell+1)|f_\ell|^2=\frac{4\pi}{k^2}\sum_\ell(2\ell+1)\sin^2\delta_\ell$$
where $\int_{-1}^1P_\ell(\cos\theta)P_{\ell'}(\cos\theta)d\cos\theta=\frac{2}{2\ell+1}\delta_{\ell,\ell'}$.
\end{proof}
\begin{remarks}
Since $0\leq\sin^2\delta_\ell\leq1$, the contribution $\sigma_\ell$ that each partial wave is limited by $\sigma_\ell\leq\frac{4\pi}{k^2}(2\ell+1)$, i.e. unitarity bound; and is saturated for $\delta_\ell=(n+0.5)\pi$ for $n\in\mathbb{Z}$, i.e. resonance scattering.
\end{remarks}
Consider the hard sphere potential in Example 3.6. For $r>a$, the solution to the free Schr\"{o}dinger's equation is written in terms of partial waves, i.e. $\psi(r,\theta)=\sum_{\ell=0}R_\ell(r)P_\ell(\cos\theta)$, where the radial wavefunction satisfying
$$\bigg(\frac{d^2}{dr^2}-\frac{\ell(\ell+1)}{r^2}+k^2\bigg)(rR_\ell(r))=0\implies\bigg(\frac{d^2}{d\rho^2}-\frac{\ell(\ell+1)}{\rho^2}+1\bigg)(\rho R_\ell(\rho))=0$$
where $\rho=kr$. Then the solutions are $R_\ell(\rho)=j_\ell(\rho)=(-\rho)^\ell(\rho^{-1}\partial_\rho)^\ell\rho^{-1}\sin\rho$ and $n_\ell(\rho)=-(-\rho)^\ell(\rho^{-1}\partial_\rho)^\ell\rho^{-1}\cos\rho$, where $j_\ell$ and $n_\ell$ are called spherical Bessel functions.
\begin{lemma}\leavevmode
\begin{enumerate}
    \item When $\ell=0$, the solutions are $j_0=\frac{\sin\rho}{\rho}$ and $n_0=-\frac{\cos\rho}{\rho}$.
    \item Asymptotic behaviour as $\rho\rightarrow\infty$,
    $$j_\ell(\rho)\rightarrow\frac{\sin(\rho-0.5\ell\pi)}{\rho},\quad n_\ell(\rho)\rightarrow-\frac{\cos(\rho-0.5\ell\pi)}{\rho}$$
    \item Asymptotic behaviour as $\rho\rightarrow0$,
    $$j_\ell(\rho)\rightarrow\frac{\rho^\ell}{(2\ell+1)!!},\quad n_\ell(\rho)\rightarrow-(2\ell-1)!!\rho^{-(\ell+1)}$$
    where $(2\ell+1)!!=1\times 3\times 5\times...\times(2\ell+1)$.
\end{enumerate}
\end{lemma}
\begin{proof}\leavevmode
\begin{enumerate}
    \item $j_0(\rho)=(-\rho)^0(\rho^{-1}\partial_\rho)^0\rho^{-1}\sin\rho=\sin\rho/\rho$ and $n_0(\rho)=-(-\rho)^0(\rho^{-1}\partial_\rho)^0\rho^{-1}\cos\rho$.
    \item As $\rho\rightarrow\infty$,
    $$\frac{1}{\rho}\frac{d}{d\rho}\bigg(\frac{\sin\rho}{\rho}\bigg)\approx\frac{\cos\rho}{\rho^2}\implies\bigg(\frac{1}{\rho}\frac{d}{d\rho}\bigg)^\ell\frac{\sin\rho}{\rho}=\frac{\sin(\rho+\ell\pi/2)}{\rho^{\ell+1}}$$
    where we differentiated $\ell$ times and $-\sin\rho/\rho^3\rightarrow 0$. Hence, $j_\ell(\rho)=\sin(\rho-0.5\ell\pi)/\rho$. Similarly, $\frac{d}{d\rho}\cos\rho=\cos(\rho+0.5\pi)$. It follows that $n_\ell(\rho)\rightarrow-(-\rho)^\ell\rho^{-\ell-1}\cos(\rho+0.5\ell\pi)=-\rho^{-1}\cos(\rho+\ell\pi/2-\ell\pi)$.
    \item Substitute $u=\rho^2\implies\rho^{-1}\partial_\rho=2\partial_u$, then
    \begin{align}
        j_\ell(u)&=(-1)^\ell u^{\ell/2}2^\ell\bigg(\frac{d}{du}\bigg)^\ell\frac{\sin\sqrt{u}}{\sqrt{u}}=(-2)^\ell u^{\ell/2}\bigg(\frac{d}{du}\bigg)^\ell\frac{1}{\sqrt{u}}\sum_j\frac{(\sqrt{u})^{2j+1}(-1)^j}{(2j+1)!}\nonumber\\&=(-2)^\ell u^{\ell/2}\sum_{j=\ell}^\infty\frac{-j!(-1)^j}{(j-\ell)!(2j+1)!}u^{|j-\ell|}\implies j_\ell(\rho)=(-2)^\ell\rho^\ell\sum_{j=0}^\infty\frac{-(j+\ell)!(-1)^{j+\ell}}{j!(2\ell+2j+1)!}\rho^{2j}\nonumber
    \end{align}
    where $(\partial_u)^\ell (\sqrt{u})^{2j+1-1}=u^{|j-\ell|}$ since $(\partial_x)^\ell x^n=0$ if $\ell>n$. For small arguments $\rho\rightarrow 0$, take the $j=0$ term:  $j_\ell(\rho)\approx(-2)^\ell\rho^\ell(-1)\frac{\ell!}{(2\ell+1)!}(-1)^\ell=\frac{\rho^\ell}{(2\rho+1)!!}$, where $\frac{(2\ell+1)!}{2^\ell\ell!}=\frac{(2\ell+1)(2\ell)(2\ell-1)(2\ell-2)\dots(\ell+1)}{(2\ell)(2\ell-2)\dots1}=(2\ell+1)!!$. Similarly,
        \begin{align}
        n_\ell(u)&=-(-1)^\ell u^{\ell/2}2^\ell\bigg(\frac{d}{du}\bigg)^\ell\frac{\cos\sqrt{u}}{\sqrt{u}}=-(-2)^\ell u^{\ell/2}\bigg(\frac{d}{du}\bigg)^\ell\frac{1}{\sqrt{u}}\sum_j\frac{(\sqrt{u})^{2j}(-1)^j}{(2j)!}\nonumber\\&=-(-2)^\ell u^{\ell/2}\sum_{j=0}^\infty\frac{(j-0.5)!(-1)^j}{(j-0.5-\ell)!(2j)!}u^{|j-0.5-\ell|}\implies n_\ell(\rho)=-(-2)^\ell\rho^\ell\sum_{j=0}^\infty\frac{(-1)^j(j-0.5)!}{(j-0.5-\ell)!(2j)!}\rho^{|2j-2\ell-1|}\nonumber
    \end{align}
    For small arguments $\rho\rightarrow 0$, again take the $j=0$ term: $n_\ell(\rho)\approx(-1)(-2)^\ell\rho^\ell\frac{(-0.5)!}{(-0.5-\ell)!}\rho^{-2\ell-1}=-\frac{(2\ell-1)!!}{\rho^{\ell+1}}$, where $\frac{(-0.5)!}{(-0.5-\ell)!}=(-0.5)(-1.5)\dots(-0.5-\ell+1)=(-2)^{-\ell}(2\ell-1)!!$
\end{enumerate}
\end{proof}
\begin{eg}
Back to the hard sphere problem, the general solution for $r\geq a$ is 
$$R_\ell(\rho)=A_\ell[\cos(\alpha_\ell)j_\ell(\rho)-\sin(\alpha_\ell)n_\ell(\rho)]$$
where $A_\ell$ and $\alpha_\ell$ are fixed by the boundary conditions $r=a$. By Lemma 3.1.2, $$R_\ell(\rho\rightarrow\infty)\rightarrow\frac{1}{\rho}[\cos(\alpha_\ell)\sin(\rho-0.5\ell\pi)+\sin(\alpha_\ell)\cos(\rho-0.5\ell\pi)]=\frac{1}{\rho}\sin(\rho-0.5\pi \ell+\alpha_\ell)$$
Let us compare this to the expected asymptotic form (Eqn.~\ref{fullexpansion} and $1+2if_\ell=e^{2i\delta_\ell}$):
$$R_\ell(\rho)\sim\bigg[(-1)^{\ell+1}\frac{e^{-i\rho}}{\rho}+e^{2i\delta_\ell}\frac{e^{i\rho}}{\rho}\bigg]=\frac{e^{i\delta_l}e^{i\pi \ell/2}}{\rho}[-e^{-i(\rho+\delta_\ell-0.5\pi \ell)}+e^{i(\rho+\delta_\ell-0.5\pi \ell)}]\implies\alpha_\ell=\delta_\ell$$
The integration constant is our phase shift. At $r=a$ (hard wall condition), we must have 
$$R_\ell(a)=0\implies\cos\delta_\ell~ j_\ell(ka)=\sin\delta_\ell ~n_\ell(ka)\implies\tan\delta_\ell=\frac{j_\ell(ka)}{n_\ell(ka)}$$
At low momentum, $\rho=kr\rightarrow0$, the s-wave ($\ell=0$) will have phase shift $\tan\delta_0\rightarrow\tan(-ka)\implies\delta_0\rightarrow -ka$ (Lemma 3.1.1). The higher $\ell$, the phase shift is smaller, i.e. $\delta_\ell\approx-\frac{(ka)^{2\ell+1}}{(2\ell+1)!!(2\ell-1)!!}$ (Lemma 3.1.3). The total cross-section is dominated by s-wave.
$$kr\rightarrow 0:\quad\sigma_T=\frac{4\pi}{k^2}\sum_\ell(2\ell+1)\sin^2\delta_\ell= 4\pi a^2(1+O((ka)^4))$$
The quantum total cross-section $\sigma_T$ is the total surface area of the sphere, rather than the transverse area the sphere presents to the beam. For general potentials, the low momentum scattering is dominated by s-wave, with $\delta_0=-ka_s+O(k^3)$ where $a_s$ is the scattering length. The total cross-section is simply $\sigma_T\rightarrow 4\pi a_s^2$ on $k\rightarrow 0$.\\[5pt]
At high momentum, $\rho=kr\rightarrow\infty$, $\tan\delta_\ell=\frac{j_\ell(ka)}{n_\ell(ka)}\rightarrow-\tan(ka-0.5\ell\pi)\implies\delta_\ell\approx -ka+0.5\ell\pi$ (Lemma 3.1.2), and only holds for $\ell<ka$ since Lemma 3.1.2 is only applicable for $\ell>>\rho$. It follows that $\sigma_T\sim\frac{4\pi}{k^2}\frac{1}{2}k^2a^2=2\pi a^2$. It is larger than the classical cross-section due to diffraction effects. 
\end{eg}
\begin{remarks}\leavevmode
\begin{enumerate}
\item We can also define the hard sphere phase shifts to be $\delta_{\ell,\text{hard}}$:
$$e^{i2\delta_{\ell,\text{hard}}}=\frac{1+i\tan\delta_{\ell,\text{hard}}}{1-i\tan\delta_{\ell,\text{hard}}}=\frac{i(-in_\ell(ka)+j_\ell(ka))}{-i(in_\ell(ka)+j_\ell(ka))}:=-\frac{h_\ell^{(2)}(ka)}{h_\ell^{(1)}(ka)}$$
where $\tan\delta_\ell=\frac{j_\ell(ka)}{n_\ell(ka)}$ for the hard sphere. The numerator and denominator are the spherical Hankel functions. By Lemma 3.1, $h_0^{(1)}(\rho)=\frac{e^{i\rho}}{i\rho}$ and $h_\ell^{(1)}(\rho\rightarrow\infty)\rightarrow-\frac{i}{\rho}e^{i(\rho-\ell0.5\pi)}$.
\item Suppose $ka\sim\ell_0$ for some $\ell_0\in\mathbb{Z}^+$, then the partial waves with $\ell\leq\ell_0$ give the largest contribution to scattering. For a potential of finite range $a$, we will have no classical scattering when the impact parameter $b>a$. Quantum mechanically. We expect $L\approx\hbar\ell\approx b\hbar k\implies b\approx\ell/k>a$, hence little scattering for $\ell>ka$.
\item The free $\ell$-th partial wave is negligible for impact parameter $b=\ell/k>r$. Although $V=0$ when free, the centrifugal barrier prevents the wave from reaching the origin. If $r_*$ is the position of the turning point, the wave should be small in the classically forbidden region $r<r_*$. The turning point $r=r_*$ is determined by $\frac{\hbar^2\ell(\ell+1)}{2mr_*^2}=\frac{\hbar^2k^2}{2m}\implies kr_*\approx\ell$.
\item Semiclassically, the partial cross-section $\sigma_\ell$ can be thought of as the area in the annulus with inner radius impact parameter $b(\ell)$ for $\ell$ and outer radius impact parameter $b(\ell+1)$ for $\ell+1$. Hence, $\sigma_\ell\approx 2\pi b(\ell)\Delta b=2\pi b(\ell)(b(\ell+1)-b(\ell))=2\pi\frac{\ell}{k}\frac{1}{k}$.
\end{enumerate}
\end{remarks}
\begin{eg}[General computation of phase shift]
For an arbitrary potential with range $a$, we need to find a radial solution $R_\ell(r)$ that applies for $kr\leq a$. Continuity at $r=a$:
$$R_\ell(a)=A_\ell j_\ell(ka)+B_\ell n_\ell(ka),\quad aR_\ell'(a)=ka(A_\ell j_\ell'(ka)+B_\ell n'_\ell(ka))$$
Define the logarithmic derivative of the radial solution to be
$$\beta_\ell=\frac{aR_\ell'(a)}{R_\ell(a)}=ka\frac{j_\ell'(ka)+(B_\ell/A_\ell)n_\ell'(ka)}{j_\ell(ka)+(B_\ell/A_\ell)n_\ell(ka)}\implies\tan\delta_\ell=\frac{B_\ell}{A_\ell}=\frac{j_\ell(ka)-(ka/\beta_\ell)j_\ell'(ka)}{n_\ell(ka)-(ka/\beta_\ell)n_\ell'(ka)}$$
If $\beta_\ell\rightarrow\infty$, we recover the phase shifts of a hard sphere. One can show that
$$e^{2i\delta_\ell}=e^{2i\delta_{\ell,\text{hard}}}\frac{\beta_\ell-ka\frac{j_\ell'-in_\ell'}{j_\ell-in_\ell}}{\beta_\ell-ka\frac{j_\ell'+in_\ell'}{j_\ell+in_\ell}}$$
\end{eg}
Since both the Cartesian plane wave solutions and the spherical wave solutions are complete, it should be possible to write any plane wave solution as a superposition of spherical wave solutions:
\begin{cor}[Rayleigh formula]
\begin{equation}
    e^{ikz}=\sqrt{4\pi}\sum_{\ell=0}^\infty\sqrt{2\ell+1}i^\ell Y_{\ell,0}(\theta)j_\ell(kr)\label{Rayleigh}
\end{equation}
\begin{equation}
    f_k(\theta)=\frac{\sqrt{4\pi}}{k}\sum_{\ell=0}^\infty\sqrt{2\ell+1}Y_{\ell,0}(\theta)e^{i\delta_\ell}\sin\delta_\ell\label{Rayleigh2}
\end{equation}
\end{cor}
\begin{proof}
In general, $e^{i\mathbf{k}\cdot\mathbf{r}}=\sum_{\ell=0}^\infty\sum_{m=-\ell}^\ell A_{\ell,m}(\mathbf{\hat{k}})j_\ell(kr)Y_{\ell,m}(\Omega)$. These coefficients can depend on the direction $\mathbf{\hat{k}}$ of $\mathbf{k}$. Need one particular case $\mathbf{k}=k\mathbf{\hat{z}}$:
$$e^{ikz}=e^{ikr\cos\theta}=\sum_{\ell=0}^\infty\sum_{m=-\ell}^\ell A_{\ell,m}j_\ell(kr)Y_{\ell,m}(\Omega)$$
Since the LHS has no $\phi$ dependence, $m=0$ is the only allowed value on the RHS. But, $Y_{\ell,0}\propto P_\ell(\cos\theta)$, we have $e^{ikr\cos\theta}=\sum_{\ell=0}^\infty a_\ell P_\ell(\cos\theta)j_\ell(kr)$ for some coefficients $a_\ell$ that must be determined. The spherical Bessel functions can be written as an integral involving Legendre polynomials:
$$j_\ell(x)=\frac{1}{2i^\ell}\int_{-1}^1e^{ixu}P_\ell(u)du$$
Together with $Y_{\ell,0}(\theta)=\sqrt{\frac{2\ell+1}{4\pi}}P_\ell(\cos\theta)$, giving us the Rayleigh formula. Similar for $f_k(\theta)$
\end{proof}
\begin{remarks}
For large $r$, $j_\ell(kr)\rightarrow\frac{\sin(kr-\ell\pi/2)}{kr}=\frac{1}{2ik}(r^{-1}e^{i(kr-0.5\ell\pi)}-r^{-1}e^{-i(kr-0.5\ell\pi)})$ (Lemma 3.1.2). The Rayleigh formula for the incident wave then gives
$$e^{ikz}\rightarrow\frac{\sqrt{4\pi}}{k}\sum_{\ell=0}^\infty\sqrt{2\ell+1}i^\ell Y_{\ell,0}(\theta)\frac{1}{2i}\bigg[\frac{e^{i(kr-\ell0.5\pi)}}{r}-\frac{e^{-i(kr-0.5\ell\pi)}}{r}\bigg]$$
where $r\rightarrow\infty$, giving a form consistent with Eqn.~\ref{incident} for the incident wave, since $i^\ell=e^{i\ell\pi/2}\implies\frac{(i)^\ell}{r}(e^{ikr}(-i)^\ell-e^{-ikr}i^\ell)=\frac{1}{r}(e^{ikr}-(-1)^\ell e^{-ikr})$. For the scattering amplitude, Eqn.~\ref{Rayleigh2} is equivalent to Eqn.~\ref{scattered} with $f_\ell=e^{i\delta_\ell}\sin\delta_\ell$ and $Y_{\ell,0}(\theta)=\sqrt{\frac{2\ell+1}{4\pi}}P_\ell(\cos\theta)$.
\end{remarks}
\subsubsection{Bound States}
We will study the effects of bound states (occur only for attractive potentials) on scattering.
\begin{eg}
Consider the attractive version of the hard sphere potential, i.e. $V(r)=-V_0$ for $r<a$. We set $U(r):=\frac{2m}{\hbar^2}V(r)$ and $\gamma^2=\frac{2mV_0}{\hbar^2}$. We focus on the s-wave ($\ell=0$). For $r>a$ and $r<a$ respectively:
$$\bigg(\frac{d^2}{dr^2}+k^2\bigg)(r\psi)=0\implies\psi(r)=\frac{A\sin(kr+\delta_0)}{r},\quad\bigg(\frac{d^2}{dr^2}+k^2+\gamma^2\bigg)(r\psi)=0\implies\psi(r)=\frac{B\sin(\sqrt{k^2+\gamma^2}r)}{r}$$
The continuity of $\frac{\psi'}{\psi}$ at $r=a$:
$$\frac{\tan(ka+\delta_0)}{ka}=\frac{\tan(\sqrt{k^2+\gamma^2}a)}{\sqrt{k^2+\gamma^2}a}$$
For very high momentum scattering, $k^2>>\gamma^2$, so $\delta_0\rightarrow 0$, i.e. the energy of the particle is so large that there is no scattering. For low energies, $k^2<<\gamma^2$ and $ka<<1$, we have
$$\frac{\tan(ka+\delta_0)}{ka}\approx\frac{\tan(\gamma a)}{\gamma a}\implies\frac{\tan(ka)+\tan\delta_0}{1-\tan(ka)\tan\delta_0}\approx\frac{k}{\gamma}\tan(ka)\implies\tan\delta_0=ka\bigg(\frac{\tan\gamma a}{\gamma a}-1\bigg)+O(k^3)$$
If $\delta_0$ is small and so $$\tan\delta_0\approx\delta_0=-ka_s,\quad a_s=a-\frac{\tan\gamma a}{\gamma}$$
Surprisingly, for small $\gamma$, $a_s<0$ due to the attractive nature of the potential. A more surprising behaviour is $a_s$ diverges at $\gamma=\frac{\pi}{a}(m+0.5)$, accompanied by the formation of an additional bound state. $a_s\rightarrow-\infty$, followed by a return to positive values. 
\end{eg}
\begin{remarks}
As a side point, for $V_0<0$ (repulsive potentials), $a_s=a-\frac{\tanh(|\gamma|a)}{|\gamma|}>0$, which increases monotonically from $a_s=0$ when $\gamma=0$ to $a_s=a$ when $|\gamma|\rightarrow\infty$.\\[5pt]
So why the divergent behaviour? The key lies in the bound states. It is simple to solve the $\ell=0$. We use the ansatz
$$r\psi(r)\sim
\left\{
        \begin{array}{ll}
      A\sin(\sqrt{\gamma^2-\lambda^2}r) & r<a \\
      Be^{-\lambda r} & r>a
        \end{array}
    \right.$$
with $E=-\frac{\hbar^2\lambda^2}{2m}$. Then, $\frac{\psi}{\psi'}$ is continuous at $r=a$:
\begin{equation}
    \tan(\sqrt{\gamma^2-\lambda^2}a)=-\frac{\sqrt{\gamma^2-\lambda^2}}{\lambda}\tag{\dag}
\end{equation}
This is the equation we found for odd parity bound state in 1D. Recall if the potential is too shallow, then no bound state will exist. A new bound state with $E=0$ appears at $\gamma=\gamma_*=(m+0.5)\frac{\pi}{a}$, where $m=0,1,2...$ This is called a bound state at threshold. As in 1D, this is related to a pole in the S-matrix. From Example 3.9,
$$\tan(ka+\delta_0)=\frac{k}{\sqrt{k^2+\gamma^2}}\tan(\sqrt{k^2+\gamma^2}a)$$
Let $g(k):=\frac{k}{\sqrt{k^2+\gamma^2}}\tan(\sqrt{k^2+\gamma^2}a)$ and the S-wave component of the S-matrix to be
$$S_0(k):=e^{2i\delta_0}=e^{2i(\delta_0+ka)}e^{-2ika}=e^{-2ika}\frac{1+i\tan(\delta_0+ka)}{1-i\tan(\delta_0+ka)}=e^{-2ika}\frac{1+ig(k)}{1-ig(k)}$$
The pole thus occur at $g(k)=-i$ (recall condition for bound state) and hence
$$\tan(\sqrt{k^2+\gamma^2}a)=\frac{\sqrt{k^2+\gamma^2}}{ik}$$
This has no solution for $k\in\mathbb{R}$. But for $k=i\lambda$, it coincides with (\dag). More generally, close to a pole the S-matrix takes the form
$$S_0=e^{2i\delta_0}=\frac{i\lambda+k}{i\lambda-k}$$
For the bound states at threshold, $\lambda$ is small and this form is valid around $k=0$. So let's expand in $k/\lambda$, $\delta_0\approx-k/\lambda\implies a_s=1/\lambda$, i.e. divergent scattering length.
\end{remarks}
In summary, bound states occur at poles in the S-matrix. The bound states at threshold give rise to divergent scattering lengths and divergent total cross-sections.
\subsubsection{Resonances}
Recall as in 1D, resonances appear as poles in the S-matrix in the lower-half complex $k$-plane. $E=\frac{\hbar^2k^2}{2m}\implies E=E_0-\frac{i\Gamma}{2}$. Then, close to a pole, 
$$S(E)=e^{2i\delta(E)}=\frac{E-(E_0+i\Gamma/2)}{E-(E_0-i\Gamma/2)}$$
where we neglect some overall phase at the end. Then we have
$$\cos2\delta=\frac{(E-E_0)^2-(\Gamma^2/4)}{(E-E_0)^2+(\Gamma^2/4)}\implies\sin^2\delta=\frac{\Gamma^2}{4(E-E_0)^2+\Gamma^2/4}\implies\sigma_T\sim\frac{4\pi}{k^2}(2l+1)\frac{\Gamma^2}{4(E-E_0)^2+\Gamma^2}$$
This is the Breit-Wigner distribution. This shows up as a bump at $E_0$ and width $\frac{\Gamma}{2}$ in the $\sigma_T$ against $E$ plot. This is a distinctive sign of a resonance. 
\begin{center}
\begin{tikzpicture}
      \draw[->] (0,0) -- (3,0) node[right] {$E$};
      \draw[->] (0,0) -- (0,3) node [left]{$\sigma_T$};
      \draw[domain=0:2.5,smooth,variable=\x,blue] plot ({\x},{3/(1+4*(\x-1)^2))});
      \draw (1,0) node[below]{$E_0$};
      \draw (0,0) node[below]{0};
\end{tikzpicture}
\end{center}
\begin{remarks}
In particle physics, it is how we discover new particles with $mc^2=E_0$ and half-life $\tau=\hbar/\Gamma$.
\end{remarks}
\newpage
\subsection{Lippmann-Schwinger Equation}
\begin{prop}
One solution to the integral equation $(\nabla^2+k^2)\psi(\mathbf{r})=U(\mathbf{r})\psi(\mathbf{r})$ (which represents the time-independent Schr\"{o}dinger's equation), consistent with the expected asymptotic behaviour, is
\begin{equation}
    \psi(\mathbf{r})=e^{i\mathbf{k_i}\cdot\mathbf{r}}-\frac{1}{4\pi}\frac{e^{ikr}}{r}\int e^{-i\mathbf{k_s}\cdot\mathbf{r'}}U(\mathbf{r'})\psi(\mathbf{r'})d^3\mathbf{r'}\label{scattering}
\end{equation}
\end{prop}
\begin{proof}
With $E=\frac{\hbar^2k^2}{2m}$ and $V(\mathbf{r})=\frac{\hbar^2}{2m}U(\mathbf{r})$, the Schr\"{o}dinger's equation gives $(\nabla^2+k^2)\psi(\mathbf{r})=U(\mathbf{r})\psi(\mathbf{r})$. The corresponding Green's function satisfies the equation
$$(\nabla^2+k^2)G(\mathbf{r}-\mathbf{r'})=\delta^{(3)}(\mathbf{r}-\mathbf{r'})$$
where the Laplacian $\nabla^2$ acts on $\mathbf{r}$ and not $\mathbf{r'}$. Let $\psi_0(\mathbf{r})$ be the solution to the homogeneous equation, i.e. $(\nabla^2+k^2)\psi_0(\mathbf{r})=0$. The solution of the integral equation is
$$\psi(\mathbf{r})=\psi_0(\mathbf{r})+\int G(\mathbf{r}-\mathbf{r'})U(\mathbf{r'})\psi(\mathbf{r'})d^3\mathbf{r'}$$
By translational invariance, set $\mathbf{r'}=\boldsymbol{0}$, then $G(\mathbf{r}-\mathbf{r'})\rightarrow G(\mathbf{r})$, giving $(\nabla^2+k^2)G(\mathbf{r})=\delta^{(3)}(\mathbf{r})$. Away from the origin, the operator $\nabla^2+k^2$ must kill $G$. As $r\rightarrow 0$, $e^{\pm ikr}/r\rightarrow 1/r\implies\nabla^2(1/r)=-4\pi\delta(\mathbf{r})$. This gives $G_\pm(\mathbf{r})=-\frac{1}{4\pi}\frac{e^{\pm ikr}}{r}$ which solves the desired equation $(\nabla^2+k^2)G_\pm(\mathbf{r})=\delta^{(3)}(\mathbf{r})$. In our scattering problem, we have outgoing waves emerging from the region where the potential is non-zero. We also have an incoming plane wave $\psi_0(\mathbf{r})=e^{ikz}$ that is a solution of the homogeneous equation $(\nabla^2+k^2)e^{ikz}=0$. By causality, we choose $G_+$. Hence,
\begin{equation}
\psi(\mathbf{r})=e^{ikz}+\int G_+(\mathbf{r}-\mathbf{r'})U(\mathbf{r'})\psi(\mathbf{r'})d^3\mathbf{r'},\quad G_+(\mathbf{r}-\mathbf{r'})=-\frac{1}{4\pi}\frac{e^{ik|\mathbf{r}-\mathbf{r'}|}}{|\mathbf{r}-\mathbf{r'}|}\label{integralsol}
\end{equation}
Finally, we want to prove that this choice implements our asymptotic expansion for the energy eigenstates. For that we need to make some approximations in the Green's function that are valid when the observation point $\mathbf{r}$ is far away from the region where the potential is non-zero. This is the region of integration for $\mathbf{r'}$ in our integral equation. Being far away means $r>>r'\implies|\mathbf{r}-\mathbf{r'}|\approx r-\mathbf{n}\cdot\mathbf{r'}$. We can further approximate it as $r$ in the denominator of the Green's function, but for the more sensitive phase in the exponent we need the first correction above, i.e. $G_+(\mathbf{r}-\mathbf{r'})\approx-\frac{e^{ikr}e^{-ik\mathbf{n}\cdot\mathbf{r'}}}{4\pi r}$. By generalizing $\psi(\mathbf{r})=e^{ikz}-\frac{1}{4\pi}\int e^{-ik\mathbf{n}\cdot\mathbf{r'}}U(\mathbf{r'})\psi(\mathbf{r'})d^3\mathbf{r'}\frac{e^{ikr}}{r}$, this gives our desired result. 
\end{proof}
\begin{remarks}\leavevmode
\begin{enumerate}
\item The Green's function $G_+$ represents outgoing waves and it is called a retarded Green's function. Although there is no time dependence in the above analysis, if we appended the usual $e^{-iEt/\hbar}$ it would represent outgoing waves. The Green's function $G_-$ is called an advanced Green's function and represents incoming waves. 
\item In Eqn.~\ref{scattering}, the factor $e^{ikr}/r$ to the right of the integral represents the outgoing waves of our asymptotic ansatz. Since $\mathbf{r'}$ is integrated over, the expression in the brackets is a function of the unit vector $\mathbf{n}$ in the direction of the observation point $\mathbf{r}$. This can be identified with the scattering amplitude $f_k(\theta,\phi)$ in the asymptotic ansatz:
\begin{equation}
    f_k(\theta,\phi)=-\frac{1}{4\pi}\int e^{-ik\mathbf{n}\cdot\mathbf{r'}}U(\mathbf{r'})\psi(\mathbf{r'})d^3\mathbf{r'}\label{scatteringamp}
\end{equation}
\item To properly solve for $G_+$, we can solve it by Fourier transform. Let $\mathbf{q}=\mathbf{r}-\mathbf{r'}$.
$$\tilde{G}_0(k;\mathbf{q})=\int e^{-i\mathbf{q}\cdot\mathbf{x}}G_0(k;\mathbf{x})d^3x,\quad G_0(k,\mathbf{x})=\int e^{+i\mathbf{k}\cdot\mathbf{q}}\tilde{G}_0(k,\mathbf{q})\frac{d^3q}{(2\pi)^3}$$
Then, plug in the Green's function
$$(-q^2+k^2)\tilde{G}_0(k;\mathbf{q})=\frac{2m}{\hbar^2}\implies \tilde{G}_0(k;\mathbf{q})=-\frac{2m}{\hbar^2}\frac{1}{q^2-k^2}$$
Convert into real space
$$G_0(k;\mathbf{x})=-\frac{2m}{\hbar^2}\int\frac{e^{i\mathbf{q}\cdot\mathbf{x}}}{q^2-k^2}\frac{d^3q}{(2\pi)^3}$$
This definition of our Green's function is ambiguous due to the singularity at $q^2=k^2$. To resolve, we define
\begin{eqnarray}
G_0^+(k;\mathbf{x})&=&-\frac{2m}{\hbar^2}\int\frac{e^{i\mathbf{q}\cdot\mathbf{x}}}{q^2-k^2-i\epsilon}\nonumber\\&=&-\frac{2m}{\hbar^2(2\pi)^3}\int_0^{2\pi}d\phi\int_{-1}^1d(\cos\theta)\int\frac{q^2e^{iqx\cos\theta}}{q^2-k^2-i\epsilon}dq\nonumber\\&=&-\frac{2m}{\hbar^2}\frac{1}{(2\pi)^2}\frac{1}{ix}\int_{-\infty}^\infty\frac{qe^{iqx}dq}{(q-k-i\epsilon)(q+k+i\epsilon)}\nonumber
\end{eqnarray}
Introducing $q+k+i\epsilon$ in the denominator shifts the pole slightly off the real $q$-axis. We complete the contour in the upper-half plane. We then pick up a residue at $k+i\epsilon$
$$G_0^+(k;\mathbf{r}-\mathbf{r'})=\frac{-2m}{\hbar^2}\frac{1}{4\pi}\frac{e^{ik|\mathbf{r}-\mathbf{r'}|}}{|\mathbf{r}-\mathbf{r'}|}$$
Set to $\mathbf{r}-\mathbf{r'}$ by translational invariance. We could also have chosen to use the less preferred Green's function $G_0^-(k;\mathbf{x})\sim\frac{e^{-ikx}}{4\pi x}$. We will justify our choice soon.
\begin{center}
    \begin{tikzpicture}
      \draw [->] (-3, 0) -- (3, 0);
      \draw [->] (0, -1) -- (0, 3);

      \draw [black, thick, ->-=0.3, ->-=0.8] (-2, 0) -- (2, 0) node [pos=0.3, below] {$\gamma_0$} arc(0:180:2) node [pos=0.3, right] {$\gamma_R$};

      \node [below] at (-2, 0) {$-R$};
      \node [below] at (2, 0) {$R$};

      \node at (1.25, 1) {$\times$};
      \node [below] at (1.25,1) {$k+i\varepsilon$};
      \node at (-1.25, -1) {$\times$};
      \node [below] at (-1.25,-1) {$-k-i\varepsilon$};
    \end{tikzpicture}
  \end{center}
\end{enumerate}
\end{remarks}
\subsection{The Born Approximation}
\begin{defi}[Born series]
To solve: 
$$\psi(\mathbf{r})=e^{i\mathbf{k_i}\cdot\mathbf{r}}+\int G_+(\mathbf{r}-\mathbf{r'})U(\mathbf{r'})\psi(\mathbf{r'})d^3\mathbf{r'}$$
We write the solution as a series expansion (called the Born series) $\psi(\mathbf{r})=\sum_{m=0}^\infty\phi_m(\mathbf{r})$ with $\phi_0(\mathbf{r})=e^{i\mathbf{k}\cdot\mathbf{r}}$ being the leading term. Plug into (*) gives
$$\phi_{m+1}(\mathbf{r})=\int G_+(\mathbf{r}-\mathbf{r'})U(\mathbf{r'})\phi_m(\mathbf{r'})d^3\mathbf{r'}$$
If $\phi_n$ obeys this recursion relation, then this series solution is indeed a solution. Roughly speaking, the series converges if $U$ is small. 
\end{defi}
\begin{defi}[Born approximation]
The Born approximation is the truncation to $\psi=\phi_0+\phi_1$ (to first order). Could replace $\psi(\mathbf{r'})$ with $e^{i\mathbf{k_i}\cdot\mathbf{r'}}$. The scattering amplitude in the Born's approximation is
\begin{equation}
f_k^B(\theta,\phi)=-\frac{1}{4\pi}\int U(\mathbf{r})e^{-i\mathbf{q}\cdot\mathbf{r}}d^3\mathbf{r},\quad\mathbf{q}=\mathbf{k_s}-\mathbf{k_i},~U=\frac{2m}{\hbar^2}V\label{Born}
\end{equation}
where $\mathbf{q}$ is the wavevector transfer - the momentum that must be added to the incident one $\mathbf{k_i}=k\mathbf{n_i}$ to get the scattered one $\mathbf{k_s}=k\mathbf{n}$. We will call $\theta$ the angle between $\mathbf{k_i}$ and $\mathbf{k_s}$ (spherical angle if $\mathbf{k_i}$ is along the positive $z$-axis).
\end{defi}
\begin{remarks}\leavevmode
\begin{enumerate}
    \item Eqn.~\ref{Born} says that the scattering amplitude is the Fourier transform of the potential $U(\mathbf{r})$ evaluated at the momentum transfer $\mathbf{q}$.
    \item By conservation of energy (elastic scattering), $|\mathbf{k_i}|=|\mathbf{k_s}|=k$, then $q=2k\sin(\theta/2)$.
    \item For a central potential $V(r)$ (spherical symmetry), the integral just depends on $|\mathbf{q}|$, hence only depending on $\theta$ and independent of $\phi$. By Born's approximation,
    \begin{align}
        f_k^B(\theta)&=-\frac{m}{2\pi\hbar^2}\int_0^\infty 2\pi r^2V(r)\int_{-1}^1e^{-iqr\cos\theta}d\cos\theta~dr\nonumber\\&=-\frac{m}{\hbar^2}\int_0^\infty r^2V(r)\frac{e^{-iqr}-e^{iqr}}{-iqr}dr\nonumber\\&=-\frac{m}{\hbar^2}\int_0^\infty r^2V(r)\frac{2\sin(qr)}{qr}dr\nonumber
    \end{align}
\end{enumerate}
\end{remarks}
\begin{eg}[Yukawa Potential]
Consider the potential $V(r)=\frac{e^{-\mu r}}{r}A$, where $\frac{1}{\mu}$ is the range of the force. By the Born's approximation (Eqn.~\ref{Born}),
$$f_k^B(\theta)=-\frac{2mA}{q\hbar^2}\int_0^\infty e^{-\mu r}\sin(qr)dr=-\frac{2mA}{\hbar^2(\mu^2+q^2)}$$
Let's write this in terms of $\theta$ and $\phi$. Then, $q^2=2k^2-2kk'=2k^2(1-\cos\theta)=4k^2\sin^2\theta/2$. The leading order contribution to the cross-section given by the Yukawa potential is
$$\frac{d\sigma}{d\Omega}=\bigg(\frac{2Am}{\hbar^2\mu^2+8mE\sin^2(\theta/2)}\bigg)^2$$
with $E=\frac{\hbar^2k^2}{2m}$. As $\mu\rightarrow 0$, we will obtain the classical result expected for Coulomb potential, where $\frac{d\sigma}{d\Omega}=\frac{A^2}{16E^2\sin^4(\theta/2)}$. Below is a plot for the cross-section $\frac{d\sigma}{d\Omega}$ for the Yukawa (blue) and the Coulomb (black) potential against scattering angle $\theta$.
\begin{center}
\begin{tikzpicture}
      \draw[->] (0,0) -- (3,0) node[right] {$\theta$};
      \draw[->] (0,0) -- (0,3) node [left]{$\frac{d\sigma}{d\Omega}$};
      \draw[domain=1.75:3,smooth,variable=\x,red] plot ({\x},{1/(sin(deg(\x)/2))^4});
      \draw[domain=0:3,smooth,variable=\x,blue] plot ({\x},{1/(1+(sin(deg(\x)/2))^2)^2});
      \draw (0,0) node[below]{0};
\end{tikzpicture}
\end{center}
\end{eg}
\begin{remarks}\leavevmode
\begin{enumerate}
    \item The Born's approximation is good if the term $\psi_1=\int G_+(\mathbf{r}-\mathbf{r'})U(\mathbf{r'})e^{i\mathbf{k_i}\cdot\mathbf{r'}}d^3\mathbf{r}$ arising from the integral is small compared to the incident wave $\psi_0=e^{i\mathbf{k_i}\cdot\mathbf{r}}$, i.e. $|\psi_1|<<|\psi_0|=1$. But, $\psi_1\sim 1/r$ for large $r$ so this inequality will hold for large $r$. Since the integral $\sim a^3$ and the Green's function $G_+\sim 1/a$, we have the weak potential condition:
    $$1>>|\psi_1|\sim a^3\frac{1}{a}|U|=a^2|U|\implies|V|<<\frac{\hbar^2}{ma^2}$$
    \item Further, for large energy $ka>>1$, the condition can be weakened. The phase in $G_+$ will oscillate very quickly, thus cutting off the integral that defines $|\psi_1|$. It can then be shown $|V|<<\frac{\hbar^2}{ma^2}ka$. For sufficiently large energy, the Born approximation will always be valid.
    \item We can expand the Born's series to the next order:
    $$\psi(\mathbf{r})=e^{i\mathbf{k_i}\cdot\mathbf{r}}+\int G_+(\mathbf{r}-\mathbf{r'})U(\mathbf{r'})e^{i\mathbf{k_i}\cdot\mathbf{r'}}d^3\mathbf{r'}+\int G_+(\mathbf{r}-\mathbf{r'})U(\mathbf{r'})\int G_+(\mathbf{r'}-\mathbf{r''})U(\mathbf{r''})e^{i\mathbf{k_i}\cdot\mathbf{r''}}d^3\mathbf{r''}d^3\mathbf{r'}+\dots$$
    In the second order Born term, the incident wave scatters at $\mathbf{r''}$ and the generated wave propagates to $\mathbf{r'}$ where it scatteres again, sending a wave all the way to $\mathbf{r}$.
\end{enumerate}
\end{remarks}
\newpage
\section{Identical Particles}
\subsection{Bosons, fermions}
\begin{defi}[Bosons, fermions]
Bosons (fermions) are quantum particles that are symmetric (anti-symmetric) under the interchange of any two particles, incurring a positive (negative) sign.
\end{defi}
\begin{remarks}
Bosons and fermions are said to be indistinguishable, since $|\Psi_{12}|^2=|\Psi_{21}|^2\implies\Psi_{1,2}=\pm\Psi_{2,1}$.
\end{remarks}
\begin{defi}[Permutation operator]
$P_{ij}$ produces the amplitude for observing the $j$th particle at $\mathbf{r_i}$ with spin $s_i$, and the $i$th particle at $\mathbf{r_j}$ with spin $s_j$:
\begin{equation}
    \langle 1,2,\dots,i,\dots,j,\dots,n|P_{ij}|\Psi\rangle=\langle1,2,\dots,j,\dots,i,\dots,n|\Psi\rangle\label{transposition}
\end{equation}
A permutation involving two particles is said to be a transposition. More generally, we have $P_{abc\dots k}$ where particle b is replaced with a, c with b, etc, a with k... The sign of a permutation $\sgn(P)=(-1)^P$ is the parity of the total number of transpositions the permutation may decompose into.
\end{defi}
\begin{remarks}
Any permutation can be written as a product of simple transpositions $P_{ij}$. In general, the product of transpositions do not commute.
\end{remarks}
\begin{eg}
$P_{12}P_{13}\psi(123)=P_{12}\psi(321)=\psi(231)=P_{123}\psi(123)\implies P_{123}=P_{12}P_{13}$.
\end{eg}
\begin{prop}
If $A(1,2,\dots,n)$ is any operator, not necessarily symmetric, then $P_{ij}A(1,2,\dots,i,\dots,j,\dots,n)=A(1,2,\dots,j,\dots,i,\dots,n)P_{ij}$, i.e. $PAP^{-1}=A$ with $P^{-1}_{ij}=P_{ji}=P_{ij}$.
\end{prop}
\begin{eg}
Consider the momentum operator, we can show $P_{12}p_1=p_2P_{12}$:
$$\langle \mathbf{r_1},s_1;\mathbf{r_2},s_2|P_{12}p_1|\Psi\rangle=\langle\mathbf{r_2},s_2;\mathbf{r_1},s_1|p_1|\Psi\rangle=\frac{\hbar}{i}\boldsymbol{\nabla}_2\langle\mathbf{r_2},s_2,\mathbf{r_1},s_1|\Psi\rangle$$
where we differentiated the argument in the first slot. But, $\frac{\hbar}{i}\boldsymbol{\nabla}_2\langle\mathbf{r_2},s_2;\mathbf{r_1},s_1|\Psi\rangle=\frac{\hbar}{i}\boldsymbol{\nabla}_2\langle\mathbf{r_1},s_1;\mathbf{r_2},s_2|P_{12}|\Psi\rangle=\langle\mathbf{r_1},s_1;\mathbf{r_2},s_2|p_2P_{12}|\Psi\rangle$.
\end{eg}
\begin{remarks}
The expectation of a symmetric operator in any state is independent of the order of the labelling of the particles in $\psi$.
\end{remarks}
\begin{prop}
Suppose $|\Psi\rangle$ is an eigenstate of a symmetric many particle Hamiltonian $H(1,2,\dots,n)$ with energy $E$. Since $H$ commutes with any $P$, then $P|\Psi\rangle$ is also an eigenstate of $H$ with the same energy $E$. If $P|\Psi\rangle$ is distinct from $|\Psi\rangle$, the Hamiltonian has exchange degeneracy. Also, the eigenstates have a definite exchange symmetry
\end{prop}
\begin{proof}
$HP|\Psi\rangle=PH|\Psi\rangle=PE|\Psi\rangle=EP|\Psi\rangle$.
\end{proof}
\begin{remarks}
A system of $n$ identical particles starting out in a completely symmetric, or anti-symmetric, state must always remain in such a state, i.e. the symmetry of the wavefunction is a constant of the motion. Any perturbation $V(1,2,\dots,n)$ that can act to change the state must be a completely symmetric function of the coordinates of the $n$ particles, and therefore commutes with any $P$, i.e. $PV|\Psi\rangle=VP|\Psi\rangle=\pm V|\Psi\rangle$.
\end{remarks}
\begin{cor}
A simultaneous eigenstates of all the $P_{ij}$ must have all eigenvalues $+1$ or all $-1$, i.e. totally symmetric or totally antisymmetric.
\end{cor}
\begin{eg}
The wavefunction of two indistinguishable particles is
$$\Psi(\mathbf{r_1},\mathbf{r_2})=\frac{1}{\sqrt{2}}(\phi_1(\mathbf{r_1})\phi_2(\mathbf{r_2})\pm\phi_2(\mathbf{r_1})\phi_1(\mathbf{r_2}))$$
where $+$ is for bosons, $-$ for fermions. When $\phi_1=\phi_2$ for fermions, we obtain Pauli exclusion principle, i.e. $\Psi=0$. The joint probability distribution is
$$P_{12}(\mathbf{r_1},\mathbf{r_2})=\frac{1}{2}(|\phi_1(\mathbf{r_1})|^2|\phi_2(\mathbf{r_2})|^2\pm2\text{Re}[\phi_1(\mathbf{r_1})\phi_1^*(\mathbf{r_2})\phi_2(\mathbf{r_2})\phi_2^*(\mathbf{r_1})]+|\phi_1(\mathbf{r_2})|^2|\phi_2(\mathbf{r_1})|^2)$$
The probability density of finding particle 1 at position $\mathbf{r_1}$ and separately for particle 2 at $\mathbf{r_2}$ are respectively:
$$P_1(\mathbf{r_1})=\int P_{12}(\mathbf{r_1},\mathbf{r_2})d^3\mathbf{r_2}=\frac{1}{2}(|\phi_1(\mathbf{r_1})|^2\pm2\text{Re}[\phi_1(\mathbf{r_1})\phi_2^*(\mathbf{r_1})\langle\phi_1|\phi_2\rangle]+|\phi_2(\mathbf{r_1})|^2)$$
$$P_2(\mathbf{r_2})=\int P_{12}(\mathbf{r_1},\mathbf{r_2})d^3\mathbf{r_1}=\frac{1}{2}(|\phi_1(\mathbf{r_1})|^2\pm2\text{Re}[\phi_1(\mathbf{r_2})\phi_2^*(\mathbf{r_1})\langle\phi_2|\phi_1\rangle]+|\phi_1(\mathbf{r_2})|^2)$$
If $\langle\phi_1|\phi_2\rangle=0$, $P_{12}\neq P_1P_2$. The term $\text{Re}[\phi_1(\mathbf{r_1})\phi_1^*(\mathbf{r_2})\phi_2(\mathbf{r_2})\phi_2^*(\mathbf{r_1})]$ contributes to the bunching of bosons and effective repulsion between fermions.
\end{eg}
\begin{prop}
The overall symmetric and anti-symmetric $N$ identical particle state is
\begin{equation}
|\Psi^S_{\alpha_1,\dots,\alpha_N}\rangle=\sqrt{\frac{1}{N!\prod_\alpha N_\alpha!}}\sum_PP\phi_{\alpha_1}(\mathbf{r_1})\phi_{\alpha_2}(\mathbf{r_2})\dots\phi_{\alpha_N}(\mathbf{r_N}),\quad \mathcal{S}=\frac{1}{N!}\sum_PP\label{symmetric}
\end{equation}
\begin{equation}
|\Psi^A_{\alpha_1,\dots,\alpha_N}\rangle=\frac{1}{\sqrt{N!}}\sum_P(-1)^PP\phi_{\alpha_1}(\mathbf{r_1})\phi_{\alpha_2}(\mathbf{r_2})\dots\phi_{\alpha_N}(\mathbf{r_N}),\quad\mathcal{A}=\frac{1}{N!}\sum_P\sgn(P)P\label{antisymmetric}
\end{equation}
\end{prop}
\begin{proof}
Take $\langle\psi_N^S|\psi_N^S\rangle$: assume w.l.o.g. that $|\alpha_i\rangle$ have been ordered in groups of identical single-particle states. The only permutations leading to non-zero contributions are the ones where the $\langle\alpha_i|$ are ordered in the same groups, and only permutations within those groups occur.
\begin{eqnarray}
&&\frac{1}{N!\prod_\alpha N_\alpha}\sum_{P_N,Q_N}\langle\alpha_{P_N(1)}|\alpha_{Q_N(1)}\rangle\dots\langle\alpha_{P_N(N)}|\alpha_{Q_N(N)}\rangle\nonumber\\&=&\frac{1}{\prod_\alpha N_\alpha}\sum_{P_N}\langle\alpha_{P_N(1)}|\alpha_1\rangle\langle\alpha_{P_N(2)}|\alpha_2\rangle\dots\langle\alpha_{P_N(N)}|\alpha_N\rangle\nonumber\\&=&\frac{1}{\prod_\alpha N_\alpha}\prod_\alpha\bigg(\sum_{P_{N_\alpha}}\langle\alpha_{P_{N_\alpha}(1)}|\alpha_1\rangle\dots\langle\alpha_{P_{N_{\alpha_1}}(N_{\alpha_1})}|\alpha_{N_{\alpha_1}}\rangle\dots\langle\alpha_{P_{N_{\alpha_M}}(N-N_{\alpha_M})}|\alpha_{N-N_{\alpha_M}}\rangle\dots\langle\alpha_{P_{N_{\alpha_1}}(N)}|\alpha_N\rangle\bigg)\nonumber\\&=&\frac{1}{\prod_\alpha N_\alpha}\prod_\alpha\sum_{P_{N_\alpha}}(1)=\frac{\prod_\alpha N_\alpha}{\prod_\alpha N_\alpha}=1\nonumber
\end{eqnarray}
Also, $\sum_P\langle\alpha_{P(1)}|\alpha_{Q(1)}\rangle\langle\alpha_{P(2)}|\alpha_{Q(2)}\rangle\dots\langle\alpha_{P(N)}|\alpha_{Q(N)}\rangle$ is independent of which permutation $Q$ we choose, since all choices of pairing are being evaluated and summed. Take $\langle\psi_N^A|\psi_N^A\rangle$:
\begin{eqnarray}
&&\frac{1}{N!}\sum_{P_N,Q_N}\sgn(P_N)\sgn(Q_N)\langle\alpha_{P_N(1)}|\alpha_{Q_N(1)}\rangle\dots\langle\alpha_{P_N(N)}|\alpha_{Q_N(N)}\rangle\nonumber\\&=&\frac{1}{N!}\sum_{P_N,Q_N}\sgn(P_N)\sgn(Q_N)\delta_{P_N(1),Q_N(1)}\dots\delta_{P_N(N),Q_N(N)}\nonumber\\&=&\frac{1}{N!}\sum_{P_N}\sgn(P_N)^2=\frac{1}{N!}\sum_{P_N}(1)=\frac{N!}{N!}=1\nonumber
\end{eqnarray}
where $P$ and $Q$ are $N$-body permutations. There are a total of $N!$ different permutation operators.
\end{proof}
\begin{eg}
For three fermions,
\begin{align}
&\frac{1}{\sqrt{3!}}\bigg[\phi_1(\mathbf{r_1})\phi_2(\mathbf{r_2})\phi_3(\mathbf{r_3})+\phi_1(\mathbf{r_2})\phi_2(\mathbf{r_3})\phi_3(\mathbf{r_1})+\phi_1(\mathbf{r_3})\phi_2(\mathbf{r_1})\phi_3(\mathbf{r_2})\bigg]\nonumber\\&-\frac{1}{\sqrt{3!}}\bigg[\phi_1(\mathbf{r_3})\phi_2(\mathbf{r_2})\phi_3(\mathbf{r_1})+\phi_1(\mathbf{r_1})\phi_2(\mathbf{r_3})\phi_3(\mathbf{r_2})+\phi_1(\mathbf{r_2})\phi_2(\mathbf{r_1})\phi_3(\mathbf{r_3})\bigg]\nonumber
\end{align}
For three bosons, suppose we have two in state $\phi_1$ and one in state $\phi_2$,
\begin{align}
&\frac{1}{\sqrt{2!3!}}\bigg[2\phi_1(\mathbf{r_1})\phi_1(\mathbf{r_2})\phi_2(\mathbf{r_3})+2\phi_1(\mathbf{r_2})\phi_1(\mathbf{r_2})\phi_2(\mathbf{r_1})+2\phi_1(\mathbf{r_3})\phi_1(\mathbf{r_1})\phi_2(\mathbf{r_2})\bigg]\nonumber\\&=\frac{1}{\sqrt{3!}}\bigg[\phi_1(\mathbf{r_1})\phi_1(\mathbf{r_2})\phi_2(\mathbf{r_3})+\phi_1(\mathbf{r_2})\phi_1(\mathbf{r_3})\phi_2(\mathbf{r_1})+\phi_1(\mathbf{r_3})\phi_1(\mathbf{r_1})\phi_2(\mathbf{r_3})\bigg]\nonumber\end{align}
\end{eg}
\begin{remarks}
The fully anti-symmetric wavefunction takes the form of a determinant (Slater determinant):
$$|\Psi^A_{\alpha_1,\alpha_2,\dots,\alpha_N}\rangle=\frac{1}{\sqrt{N!}}\begin{vmatrix}\phi_{\alpha_1}(\mathbf{r_1})&\phi_{\alpha_1}(\mathbf{r_2})&\dots &\phi_{\alpha_1}(\mathbf{r_N})\\\vdots &\vdots&\ddots &\vdots\\\phi_{\alpha_N}(\mathbf{r_1})&\phi_{\alpha_N}(\mathbf{r_2})&\dots &\phi_{\alpha_N}(\mathbf{r_N})\\\end{vmatrix}$$
\end{remarks}
\begin{defi}[Hanbury-Brown and Twiss experiment]
This experiment observes boson clumping, by measuring the probability of observing two photons simultaneously at different points in a beam of incoherent light. A half-silvered mirror splits the beam into two identical beams. The amplitude for a photon to be transmitted, or reflected by the mirror is $1/\sqrt{2}$. The product of light intensities observed in detector 1 at time $t$ and in detector 2 at a later time $t+\tau$, averaged over $t$, keeping $\tau$ fixed - determines the relative probability of observing two photons at two points separated by distance $c\tau$ in the beam, i.e. $\langle I_1(t)I_2(t+\tau)\rangle$ same functional dependence as $g(r=c\tau)$.
\end{defi}
The bosonic nature of photons (photon bunching) is already contained in the superposition principle obeyed by classical electromagnetic fields.
\begin{eg}
Consider two emitters A and B. Assume A emits coherent light with amplitude $\alpha$ and wavenumber $k$, B emits with $\beta$ and $k'$, and the relative phase of these two sources is random with light from each having the same polarization, then the total amplitude falling on detector 1 and 2 respectively are $\alpha e^{i\mathbf{k}\cdot\mathbf{r_i}}+\beta e^{i\mathbf{k'}\cdot\mathbf{r_i}}$ for $i=1,2$. The intensity received by detector 1 is
$$I_1=|\alpha|^2+|\beta|^2+2\text{Re}[\alpha^*\beta e^{i(\mathbf{k'}\cdot\mathbf{r_1'}-\mathbf{k}\cdot\mathbf{r_1})}]\implies\langle I_1\rangle=|\alpha|^2+|\beta|^2$$
Similar for detector 2. The product is
$$I_1I_2=|\alpha^2e^{i\mathbf{k}\cdot(\mathbf{r_1}+\mathbf{r_2})}+\beta^2e^{i\mathbf{k'}\cdot(\mathbf{r_1'}+\mathbf{r_2'})}+\alpha\beta(e^{i\mathbf{k}\cdot\mathbf{r_1}}e^{i\mathbf{k'}\cdot\mathbf{r_2'}}+e^{i\mathbf{k'}\cdot\mathbf{r_1'}}e^{i\mathbf{k}\cdot\mathbf{r_2}})|^2$$
For a well-collimated beam, $r_1-r_2\approx r_1'-r_2'$, so 
\begin{align}
\langle I_1I_2\rangle&=|\alpha|^4+|\beta|^4+|\alpha|^2|\beta|^2|e^{i\mathbf{k}\cdot\mathbf{r_1}}e^{i\mathbf{k'}\cdot\mathbf{r_2'}}+e^{i\mathbf{k'}\cdot\mathbf{r_1'}}e^{i\mathbf{k}\cdot\mathbf{r_2}}|^2\nonumber\\&=\langle I_1I_2\rangle+2|\alpha|^2|\beta|^2\cos[\mathbf{k'}\cdot(\mathbf{r_1'}-\mathbf{r_2'})-\mathbf{k}\cdot(\mathbf{r_1}-\mathbf{r_2})]\nonumber\\&\approx\langle I_1I_2\rangle+2|\alpha|^2|\beta|^2\cos[(\mathbf{k'}-\mathbf{k})\cdot(\mathbf{r_1}-\mathbf{r_2})]\nonumber
\end{align}
The correlated intensities depends on the relative separation of the two detectors, maximum when they are at the same point. The $\alpha^2$ term and $\beta^2$ term both come from A and B respectively, $\alpha\beta$ term come from A and B each.
\end{eg}
\begin{eg}[Two-photon interference]
Consider a beamsplitter whose effect can be described by a unitary matrix (1D scattering problem)
$$\begin{pmatrix}a_{\text{out}}\\b_{\text{out}}\\\end{pmatrix}=S\begin{pmatrix}a_{\text{in}}\\b_{\text{in}}\\\end{pmatrix},~S=\frac{1}{\sqrt{2}}\begin{pmatrix}1&-1\\1&1\\\end{pmatrix}\implies|A,\text{in}\rangle\rightarrow\frac{1}{\sqrt{2}}(|A,\text{out}\rangle+|B,\text{out}\rangle),~|B,\text{in}\rangle\rightarrow\frac{1}{\sqrt{2}}(|B,\text{out}\rangle-|A,\text{out}\rangle)$$
Suppose the incident wave is two-photons, then one can show:
$$|\psi_{\text{in}}\rangle=\frac{1}{\sqrt{2}}\bigg(|A,\text{in}\rangle_1|B,\text{in}\rangle_2+|B,\text{in}\rangle_1|A,\text{in}\rangle_2\bigg)\implies|\psi_{\text{out}}\rangle=\frac{1}{2\sqrt{2}}2\bigg(|B,\text{out}\rangle_1|B,\text{out}\rangle_2-|A,\text{out}\rangle_1|A,\text{out}\rangle_2\bigg)$$
'Bunching' of identical photons in the same channel.
\end{eg}
\begin{eg}[Scattering of two spinless bosons]
The overall two-particle wavefunction is $\Psi(\mathbf{r_1},\mathbf{r_2})=e^{i\mathbf{p}\cdot(\mathbf{r_1}+\mathbf{r_2})}\psi(\mathbf{r})$, where $\mathbf{p}$ is the momentum of the centre of momentum of the two particles, $\mathbf{r}=\mathbf{r_1}-\mathbf{r_2}$ is the relative separation. $e^{i\mathbf{p}\cdot(\mathbf{r_1}+\mathbf{r_2})}$ is trivially symmetric. For $\Psi(\mathbf{r_1},\mathbf{r_2})=\Psi(\mathbf{r_2},\mathbf{r_1})\implies\psi(\mathbf{r_1}-\mathbf{r_2})=\psi(\mathbf{r_2}-\mathbf{r_1})\implies\psi(\mathbf{r})=\psi(-\mathbf{r})$, i.e. states of even parity. Since $Y_{\ell,m}(-\mathbf{\hat{r}})=(-1)^\ell Y_{\ell,m}(\mathbf{\hat{r}})$, the only possible orbital angular momentum eigenstates are those with even $\ell$. The asymptotic scattering wavefunction (Eqn.~\ref{asymptotic}) is $\psi(\mathbf{r})=e^{i\mathbf{k}\cdot\mathbf{r}}+e^{-i\mathbf{k}\cdot\mathbf{r}}+[f(\theta)+f(\pi-\theta)]\frac{e^{ikr}}{r}$, where we cannot distinguish the target particle from the incident particle in the centre of momentum frame. One particle emerges at $\theta$ and an identical one emerges at $\pi-\theta$ (for $\mathbf{r}\rightarrow-\mathbf{r}$, $|\mathbf{r}|\rightarrow|\mathbf{r}|\implies\theta\rightarrow\pi-\theta$). The differential cross-section is
$$\frac{d\sigma}{d\Omega}=|f(\theta)+f(\pi-\theta)|^2=|f(\theta)|^2+|f(\pi-\theta)|^2+2\text{Re}[f^*(\theta)f(\pi-\theta)]$$
The last term is zero for two distinguishable particles. If $f(\theta)\sim e^{i\lambda\theta}$, this last term gives $\cos(\lambda(\pi-2\theta))$. In partial waves (Eqns.~\ref{Rayleigh},~\ref{Rayleigh2}), 
$$P_\ell(\cos(\pi-\theta))=P_\ell(-\cos\theta)=(-1)^\ell P_\ell(\cos\theta)\implies f(\theta)+f(\pi-\theta)=2\sum_{\text{even}~\ell}i^\ell(2\ell+1)P_\ell(\cos\theta)f_\ell$$
In the special case of $\theta=\pi/2$, $\frac{d\sigma}{d\Omega}=4|f(\pi/2)|^2$, which is two times more than that of two distinguishable particles.
\end{eg}
\begin{eg}[Scattering of two spin-1/2 fermions]
Suppose the interaction between particles does not depend on their spin. The overall wavefunction is $\Psi(\mathbf{r_1},\mathbf{r_2})=e^{i\mathbf{p}\cdot(\mathbf{r_1}+\mathbf{r_2})}\psi(\mathbf{r})\chi(s_1,s_2)$. 
\begin{itemize}
    \item For a spin singlet $\chi$, the spatial wavefunction $\psi(\mathbf{r})$ must be symmetric, thus giving $\frac{d\sigma}{d\Omega}$ to be identical to two spinless bosons, i.e. $|f(\theta)+f(\pi-\theta)|^2$.
    \item For a spin triplet $\chi$, the spatial wavefunction $\psi(\mathbf{r})$ must be anti-symmetric (only odd $\ell$ orbital angular momentum eigenstates are allowed), thus giving $\frac{d\sigma}{d\Omega}$ to be 
    $$\frac{d\sigma}{d\Omega}=|f(\theta)-f(\pi-\theta)|^2=|f(\theta)|^2+|f(\pi-\theta)|^2-2\text{Re}[f^*(\theta)f(\pi-\theta)]$$
    Again, for the special case of $\theta=\pi/2$, $\frac{d\sigma}{d\Omega}=0$, i.e. can never scatter through 90\degree. Again, in partial waves, $f(\theta)-f(\pi-\theta)=2\sum_{\text{odd}~\ell}i^\ell(2\ell+1)P_\ell(\cos\theta)f_\ell$.
\end{itemize}
If the spins of either target or beam particles are unpolarized, then the spins of the scattering particles are equally likely to be found in each of the four spin states. The observed cross-section is an average:
$$\frac{d\sigma}{d\Omega}=\frac{3}{4}\frac{d\sigma}{d\Omega}\bigg|_{\text{triplet}}+\frac{1}{4}\frac{d\sigma}{d\Omega}\bigg|_{\text{singlet}}=|f(\theta)|^2+|f(\pi-\theta)|^2-\text{Re}[f^*(\theta)f(\pi-\theta)]$$
\end{eg}
\begin{eg}[Particles on a ring]
Consider non-interacting particles on a ring of circumference $L$. The single particle eigenstates are $\phi_n(x)=\frac{1}{\sqrt{L}}e^{2\pi inx/L}$ with $n=0,\pm1,\pm2,\dots$ with energies $E_n=\frac{\hbar^2n^2}{2mL^2}$. 
\begin{itemize}
\item For bosons, the $N$ particle ground state consists of every particle in the state $\phi_0$ with zero energy, so $N_0=N$. Trivially, $\Psi^S(x_1,x_2\dots,x_N)=1$.
\item For fermions, the occupation of each level is at most one. The lowest energy is obtained by filling each level with one particle, starting at the bottom. If we have an odd number of particles, this means filling the levels with $n=-(N-1)/2,-(N-3)/2,\dots,-1,0,1,\dots,(N-1)/2$ (for an even number we have to decide whether to put the last particle at $n=\pm N/2$). Introduce $z_i=e^{2\pi ix_i/L}$, the Slater determinant is
$$\begin{vmatrix}z_1^{-(N-1)/2}&\dots &z_N^{-(N-1)/2}\\\vdots &\ddots &\vdots\\z_1^{(N-1)/2}&\dots &z_N^{(N-1)/2}\\\end{vmatrix}$$
For the simple case of three particles,
$$\begin{vmatrix}z_1^{-1}&z_2^{-1}&z_3^{-1}\\1&1&1\\z_1&z_2&z_3\\\end{vmatrix}=\frac{z_1}{z_2}-\frac{z_2}{z_1}+\frac{z_3}{z_1}-\frac{z_1}{z_3}+\frac{z_2}{z_3}-\frac{z_3}{z_2}=\bigg(\sqrt{\frac{z_3}{z_1}}-\sqrt{\frac{z_1}{z_3}}\bigg)\bigg(\sqrt{\frac{z_1}{z_2}}-\sqrt{\frac{z_2}{z_1}}\bigg)\bigg(\sqrt{\frac{z_2}{z_3}}-\sqrt{\frac{z_3}{z_2}}\bigg)$$
Since $\sqrt{z_j/z_k}=\sqrt{\exp(2\pi i(x_j-x_k)/L)}$, we have $\propto\sin(\pi|x_1-x_2|/L)\sin(\pi|x_3-x_1|/L)\sin(\pi|x_2-x_3|/L)$. The nodal surfaces $x_i=x_j$ (vanishing of the wavefunction) is consistent with the Pauli principle. Due to the periodic boundary conditions, the three dimensional space of particle coordinates is divided into two regions, corresponding to the even (123), (231), (312) and odd (132), (321), (213) permutations. For generic (odd) $N$, the Slater determinant gives
$$(z_1z_2\dots z_N)^{-(N-1)/2}\begin{vmatrix}1&\dots&1\\z_1&\dots&z_N\\\vdots &\ddots&\vdots\\z_1^{N-1}&\dots&z_N^{N-1}\\\end{vmatrix}=(z_1z_2\dots z_N)^{-(N-1)/2}\prod_{i<j}^N(z_i-z_j)\propto\prod_{i<j}^N\frac{z_i-z_j}{2iz_i^{1/2}z_j^{1/2}}$$
which is equal to $\prod_{i<j}^N\sin\frac{\pi|x_i-x_j|}{L}$, where we used the Vandermonde determinant result.
\end{itemize}
\end{eg}
\subsection{Second quantization}
\subsubsection{States}
To avoid the need to totally symmetrize the states, we would rather label states purely by the occupation numbers that describe how particles populate the single particle states, i.e. $|\psi_{\{\alpha_i\}}\rangle\rightarrow|\{N_\alpha\}\rangle$ with the total number of particles being $N=\sum_\alpha N_\alpha$.
\begin{eg}
Consider a many-boson state with $n_0$ particles in the state $\phi_0$ and $n_1$ particles in the state $\phi_1$, $|n_0,n_1\rangle$. We can define harmonic oscillator operators that create and annihilate a particle in the respective states:
$$a_0|n_0,n_1\rangle=\sqrt{n_0}|n_0-1,n_1\rangle,\quad a_0^\dag|n_0,n_1\rangle=\sqrt{n_0+1}|n_0+1,n_1\rangle$$
$$a_1|n_0,n_1\rangle=\sqrt{n_1}|n_0,n_1-1\rangle,\quad a^\dag_1|n_0,n_1\rangle=\sqrt{n_1+1}|n_0,n_1+1\rangle$$
where the operators satisfy the usual harmonic oscillator commutation relations $[a_0,a_0^\dag]=1$, $[a_1,a_1^\dag]=1$. Further,
$$[a_0,a_1],~[a_0^\dag,a_1^\dag],~[a_0,a_1^\dag],~[a_0^\dag,a_1]=0$$
where $a_0$ and $a_1$ act on different subspaces. In general, $|n_0,n_1\rangle=\frac{(a_1^\dag)^{n_1}}{\sqrt{n_1!}}\frac{(a_0^\dag)^{n_0}}{\sqrt{n_0!}}|0,0\rangle$. We can define the number operator to be $N=a_0^\dag a_0+a_1^\dag a_1$ such that $N|n_1,n_2\rangle=(n_1+n_2)|n_1,n_2\rangle$. For fermions, the only allowed states (given by the quantum statistics) are $\{|0,0\rangle,|0,1\rangle,|1,0\rangle,|1,1\rangle\}$ where the ladder operators' actions are
$$a_1^\dag|0,0\rangle=|0,1\rangle,~a_1^\dag|1,0\rangle=|1,1\rangle,~a_1^\dag|0,1\rangle=0,~a_1^\dag|1,1\rangle=0$$
$$a_1 |0,0\rangle=0,~a_1|1,0\rangle=0,~a_1|0,1\rangle=|0,0\rangle,~a_1|1,1\rangle=|1,0\rangle$$
$$a_0^\dag|0,0\rangle=|1,0\rangle,~a_0^\dag|1,0\rangle=0,~a_0|1,0\rangle=|0,0\rangle,~a_0|0,0\rangle=0$$
However, we must be careful in defining how $a_0$ and $a_0^\dag$ act on the states $|0,1\rangle$ and $|1,1\rangle$, which already have a particle in $\phi_1$. Consider the state $|1,1\rangle$. By interchanging the two particles, the state will incur a negative sign according to fermionic statistics. This swapping can be achieved by $a_0^\dag a_1^\dag a_0a_1$:
$$a_0^\dag a_1^\dag a_0a_1|1,1\rangle=a_0^\dag a_1^\dag a_0|1,0\rangle=a_0^\dag|0,1\rangle=-|1,1\rangle$$
This gives $a_0^\dag|0,1\rangle=-|1,1\rangle$, similarly we have $a_0|1,1\rangle=-|0,1\rangle$. The operators satisfy the anti-commutation relations:
$$\{a_0,a_0^\dag\}=1,~\{a_1,a_1^\dag\}=1,~\{a_0,a_0\}=\{a_1,a_1\}=0,~\{a_0^\dag,a_0^\dag\}=\{a_1^\dag,a_1^\dag\}=0$$
Further, $\{a_0,a_1\}=\{a_0^\dag,a_1^\dag\}=0$ and $\{a_0,a_1^\dag\}=\{a_0^\dag,a_1\}=0$. One can check, for instance:
$$\{a_0,a_0^\dag\}|0,1\rangle=a_0a_0^\dag|0,1\rangle-a_0^\dag a_0|0,1\rangle=-a_0|1,1\rangle-0=|0,1\rangle$$
In general, $|n_0,n_1\rangle=(a_1^\dag)^{n_1}(a_0^\dag)^{n_0}|0,0\rangle$ where there are no factorials since $n!=1$ for $n=0$ or 1 (allowed occupancy numbers for fermions).
\end{eg}
\begin{defi}[Generic multi-particle quantum states]
A generic multi-particle quantum state is
$$|n_0,n_1,n_2,\dots\rangle=\dots\frac{(a_2^\dag)^{n_2}}{\sqrt{n_2!}}\frac{(a_1^\dag)^{n_1}}{\sqrt{n_1!}}\frac{(a_0^\dag)^{n_0}}{\sqrt{n_0!}}|0\rangle$$
where $|0\rangle:=|0\dots0\rangle$ is the vacuum state in short. Trivially for fermions, the factorial gives 1. The ladder operators for bosons and fermions respectively satisfy:
$$[a_i,a_j^\dag]=\delta_{i,j},\quad[a_i,a_j]=0=[a_i^\dag,a_j^\dag]$$
$$\{a_i,a_j^\dag\}=\delta_{i,j},\quad\{a_i,a_j\}=0=\{a_i^\dag,a_j^\dag\}$$
For both, $[a_i^\dag a_i,a_j^\dag a_j]=0$ since the occupancy in each states are independent.
\end{defi}
\begin{remarks}
We generally want the number of particles to be conserved, so observables of interest are typically products of equal number of creation and annihilation operators whose overall effect is to redistribute particles among the single particle states. 
\end{remarks}
\begin{defi}[Change of basis]
Suppose we want to move to a different basis of single particle states $\{|\tilde{\phi}_\alpha\rangle\}$, corresponding to a unitary transformation
\begin{equation}
    |\tilde{\phi}_\alpha\rangle=\sum_\beta\langle\phi_\beta|\tilde{\phi}_\alpha\rangle|\phi_\beta\rangle\label{changeofbasis}
\end{equation}
\end{defi}
\begin{eg}
Often, we will work in the basis of position eigenstates $\{|\mathbf{r}\rangle\}$, the position representation of creation and annihilation operators are called field operators:
$$\psi^\dag(\mathbf{r})=\sum_\beta\phi_\beta^*(\mathbf{r})a_\beta^\dag,\quad\psi(\mathbf{r})=\sum_\beta\phi_\beta(\mathbf{r})a_\beta$$
\end{eg}
We sometimes include the spin degree of freedom $s$.
\begin{prop}[Field operators]
\begin{equation}
\psi_s^\dag(\mathbf{r})=\sum_\mathbf{p}\frac{e^{-i\mathbf{p}\cdot\mathbf{r}}}{\sqrt{V}}a_{\mathbf{p},s}^\dag,\quad \psi_s(\mathbf{r})=\sum_\mathbf{p}\frac{e^{i\mathbf{p}\cdot\mathbf{r}}}{\sqrt{V}}a_{\mathbf{p},s}\label{field}
\end{equation}
\end{prop}
\begin{proof}
Let the complete set of states be plane waves in a box subjected to periodic boundary conditions, $\phi_\mathbf{p}(\mathbf{r})=\frac{e^{i\mathbf{p}\cdot\mathbf{r}}}{\sqrt{V}}$, where $p_x=\frac{2\pi n_x}{L_x}$ are quantized with $n_x\in\mathbb{Z}\cup\{0\}$. $a^\dag_{\mathbf{p},s}$ adds a particle with momentum $\mathbf{p}$ and spin orientation $s$ to the box. The amplitude at the point $\mathbf{r'}$ for finding the particle added by $a^\dag_{\mathbf{p},s}$ is $\phi_\mathbf{p}(\mathbf{r'})$. The field operator is 
$$\psi_s^\dag(\mathbf{r})=\sum_\mathbf{p}\phi^*_\mathbf{p}(\mathbf{r})a^\dag_{\mathbf{p},s}=\sum_\mathbf{p}\frac{e^{-i\mathbf{p}\cdot\mathbf{r}}}{\sqrt{V}}a^\dag_{\mathbf{p},s}$$
This field operator adds a particle to the system in a superposition of momentum states with amplitude $e^{-i\mathbf{p}\cdot\mathbf{r}}/\sqrt{V}$. Similar for $\psi_s(\mathbf{r})$.
\end{proof}
\begin{remarks}
The amplitude at point $\mathbf{r'}$ for finding the particle added by $\psi_s^\dag(\mathbf{r})$ is a coherent sum of amplitudes $e^{i\mathbf{p}\cdot\mathbf{r'}}/\sqrt{V}$ with coefficients $e^{-i\mathbf{p}\cdot\mathbf{r}}/\sqrt{V}$. The net amplitude is $\sum_{\mathbf{p}}\frac{e^{-i\mathbf{p}\cdot\mathbf{r}}}{\sqrt{V}}\frac{e^{i\mathbf{p}\cdot\mathbf{r'}}}{\sqrt{V}}=\delta(\mathbf{r}-\mathbf{r'})$, i.e. $\psi_s^\dag(\mathbf{r})$ adds a particle of spin orientation $s$ at point $\mathbf{r}$ by adding all the amplitude at point $\mathbf{r}$. We could have also written it as
$$a^\dag_{\mathbf{p},s}=\int\frac{e^{i\mathbf{p}\cdot\mathbf{r}}}{\sqrt{V}}\psi_s^\dag(\mathbf{r})d^3\mathbf{r},\quad a_{\mathbf{p},s}=\int\frac{e^{-i\mathbf{p}\cdot\mathbf{r}}}{\sqrt{V}}\psi_s(\mathbf{r})d^3\mathbf{r}$$
To add a particle with momentum $\mathbf{p}$, one adds a particle at different points $\mathbf{r}$ with relative amplitude $\frac{e^{i\mathbf{p}\cdot\mathbf{r}}}{\sqrt{V}}$.
\end{remarks}
\begin{cor}
\begin{equation}
    \psi_s(\mathbf{r})\psi_{s'}(\mathbf{r'})\mp\psi_{s'}(\mathbf{r'})\psi_s(\mathbf{r})=0=\psi_s^\dag(\mathbf{r})\psi_{s'}(\mathbf{r})\mp\psi_{s'}^\dag(\mathbf{r'})\psi_s^\dag(\mathbf{r}),\quad\psi_s(\mathbf{r})\psi_{s'}^\dag(\mathbf{r'})\mp\psi_{s'}^\dag(\mathbf{r'})\psi_s(\mathbf{r})=\delta(\mathbf{r}-\mathbf{r'})\delta_{s,s'}\label{commutationfield}
\end{equation}
\end{cor}
\begin{proof}
For bosons ($-$) and fermions ($+$), the commutation relations satisfy $a_{\mathbf{p},s}a_{\mathbf{p'},s'}\mp a_{\mathbf{p'},s'}a_{\mathbf{p},s}=0$. Finally,
$$\psi_s(\mathbf{r})\psi_{s'}^\dag(\mathbf{r'})\mp\psi_{s'}^\dag(\mathbf{r'})\psi_s(\mathbf{r})=\sum_{\mathbf{p},\mathbf{p'}}\frac{e^{i\mathbf{p}\cdot\mathbf{r}}e^{-i\mathbf{p'}\cdot\mathbf{r'}}}{V}(a_{\mathbf{p},s}a_{\mathbf{p'},s'}^\dag\mp a_{\mathbf{p'},s'}^\dag a_{\mathbf{p},s})=\sum_{\mathbf{p},\mathbf{p'}}\frac{e^{i\mathbf{p}\cdot\mathbf{r}}e^{-i\mathbf{p'}\cdot\mathbf{r'}}}{V}\delta_{\mathbf{p},\mathbf{p'}}\delta_{s,s'}=\delta(\mathbf{r}-\mathbf{r'})\delta_{s,s'}$$
Adding a particle commutes (or anticommutes) with removing a particle.
\end{proof}
\begin{remarks}
For bosons, adding a particle at $\mathbf{r}$ is an operation that commutes with adding a particle at $\mathbf{r'}$. For fermions, these operations commute except for a change of sign of the state.
\end{remarks}
\begin{cor}
Removing a particle at $\mathbf{r}$ can work only if $\mathbf{r}=\mathbf{r_n}$ or $\mathbf{r}=\mathbf{r_{n-1}}$, $\dots$, or $\mathbf{r}=\mathbf{r_1}$. What remains is the correctly symmetrized combination of $n-1$ particle states. Similarly, adding a particle gives a correctly symmetrized $n+1$ particle state.
\end{cor}
\begin{proof}
Let the generic $n$-particle state be
$$|\mathbf{r_1}\mathbf{r_2},\dots,\mathbf{r_n}\rangle=\frac{1}{\sqrt{n!}}\psi^\dag(\mathbf{r_n})\dots\psi^\dag(\mathbf{r_2})\psi^\dag(\mathbf{r_1})|0\rangle$$
Since the field operators for fermions, for instance, anti-commute (Eqn.~\ref{commutationfield}), then
$$\psi^\dag(\mathbf{r_2})\psi^\dag(\mathbf{r_1})=-\psi^\dag(\mathbf{r_1})\psi^\dag(\mathbf{r_2})\implies|\mathbf{r_2},\mathbf{r_1},\mathbf{r_3},\dots,\mathbf{r_n}\rangle=-|\mathbf{r_1},\mathbf{r_2},\mathbf{r_3},\dots\mathbf{r_n}\rangle$$
The resulting states satisfy the correct symmetry, by construction. We now add a particle:
$$\psi^\dag(\mathbf{r})|\mathbf{r_1},\dots,\mathbf{r_n}\rangle=\sqrt{n+1}|\mathbf{r_1},\dots,\mathbf{r_n},\mathbf{r}\rangle$$
Similarly, we remove a particle:
\begin{eqnarray}
    &&\psi(\mathbf{r})|\mathbf{r_1},\dots,\mathbf{r_n}\rangle\nonumber\\&=&\frac{1}{\sqrt{n!}}\psi(\mathbf{r})\psi^\dag(\mathbf{r_n})\dots\psi^\dag(\mathbf{r_1})|0\rangle\nonumber\\&=&\frac{1}{\sqrt{n!}}\bigg[\delta(\mathbf{r}-\mathbf{r_n})\pm\psi^\dag(\mathbf{r_n})\psi(\mathbf{r})\bigg]\psi^\dag(\mathbf{r_{n-1}})\dots\psi^\dag(\mathbf{r_1})|0\rangle\nonumber\\&=&\frac{1}{\sqrt{n!}}\bigg[\delta(\mathbf{r}-\mathbf{r_n})|\mathbf{r_1},\dots,\mathbf{r_{n-1}}\rangle\pm\delta(\mathbf{r}-\mathbf{r_{n-1}})|\mathbf{r_1},\dots,\mathbf{r_{n-2}},\mathbf{r_n}\rangle+\dots+(\pm1)^{n-1}\delta(\mathbf{r}-\mathbf{r_1})|\mathbf{r_2},\dots,\mathbf{r_n}\rangle\bigg]\nonumber
\end{eqnarray}
which involves $n-1$ swaps.
\end{proof}
\begin{remarks}
$$\langle\mathbf{r_1'},\dots,\mathbf{r_n'}|\mathbf{r_1},\dots,\mathbf{r_n}\rangle=\frac{\delta_{n,n'}}{n!}\sum_P(\pm1)^PP\delta(\mathbf{r_1}-\mathbf{r_1'})\delta(\mathbf{r_2}-\mathbf{r_2'})\dots\delta(\mathbf{r_n}-\mathbf{r_n'})$$
where we summed over all permutations of $\{\mathbf{r_1'},\dots,\mathbf{r_n'}\}$. $(\pm1)^P=+1$ for bosons and $\sgn(P)$ for fermions. $n'$ must be $n$ since states with different numbers of particles are orthogonal. We can write the $n$-particle state
\begin{equation}
|\Phi\rangle=\int\phi(\mathbf{r_1},\dots,\mathbf{r_n})|\mathbf{r_1},\dots,\mathbf{r_n}\rangle d^3\mathbf{r_1}\dots d^3\mathbf{r_n}\label{multiparticlestate}
\end{equation}
$|\Phi\rangle$ is correctly symmetrized, even if the $\phi(\mathbf{r_1},\dots,\mathbf{r_n})$ used to construct $|\Phi\rangle$ is not symmetrized. The amplitude for observing particles at $\mathbf{r_1'},\dots,\mathbf{r_n'}$ if they are in the state $|\Phi\rangle$:
$$\langle\mathbf{r_1'},\dots,\mathbf{r_n'}|\Phi\rangle=\int\phi(\mathbf{r_1},\dots\mathbf{r_n})\langle\mathbf{r_1'},\dots,\mathbf{r_n'}|\mathbf{r_1},\dots,\mathbf{r_n}\rangle d^3\mathbf{r_1}\dots d^3\mathbf{r_n}=\frac{1}{n!}\sum_P(\pm1)^P\phi(\mathbf{r_1'},\dots,\mathbf{r_n'})$$
and it is equal to $\phi(\mathbf{r_1'},\dots,\mathbf{r_n'})$ if $\phi(\mathbf{r_1'},\dots,\mathbf{r_n'})$ already properly symmetrized (all $n!$ terms are equal), and it is equal to 1 if it is normalized as well. The many-particle identity operator on properly symmetrized $n$ particle states is
$$\Id_n=\int|\mathbf{r_1},\dots,\mathbf{r_n}\rangle\langle\mathbf{r_1},\dots,\mathbf{r_n}|d^3\mathbf{r_1}\dots d^3\mathbf{r_n}$$
\end{remarks}
\subsubsection{Operators}
\begin{defi}[One-particle operator]
A one-particle operator consists of a sum of terms, one for each particle, with each term acting solely on that particle's coordinate.
\end{defi}
\begin{prop}
The action of the operator A is to take the particle from state $|\phi_\beta\rangle$ to a superposition of states with amplitudes given by $A_{\alpha\beta}=\langle\phi_\alpha|A|\phi_\beta\rangle$:
\begin{equation}
    A=\sum_{\alpha,\beta}A_{\alpha,\beta}a_\alpha^\dag a_\beta\label{changeofbasis2}
\end{equation}
\end{prop}
\begin{proof}
$Aa_\beta^\dag|0\rangle=A|\phi_\beta\rangle=\sum_\alpha|\phi_\alpha\rangle A_{\alpha\beta}=\sum_{\alpha,\beta}A_{\alpha,\beta}a^\dag_\alpha|0\rangle$, result follows.
\end{proof}
\begin{remarks}
In the first quantization representation,
\begin{align}
A|\Psi_{\beta_1,\dots,\beta_N}^{S/A}\rangle&=\sum_{\alpha,\beta}A_{\alpha,\beta}a_\alpha^\dag a_\beta\mathcal{N}a_{\beta_1}^\dag\dots a_{\beta_N}^\dag|0\rangle\nonumber\\&=\mathcal{N}\sum_{\alpha,\beta}A_{\alpha,\beta}a_\alpha^\dag a^\dag_{\beta_1}\dots a_{\beta_N}^\dag a_\beta|0\rangle+\mathcal{N}\sum_{i=1}^N\sum_{\alpha,\beta}A_{\alpha,\beta}a^\dag_{\beta_1}\dots[a_\alpha^\dag a_\beta,a_{\beta_i}^\dag]_\mp\dots a^\dag_{\beta_N}|0\rangle\nonumber\\&=\mathcal{N}\sum_{i=1}^N\sum_{\alpha,\beta}A_{\alpha,\beta}a^\dag_{\beta_1}\dots\delta_{\beta,\beta_i}a_\alpha^\dag\dots a^\dag_{\beta_N}|0\rangle\nonumber\\&=\sum_{i=1}^N\sum_{\alpha,\beta}A_{\alpha,\beta}\delta_{\beta,\beta_i}|\Psi^{S/A}_{\beta_1,\dots,\beta_{i-1},\alpha,\beta_{i+1},\dots,\beta_N}\rangle\nonumber
\end{align}
which also gives $A=\sum_{\alpha,\beta}A_{\alpha,\beta}\delta_{\beta,\beta_i}$.
\end{remarks}
\begin{eg}\leavevmode
\begin{enumerate}
    \item kinetic energy: sum $\frac{p^2}{2m}$ weighted by the number of particles with momentum $\mathbf{p}$, i.e. $T=\sum_{\mathbf{p},s}\frac{p^2}{2m}a^\dag_{\mathbf{p},s}a_{\mathbf{p},s}$. Express it in terms of field operators:
    $$T=\frac{1}{2mV}\sum_{\mathbf{p},s}\int(\boldsymbol{\nabla}e^{i\mathbf{p}\cdot\mathbf{r}})\cdot(\boldsymbol{\nabla'}e^{-i\mathbf{p}\cdot\mathbf{r'}})\psi_s^\dag(\mathbf{r})\psi_s(\mathbf{r'})d^3\mathbf{r}d^3\mathbf{r'}=\frac{1}{2m}\int\boldsymbol{\nabla}\psi^\dag(\mathbf{r})\cdot\boldsymbol{\nabla}\psi(\mathbf{r})d^3\mathbf{r}$$
    where $\mathbf{p}e^{i\mathbf{p}\cdot\mathbf{r}}=-i\boldsymbol{\nabla}e^{i\mathbf{p}\cdot\mathbf{r}}$ and we integrated by parts.
    \item current density: $\mathbf{j}(\mathbf{r})=\frac{1}{2im}(\psi^\dag(\mathbf{r})\boldsymbol{\nabla}\psi(\mathbf{r})-(\boldsymbol{\nabla}\psi^\dag(\mathbf{r}))\psi(\mathbf{r}))$.
    \item spin density: $\mathbf{S}(\mathbf{r})=\frac{1}{2}\sum_{s,s'}\psi_s^\dag(\mathbf{r})\boldsymbol{\sigma}_{s,s'}\psi_{s'}(\mathbf{r'})$.
    \item To evaluate $[S_i(\mathbf{r}),S_j(\mathbf{r'})]$, it is helpful to use the identity involving $[AB,CD]$ and the field operator commutation relations.
\end{enumerate}
\end{eg}
\begin{remarks}
In second quantization, one-particle wavefunctions appear to have become operators which create and annihilate particles, while single particle expectation values appear to have become operators for physical quantities.
\end{remarks}
\begin{eg}[Two-body potential]
Consider a Hamiltonian with a two-body potential $V(\mathbf{r}-\mathbf{r'})$ which we can treat as a perturbation. The interaction energy operator is
$$V=\frac{1}{2}\sum_{s,s'}\int V(\mathbf{r}-\mathbf{r'})\psi_s^\dag(\mathbf{r})\psi_{s'}^\dag(\mathbf{r'})\psi_{s'}(\mathbf{r'})\psi_s(\mathbf{r})d^3\mathbf{r}d^3\mathbf{r'}$$
We first remove particles from $\mathbf{r}$ and $\mathbf{r'}$, count $V(\mathbf{r}-\mathbf{r'})$, and then replace the particles, adding the last removed particle first. Sum over all pairs of $\mathbf{r},\mathbf{r'}$. 
\end{eg}
\begin{eg}[Bogoliubov transformation]
For the Hamiltonian 
$$H=\varepsilon(a^\dag a+b^\dag b)+\Delta(a^\dag b^\dag+ba)$$
where the particle number is not conserved. A unitary transformation is possible: 
$$\alpha=UaU^\dag,\quad\beta=U bU^\dag,~U=\exp[\kappa(a^\dag b^\dag-ba)]$$
where $\alpha=a\cosh\kappa-b^\dag\sinh\kappa$, $\beta=b\cosh\kappa-a^\dag\sinh\kappa$ for bosons and $\alpha=a\cos\kappa+b^\dag\sin\kappa$, $\beta^\dag =-a\sin\kappa+b^\dag\cos\kappa$. For the transformation to preserve particle number, an addition criterion $\tanh2\kappa=-\Delta/\varepsilon$ and $\tan2\kappa=\Delta/\varepsilon$ for bosons and fermion respectively, is reqired (to kill off $\alpha\beta$ or $\alpha^\dag\beta^\dag$ terms).
\end{eg}
\begin{eg}[Fourier transform]
Suppose $\rho(\mathbf{r})=\sum_{\mathbf{k}}\phi_{\mathbf{k}}^*(\mathbf{r})a_{\mathbf{k}}^\dag(\mathbf{r})\sum_{\mathbf{k'}}\phi_{\mathbf{k'}}(\mathbf{r})a_{\mathbf{k'}}$, then $\rho_{\mathbf{q}}=\int\rho(\mathbf{r})e^{-i\mathbf{q}\cdot\mathbf{r}}d^3\mathbf{r}$:
$$\int\sum_{\mathbf{p},\mathbf{k}}\phi_{\mathbf{p}}^*(\mathbf{r})a_{\mathbf{p}}^\dag\phi_{\mathbf{k}}(\mathbf{r})a_{\mathbf{k}}e^{-i\mathbf{q}\cdot\mathbf{r}}d^3\mathbf{r}=\frac{1}{V}\int\sum_{\mathbf{p},\mathbf{k}}a^\dag_{\mathbf{p}}a_{\mathbf{k}}e^{i(\mathbf{k}-\mathbf{p}-\mathbf{q})\cdot\mathbf{r}}d^3\mathbf{r}=\sum_{\mathbf{p},\mathbf{k}}\delta_{\mathbf{k},\mathbf{p}+\mathbf{q}}a_{\mathbf{p}}^\dag a_{\mathbf{k}}=\sum_{\mathbf{k}}a^\dag_{\mathbf{k}-\mathbf{q}}a_{\mathbf{k}}$$
where we can relabel indices to obtain $\sum_{\mathbf{k}}a_{\mathbf{k}-0.5\mathbf{q}}^\dag a_{\mathbf{k}+0.5\mathbf{q}}$.
\end{eg}
\newpage
\subsection{Density and correlations}
\begin{prop}[Density of particles]
\begin{equation}
    \rho(\mathbf{r})=\sum_s\psi^\dag_s(\mathbf{r})\psi_s(\mathbf{r})\label{densityfield}
\end{equation}
\end{prop}
\begin{proof}
Evaluate $\langle\Phi'|\rho(\mathbf{r})|\Phi\rangle$:
\begin{align}
    \langle\Phi'|\psi^\dag(\mathbf{r})\Id_n\psi(\mathbf{r})|\Phi\rangle&=\int\langle\Phi'|\psi^\dag(\mathbf{r})|\mathbf{r_1},\dots,\mathbf{r_{n-1}}\rangle\langle\mathbf{r_1},\dots,\mathbf{r_{n-1}}|\psi(\mathbf{r})|\Phi\rangle d^3\mathbf{r_1}\dots d^3\mathbf{r_{n-1}}\nonumber\\&=n\int\langle\Phi'|\mathbf{r_1},\dots,\mathbf{r_{n-1}},\mathbf{r}\rangle\langle\mathbf{r_1},\dots,\mathbf{r_{n-1}},\mathbf{r}|\Phi\rangle d^3\mathbf{r_1}\dots d^3\mathbf{r_{n-1}}\nonumber\\&=\int\langle\Phi'|\mathbf{r_1},\dots,\mathbf{r_n}\rangle\sum_i\delta(\mathbf{r}-\mathbf{r_i})\langle\mathbf{r_1},\dots,\mathbf{r_n}|\Phi\rangle d^3\mathbf{r_1}\dots d^3\mathbf{r_n}\nonumber
\end{align}
where in the last step, we exploit the fact that the state $|\mathbf{r_1},\dots,\mathbf{r_{n-1}},\mathbf{r}\rangle$ is already properly symmetrized. The result is the same as the matrix element of the usual density operator interpretation.
\end{proof}
\begin{cor}
The total number of the multi-particle state is
\begin{equation}
    N=\sum_{\mathbf{p},s}a^\dag_{\mathbf{p},s}a_{\mathbf{p},s}\label{number}
\end{equation}
\end{cor}
\begin{proof}
The total number is $N=\int\rho(\mathbf{r})d^3\mathbf{r}$ which is
$$\sum_s\int\sum_{\mathbf{p}}\frac{e^{-i\mathbf{p}\cdot\mathbf{r}}}{\sqrt{V}}a^\dag_{\mathbf{p},s}\sum_{\mathbf{p'}}\frac{e^{i\mathbf{p'}\cdot\mathbf{r}}}{\sqrt{V}}a_{\mathbf{p'},s}d^3\mathbf{r}=\sum_{s,\mathbf{p},\mathbf{p'}}a^\dag_{\mathbf{p},s}a_{\mathbf{p'},s}\int\frac{e^{i(\mathbf{p'}-\mathbf{p})\cdot\mathbf{r}}}{V}d^3\mathbf{r}=\sum_{s,\mathbf{p}}a^\dag_{\mathbf{p},s}a_{\mathbf{p},s}$$
where the integral vanishes unless $\mathbf{p}=\mathbf{p'}$.
\end{proof}
\begin{eg}[Non-interacting fermionic gas]
Consider a gas of non-interacting spin-1/2 fermions in their ground state $|\Phi_0\rangle$ filled up to the Fermi momentum $p_F$. The number of particles with momentum $\mathbf{p}$ and spin $\uparrow$ is $n_{\mathbf{p},\uparrow}=\langle\Phi_0|a^\dag_{\mathbf{p,\uparrow}}a_{\mathbf{p},\uparrow}|\Phi_0\rangle$, which is 1 and 0 if $|\mathbf{p}|\leq p_F$ and $|\mathbf{p}|>p_F$ respectively. We have $n_{\mathbf{p},\uparrow}=n_{\mathbf{p},\downarrow}$. The Fermi momentum is determined by the total number of particles:
$$N=\sum_{s,\mathbf{p}}n_{\mathbf{p,s}}=2\sum_{|\mathbf{p}|\leq p_F}1=2V\int_0^{p_F}\frac{d^3p}{(2\pi)^3}=\frac{p_F^3}{3\pi^2}V\implies p_F^3=3\pi^2n$$
The density expectation $\langle\rho(\mathbf{r})\rangle$ is uniform:
$$\sum_s\langle\Phi_0|\psi_s^\dag(\mathbf{r})\psi_s(\mathbf{r})|\Phi_0\rangle=\sum_{s,\mathbf{p},\mathbf{p'}}\frac{e^{-i\mathbf{p}\cdot\mathbf{r}}e^{i\mathbf{p'}\cdot\mathbf{r}}}{V}\langle\Phi_0|a^\dag_{\mathbf{p},s}a_{\mathbf{p'},s}|\Phi_0\rangle=\sum_{s,\mathbf{p},\mathbf{p'}}\frac{e^{-i\mathbf{p}\cdot\mathbf{r}}e^{i\mathbf{p'}\cdot\mathbf{r}}}{V}\delta_{\mathbf{p},\mathbf{p'}}n_{\mathbf{p},s}=\sum_{s,\mathbf{p}}\frac{n_{\mathbf{p},s}}{V}=n$$
\end{eg}
\begin{defi}[One-particle density matrix]
\begin{equation}
    G_s(\mathbf{r}-\mathbf{r'})=\langle\Phi_0|\psi_s^\dag(\mathbf{r})\psi_s(\mathbf{r'})|\Phi_0\rangle\label{oneparticledensity}
\end{equation}
\end{defi}
\begin{remarks}
The one-particle density matrix is the amplitude for removing a particle at $\mathbf{r'}$ with spin $s$ from the ground state and then returning to the ground state by replacing a particle with spin $s$ at point $\mathbf{r}$. We recover the single particle density $G(\mathbf{r},\mathbf{r})=\langle\rho(\mathbf{r})\rangle$.
\end{remarks}
\begin{prop}
For fermions, the one-particle density matrix is
\begin{equation}
    G_s^{\text{fermions}}(\mathbf{r}-\mathbf{r'})=\frac{3n}{2}\frac{\sin(p_F|\mathbf{r}-\mathbf{r'}|)-(p_F|\mathbf{r}-\mathbf{r'}|)\cos(p_F|\mathbf{r}-\mathbf{r'}|)}{(p_F|\mathbf{r}-\mathbf{r'}|)^3}\label{Gfermion}
\end{equation}
\end{prop}
\begin{proof}
\begin{align}
    G_s(\mathbf{r}-\mathbf{r'})&=\sum_{\mathbf{p},\mathbf{p'}}\frac{e^{-i\mathbf{p}\cdot\mathbf{r}}e^{i\mathbf{p'}\cdot\mathbf{r'}}}{V}\langle\Phi_0|a^\dag_{\mathbf{p},s}a_{\mathbf{p'},s}|\Phi_0\rangle\nonumber\\&=\sum_{\mathbf{p},\mathbf{p'}}\frac{e^{-i\mathbf{p}\cdot\mathbf{r}}e^{i\mathbf{p'}\cdot\mathbf{r'}}}{V}\delta_{\mathbf{p},\mathbf{p'}}n_{\mathbf{p},s}\nonumber\\&=\frac{1}{V}\sum_{\mathbf{p}}e^{-i\mathbf{p}\cdot(\mathbf{r}-\mathbf{r'})}n_{\mathbf{p},s}\nonumber\\&=\int_0^{p_F}e^{-i\mathbf{p}\cdot(\mathbf{r}-\mathbf{r'})}\frac{d^3\mathbf{p}}{(2\pi)^3}\nonumber\\&=\frac{1}{(2\pi)^3}\int_0^{p_F}p^2\int_{-1}^1e^{-ip|\mathbf{r}-\mathbf{r'}|\cos\theta}d\cos\theta~dp~\int_0^{2\pi}d\phi\nonumber
\end{align}
The result follows, where $p_F^3=3\pi^2n$.
\end{proof}
\begin{remarks}
$G_s^{\text{fermions}}(\mathbf{r}-\mathbf{r'})$ show an oscillatory profile with $|\mathbf{r}-\mathbf{r'}|$. 
\begin{center}
\begin{tikzpicture}
      \draw[->] (0,0) -- (10,0) node[right] {$|\mathbf{r}-\mathbf{r'}|$};
      \draw[->] (0,0) -- (0,4) node[left] {$G_s^{\text{fermions}}(\mathbf{r}-\mathbf{r'})$};
      \draw[domain=0.1:10,smooth,variable=\x,black] plot ({\x},{(sin(deg(2*\x))-2*\x*cos(deg(2*\x)))/(\x*\x*\x)});
      \draw (0,2.75) node[left]{$n/2$};
\end{tikzpicture}
\end{center}
For $\mathbf{r}=\mathbf{r'}$, $G_s^{\text{fermions}}(\mathbf{r}-\mathbf{r'})=n/2$ for particles with spin orientation $s$. For small $|\mathbf{r}-\mathbf{r'}|$, $G_s^{\text{fermions}}(\mathbf{r}-\mathbf{r'})\approx\frac{n}{2}(1-0.1(p_F|\mathbf{r}-\mathbf{r'}|)^2)$.
\end{remarks}
\begin{defi}[Pair correlation functions]
The pair correlation function is the relative probability of finding a particle at $\mathbf{r}$ if there is another at $\mathbf{r'}$. We remove a particle with spin $s$ at $\mathbf{r}$ from the system, leaving behind $N-1$ particles in the state $|\Phi'(\mathbf{r},s)\rangle=\psi_s(\mathbf{r})|\Phi_0\rangle$, and then find the density distribution of particles (with spin $s'$) in this new state:
\begin{equation}
\langle\Phi'(\mathbf{r},s)|\psi_{s'}^\dag(\mathbf{r'})\psi_{s'}(\mathbf{r'})|\Phi'(\mathbf{r},s)\rangle=\langle\Phi_0|\psi_s^\dag(\mathbf{r})\psi_{s'}^\dag(\mathbf{r'})\psi_{s'}(\mathbf{r'})\psi_s(\mathbf{r})|\Phi_0\rangle=(0.5n)^2g_{s,s'}(\mathbf{r}-\mathbf{r'})\label{paircorrelation}
\end{equation}
\end{defi}
\begin{remarks}
Equivalently, we could have also removed a particle from $\mathbf{r}$ using $\psi_s(\mathbf{r})$ and then one from $\mathbf{r'}$ using $\psi_{s'}(\mathbf{r'})$. The relative amplitude for ending up in some $N-2$ particle state $|\Phi_i''\rangle$ is $\langle\Phi_i''|\psi_{s'}(\mathbf{r'})\psi_s(\mathbf{r})|\Phi_0\rangle$. Sum over complete set of $N-2$ particle states:
$$\sum_i|\langle\Phi_i''|\psi_{s'}(\mathbf{r'})\psi_s(\mathbf{r})|\Phi_0\rangle|^2=\langle\Phi_0|\psi_s^\dag(\mathbf{r})\psi_{s'}^\dag(\mathbf{r})\sum_i|\Phi_i''\rangle\langle\Phi_i''|\psi_{s'}(\mathbf{r})\psi_s(\mathbf{r})|\Phi_0\rangle=\langle\Phi_0|\psi_s^\dag(\mathbf{r})\psi_{s'}^\dag(\mathbf{r'})\psi_{s'}(\mathbf{r'})\psi_s(\mathbf{r})|\Phi_0\rangle$$
\end{remarks}
\begin{prop}
The pair correlation for fermions of different spin is independent of the distance $|\mathbf{r}-\mathbf{r'}|$. But, for the same spin, the pair correlation is
\begin{equation}
    g_{s,s}^{\text{fermions}}(\mathbf{r}-\mathbf{r'})=1-\frac{9}{x^6}(\sin x-x\cos x)^2,\quad x=p_F|\mathbf{r}-\mathbf{r'}|\label{paircorrelationfermion}
\end{equation}
\end{prop}
\begin{proof}
plugging in the field operators, the pair correlation function is
$$\frac{n^2}{4}g_{s,s'}(\mathbf{r}-\mathbf{r'})=\frac{1}{V^2}\sum_{\mathbf{p},\mathbf{p'},\mathbf{q},\mathbf{q'}}e^{-i(\mathbf{p}-\mathbf{p'})\cdot\mathbf{r}}e^{-i(\mathbf{q}-\mathbf{q'})\cdot\mathbf{r'}}\langle\Phi_0|a^\dag_{\mathbf{p},s}a^\dag_{\mathbf{q},s'}a_{\mathbf{q'},s'}a_{\mathbf{p'},s}|\Phi_0\rangle=\sum_{\mathbf{p},\mathbf{q}}\frac{n_{\mathbf{p},s}n_{\mathbf{q},s'}}{V^2}=n_sn_{s'}$$
where the expectation vanishes unless $p'=p$, $q'=q$ if $s\neq s'$, $\implies \langle\Phi_0|a^\dag_{\mathbf{p},s}a^\dag_{\mathbf{q},s'}a_{\mathbf{q},s'}a_{\mathbf{p},s}|\Phi_0\rangle=\langle\Phi_0|a^\dag_{\mathbf{p},s}a_{\mathbf{p},s'}a^\dag_{\mathbf{q},s'}a_{\mathbf{q},s}|\Phi_0\rangle$. But, if $s=s'$, $\mathbf{p}=\mathbf{p'}$, $\mathbf{q}=\mathbf{q'}$ or $\mathbf{p}=\mathbf{q'}$, $\mathbf{q}=\mathbf{p'}$ (but if $\mathbf{p'}=\mathbf{q'}$, then the expectation vanishes unless $a^2_{\mathbf{p'},s}=0$), the expectation gives
\begin{align}
\langle\Phi_0|a^\dag_{\mathbf{p},s}a^\dag_{\mathbf{q},s}a_{\mathbf{q'},s}a_{\mathbf{p'},s}|\Phi_0\rangle&=\delta_{\mathbf{p},\mathbf{p'}}\delta_{\mathbf{q},\mathbf{q'}}\langle\Phi_0|a^\dag_{\mathbf{p},s}a^\dag_{\mathbf{q},s}a_{\mathbf{q},s}a_{\mathbf{p},s}|\Phi_0\rangle+\delta_{\mathbf{p},\mathbf{q'}}\delta_{\mathbf{q},\mathbf{p'}}\langle\Phi_0|a^\dag_{\mathbf{p},s}a^\dag_{\mathbf{q},s}a_{\mathbf{p},s}a_{\mathbf{q},s}|\Phi_0\rangle\nonumber\\&=(\delta_{\mathbf{p},\mathbf{p'}}\delta_{\mathbf{q},\mathbf{q'}}-\delta_{\mathbf{p},\mathbf{q'}}\delta_{\mathbf{q},\mathbf{p'}})\langle\Phi_0|a^\dag_{\mathbf{p},s}a_{\mathbf{p},s}a^\dag_{\mathbf{q},s}a_{\mathbf{q},s}|\Phi_0\rangle\nonumber\\&=(\delta_{\mathbf{p},\mathbf{p'}}\delta_{\mathbf{q},\mathbf{q'}}-\delta_{\mathbf{p},\mathbf{q'}}\delta_{\mathbf{q},\mathbf{p'}})n_{\mathbf{p},s}n_{\mathbf{q},s}\nonumber
\end{align}
where $\{a_{\mathbf{p},s},a_{\mathbf{q},s}^\dag\}=0$ since $\mathbf{q}\neq\mathbf{p}$, gaining a minus sign when swap $a_{\mathbf{p},s}$, $a_{\mathbf{q},s}^\dag$.
$$\implies\frac{n^2}{4}g_{s,s'}(\mathbf{r}-\mathbf{r'})=\frac{1}{V^2}\sum_{\mathbf{p},\mathbf{q}}(1-e^{-i(\mathbf{p}-\mathbf{q})\cdot(\mathbf{r}-\mathbf{r'})})n_{\mathbf{p},s}n_{\mathbf{q},s}=(n/2)^2-(G_s^{\text{fermions}}(\mathbf{r}-\mathbf{r'}))^2$$
where we had $s=s'$.
\end{proof}
\begin{remarks}\leavevmode
\begin{enumerate}
\item Exclusion principle causes large correlations in the motion of particles of the same spin. There is a substantial reduction in the probability for finding two fermions of the same spin at distances $\leq p_F^{-1}$. Almost as if fermions of the same spin repelled each other at short distance. The effective `repulsion' arises from exchange symmetry.
\begin{center}
\begin{tikzpicture}
      \draw[->] (0,0) -- (8,0) node[right] {$|\mathbf{r}-\mathbf{r'}|$};
      \draw[->] (0,0) -- (0,2) node[left] {$g_{s,s}^{\text{fermions}}(\mathbf{r}-\mathbf{r'})$};
      \draw[domain=0.5:8,smooth,variable=\x,black] plot ({\x},{1-9*(sin(deg(\x))-\x*cos(deg(\x)))^2/(\x*\x*\x)^2});
      \draw (0,1) node[left]{1};
\end{tikzpicture}
\end{center}
\item The factorization of pair correlation functions, in Proposition 4.8, is valid for a system of noninteracting fermions in any potential well. $\langle\Phi_0|\psi_s^\dag(\mathbf{r})\psi_{s'}^\dag(\mathbf{r'})\psi_{s'}(\mathbf{r'})\psi_s(\mathbf{r})|\Phi_0\rangle$ is
$$\langle\Phi_0|\psi_s^\dag(\mathbf{r})\psi_s(\mathbf{r})|\Phi_0\rangle\langle\Phi_0|\psi_{s'}^\dag(\mathbf{r'})\psi_{s'}(\mathbf{r})|\Phi_0\rangle-\langle\Phi_0|\psi_s^\dag(\mathbf{r})\psi_{s'}(\mathbf{r'})|\Phi_0\rangle\langle\Phi_0|\psi_{s'}^\dag(\mathbf{r'})\psi_s(\mathbf{r})|\Phi_0\rangle$$
\item In general, the density density correlations (two-body operator) is
\begin{align}
    \langle\rho(\mathbf{r})\rho(\mathbf{r'})\rangle&=\langle\psi^\dag(\mathbf{r})\psi(\mathbf{r})\psi^\dag(\mathbf{r'})\psi(\mathbf{r'})\rangle\nonumber\\&=\langle\psi^\dag(\mathbf{r})\psi^\dag(\mathbf{r'})\psi(\mathbf{r})\psi(\mathbf{r}')\rangle+\delta(\mathbf{r}-\mathbf{r'})\langle\psi^\dag(\mathbf{r})\psi(\mathbf{r'})\rangle\nonumber\\&=\langle :\rho(\mathbf{r})\rho(\mathbf{r'}):\rangle+\delta(\mathbf{r}-\mathbf{r'})\langle\rho(\mathbf{r})\rangle\nonumber
\end{align}
where $:~.~:$ is normal ordering - using commutation relations to shift all annihilation operators to the far right of all creation operators. $\langle\rho(\mathbf{r})\rangle$ is just spurious self-correlation (background off-set). All correlation encoded in the normal ordering $C_\rho(\mathbf{r},\mathbf{r'})=\langle:\rho(\mathbf{r})\rho(\mathbf{r'}):\rangle$. For the product states $|N_0,N_1,\dots\rangle$,
$$\langle\rho(\mathbf{r})\rho(\mathbf{r'})\rangle=\sum_{\alpha,\beta,\gamma,\delta}\phi_\alpha^*(\mathbf{r})\phi_\beta(\mathbf{r})\phi_\gamma^*(\mathbf{r'})\phi_\delta(\mathbf{r'})\langle a^\dag_\alpha a_\beta a^\dag_\gamma a_\delta\rangle$$
where by Wick's theorem, we can pair the creation/annihilation operators in the expectation, as follow:
$$\langle a_\alpha^\dag a_\beta a_\gamma^\dag a_\delta\rangle=\langle a^\dag_\alpha a_\beta\rangle\langle a^\dag_\gamma a_\delta\rangle+\langle a^\dag_\alpha a_\delta\rangle\langle a_\beta a^\dag_\gamma\rangle=\delta_{\alpha\beta}N_\alpha\delta_{\gamma\delta}N_\gamma+\delta_{\alpha\delta}N_\alpha\delta_{\beta\gamma}(1\pm N_\beta)$$
The density-density correlation function is thus
\begin{align}
\langle\rho(\mathbf{r})\rho(\mathbf{r'})\rangle&=\sum_{\alpha,\gamma}\phi_\alpha^*(\mathbf{r})\phi_\alpha(\mathbf{r})\phi_\gamma^*(\mathbf{r'})\phi_\gamma(\mathbf{r'})N_\alpha N_\gamma +\phi_\alpha^*(\mathbf{r})\phi_\gamma(\mathbf{r})\phi_\gamma^*(\mathbf{r'})\phi_\alpha(\mathbf{r'})N_\alpha(1\pm N_\gamma)\nonumber\\&=\langle\rho(\mathbf{r})\rangle\langle\rho(\mathbf{r'})\rangle+\langle\rho(\mathbf{r})\rangle\delta(\mathbf{r}-\mathbf{r'})\pm G(\mathbf{r},\mathbf{r'})\nonumber
\end{align}
$$\implies0.25n^2g(\mathbf{r}-\mathbf{r'})=\langle\rho(\mathbf{r})\rho(\mathbf{r'})\rangle-\delta(\mathbf{r}-\mathbf{r'})\langle\rho(\mathbf{r})\rangle=\langle\rho(\mathbf{r})\rangle\langle\rho(\mathbf{r'})\rangle\pm G(\mathbf{r},\mathbf{r'})G(\mathbf{r'},\mathbf{r})=n^2\pm|G(\mathbf{r},\mathbf{r'})|^2$$
\end{enumerate}
\end{remarks}
\begin{eg}[Normal ordering]
$$\langle:a^\dag_\alpha a_\beta a_\gamma a_\delta:\rangle=\langle a_\alpha^\dag a_\gamma^\dag a_\beta a_\delta\rangle=\langle a^\dag_\alpha a_\beta a_\gamma^\dag a_\delta\rangle-\delta_{\beta,\gamma}\langle a^\dag_\alpha a_\delta\rangle=\langle a^\dag_\alpha a_\delta a_\gamma^\dag a_\beta\rangle-\delta_{\gamma,\delta}\langle a^\dag_\alpha a_\beta\rangle$$
From number conservation we see that we only get a non-zero expectation value if either $\alpha=\beta$, $\gamma=\delta$ or $\alpha=\delta$, $\beta=\gamma$.
\end{eg}
\begin{eg}
The interaction energy is
\begin{eqnarray}
E^{(1)}&=&\frac{1}{2}\int V(\mathbf{r}-\mathbf{r'})\sum_{s,s'}\langle\Phi_0|\psi_s^\dag(\mathbf{r})\psi_{s'}^\dag(\mathbf{r'})\psi_{s'}(\mathbf{r'})\psi_s(\mathbf{r})|\Phi_0\rangle d^3\mathbf{r}d^3\mathbf{r'}\nonumber\\&=&\frac{1}{2}\int V(\mathbf{r}-\mathbf{r'})\sum_{s,s'}\frac{n^2}{4}g_{s,s'}(\mathbf{r}-\mathbf{r'})d^3\mathbf{r}d^3\mathbf{r'}\nonumber\\&=&\frac{1}{2}\int V(\mathbf{r}-\mathbf{r'})(n^2-\sum_sG_s(\mathbf{r}-\mathbf{r'})^2)d^3\mathbf{r}d^3\mathbf{r'}\nonumber\\&=&\frac{1}{2}NnV_0-\frac{9nN}{4}\int\frac{(\sin(p_Fr)-p_Fr\cos(p_Fr))^2}{(p_Fr)^6}V(r)d^3\mathbf{r}\nonumber
\end{eqnarray}
where $V_0=\int V(\mathbf{r})d^3\mathbf{r}$ is the average interaction of a uniform density of particles with itself, leaving out all correlation effects. This is the direct or Hartree energy term. The second term is the exchange energy term that accounts for particles of the same spin tending to stay apart. 
\end{eg}
\begin{remarks}
Sometimes, it is easier to write the interaction term as
$$V=\frac{1}{2\mathcal{V}}\sum_{\mathbf{p},\mathbf{p'},\mathbf{q'},\mathbf{q}}\sum_{s,s'}V_{\mathbf{p'}-\mathbf{p}}\delta_{\mathbf{p}+\mathbf{q},\mathbf{p'}+\mathbf{q'}}a^\dag_{\mathbf{p'},s}a^\dag_{\mathbf{q'},s'}a_{\mathbf{q},s'}a_{\mathbf{p},s}$$
where $V_{\mathbf{k}}=\int e^{-i\mathbf{k}\cdot\mathbf{r}}V(\mathbf{r})d^3\mathbf{r}$ and $\mathcal{V}$ is the volume. This is regarded as a sum over scattering processes. The momentum $\mathbf{p}+\mathbf{q}$ of the scattering process is conserved. The amplitude of scattering is $V_{\mathbf{p'}-\mathbf{p}}$.
\end{remarks}
\begin{prop}
For a system of noninteracting spinless bosons, the pair correlation function is
\begin{equation}
    \frac{n^2}{4}g(\mathbf{r}-\mathbf{r'})=n^2+\bigg|\frac{1}{V}\sum_{\mathbf{p}}n_{\mathbf{p}}e^{-i\mathbf{p}\cdot(\mathbf{r}-\mathbf{r'})}\bigg|^2-\frac{1}{V^2}\sum_\mathbf{p}n_{\mathbf{p}}(n_{\mathbf{p}}-1)\label{bosoncorrelation}
\end{equation}
\end{prop}
\begin{proof}
Let the state be $|\Phi\rangle=|n_{p_0},n_{p_1},\dots\rangle$ with density $\langle\Phi|\psi^\dag(\mathbf{r})\psi(\mathbf{r})|\Phi\rangle=\frac{1}{V}\sum_{\mathbf{p}}n_{\mathbf{p}}=n$. We have the pair correlation function to be
$$\frac{n^2}{4}g(\mathbf{r}-\mathbf{r'})=\frac{1}{V^2}\sum_{\mathbf{p},\mathbf{p'},\mathbf{q},\mathbf{q'}}e^{-i(\mathbf{p}-\mathbf{p'})\cdot\mathbf{r}}e^{-i(\mathbf{q}-\mathbf{q'})\cdot\mathbf{r'}}\langle\Phi_0|a^\dag_{\mathbf{p}}a_{\mathbf{q}}^\dag a_{\mathbf{q'}}a_{\mathbf{p'}}|\Phi_0\rangle$$
The expectation is non-vanishing if $\mathbf{p}=\mathbf{p'}$, $\mathbf{q}=\mathbf{q'}$ or $\mathbf{p}=\mathbf{q'}$, $\mathbf{q}=\mathbf{p'}$. But, these are not distinct cases if $\mathbf{p}=\mathbf{q}$. By considering all cases,
\begin{eqnarray}
    \langle\Phi_0|a^\dag_{\mathbf{p}}a^\dag_{\mathbf{q}}a_{\mathbf{q'}}a_{\mathbf{p'}}|\Phi_0\rangle&=&(1-\delta_{\mathbf{p},\mathbf{q}})\bigg(\delta_{\mathbf{p},\mathbf{p'}}\delta_{\mathbf{q},\mathbf{q'}}\langle\Phi_0|a^\dag_{\mathbf{p}}a^\dag_{\mathbf{q}}a_{\mathbf{q}}a_{\mathbf{p}}|\Phi_0\rangle+\delta_{\mathbf{p},\mathbf{q'}}\delta_{\mathbf{q},\mathbf{p'}}\langle\Phi_0|a_{\mathbf{p}}^\dag a_{\mathbf{q}}^\dag a_{\mathbf{p}}a_{\mathbf{q}}|\Phi_0\rangle\bigg)\nonumber\\&&+\delta_{\mathbf{p},\mathbf{q}}\delta_{\mathbf{p},\mathbf{p'}}\delta_{\mathbf{q},\mathbf{q'}}\langle\Phi_0|a^\dag_{\mathbf{p}}a_{\mathbf{p}}^\dag a_{\mathbf{p}}a_{\mathbf{p}}|\Phi_0\rangle\nonumber\\&=&(1-\delta_{\mathbf{p},\mathbf{q}})(\delta_{\mathbf{p},\mathbf{p'}}\delta_{\mathbf{q},\mathbf{q'}}+\delta_{\mathbf{p},\mathbf{q'}}\delta_{\mathbf{q},\mathbf{p'}})n_{\mathbf{p}}n_{\mathbf{q}}+\delta_{\mathbf{p},\mathbf{q}}\delta_{\mathbf{p},\mathbf{p'}}\delta_{\mathbf{q},\mathbf{q'}}n_{\mathbf{p}}(n_{\mathbf{p}}-1)\nonumber
\end{eqnarray}
The result follows, by contracting the Kronecker deltas with the exponentials.
\end{proof}
\begin{remarks}\leavevmode
\begin{enumerate}
\item The second term in Eqn.~\ref{bosoncorrelation} is a `$+$' instead of a `$-$' in fermions, due to bosonic statistics. The additional third term occurs since many bosons can be in the same state.
\item If all bosons are in one state $\mathbf{p_0}$, we have 
$$\frac{n^2}{4}g(\mathbf{r}-\mathbf{r'})=n^2+n^2-\frac{1}{V^2}N(N+1)=\frac{N(N-1)}{V^2}$$
This can be understood as - the relative amplitude for removing the first particle is $\frac{N}{V}$, while removing the second is $\frac{N-1}{V}$.
\item If $n_{\mathbf{p}}$ is smoothly varying, we could model it as a Gaussian with peak at $\mathbf{p_0}$: $n_\mathbf{p}=Ce^{-\alpha(\mathbf{p}-\mathbf{p_0})^2/2}$. We have $\frac{n^2}{4}g(\mathbf{r}-\mathbf{r'})$ to be
$$n^2+\bigg|\frac{1}{V}\sum_\mathbf{p}n_{\mathbf{p}}e^{-i\mathbf{p}\cdot(\mathbf{r}-\mathbf{r'})}\bigg|^2-\frac{1}{V^2}\sum_\mathbf{p} n_\mathbf{p}(n_\mathbf{p}+1)\approx n^2+\bigg|\int n_\mathbf{p}e^{-i\mathbf{p}\cdot(\mathbf{r}-\mathbf{r'})}\frac{d^3\mathbf{p}}{(2\pi)^3}\bigg|^2=n^2(1+e^{-(\mathbf{r}-\mathbf{r'})^2/\alpha})$$
where $V$ is large and we kept $n$ fixed. The effect of exchange increases the probability for two bosons to be found at small separations, resulting in bunching. $g(r)\sim 2$ as $r\rightarrow r'$, compared to $g(r)\sim 1$ as $r\rightarrow\infty$.
\end{enumerate}
\end{remarks}
\subsection{Bose-Einstein condensate}
\begin{defi}[Bose-Hubbard model]
The Bose–Hubbard model gives a description of the physics of interacting spinless bosons on a lattice.
\begin{equation}
    H=-J\sum_{\langle i,j\rangle}(a_i^\dag a_j+a_j^\dag a_i)+\frac{U}{2}\sum_{i=1}^Ln_i(n_i-1)\label{BoseHubbard}
\end{equation}
where $J$ is the hopping amplitude and $U$ is the on-site attraction.
\end{defi}
\begin{eg}
To realize this experimentally, cold atoms are subjected to standing wave of light ('potential'). If the potential is deep enough, the eigenstates are sufficiently localized wavepackets. We can thus use the `tight-binding' model where $a_i^\dag=\int\phi_i(x)\psi^\dag(x)dx$ creates a boson in a localized state orbital $\phi_i(x)$ (sometimes called a Wannier state). We have one orbital per site.
\end{eg}
\begin{remarks}
Fermi-Hubbard model is the fermion counterpart which is a `tight-binding' model to describe strongly interacting electrons, and applicable to high temperature superconductivity.
\end{remarks}
For $N/L=1$ (on average), the ground state shows a quantum phase transition as a function of $U/J$.
\begin{prop}
Off-diagonal long range order provides an `order parameter' for the transition.
\end{prop}
\begin{proof}
For $U<<J$, the particles are delocalized plane-wave basis, where $\tilde{a}_k=\frac{1}{\sqrt{L}}\sum_{j=1}^Le^{-ika_j}a_j$, subjected to periodic boundary conditions $\implies ka/(2\pi/L)$ is an integer. The Hamiltonian in the plane wave basis is effectively $H\approx\sum_k\varepsilon_k\tilde{a}^\dag_k\tilde{a}_k$ where $\varepsilon_k=-2J\cos(ka)$. We will obtain a Bose-Einstein condensate state $|\text{BEC}\rangle=\frac{1}{\sqrt{N!}}(\tilde{a}_0^\dag)^N|0\rangle$ which has off-diagonal long-range order, i.e. the single-particle density matrix $G(\mathbf{r},\mathbf{r'})=\langle\psi^\dag(\mathbf{r})\psi(\mathbf{r'})\rangle$ does not vanish for $|\mathbf{r}-\mathbf{r'}|\rightarrow\infty$. Write $G(\mathbf{r},\mathbf{r'})$ in the lattice version:
$$G(i,j)=\langle a_i^\dag a_j\rangle=\frac{1}{N!}\langle 0|(\tilde{a}_0)^Na_i^\dag a_j(\tilde{a}_0^\dag)^N|0\rangle$$
We have $[a_j,\tilde{a}_0^\dag]=\frac{1}{\sqrt{L}}\implies[a_j,(\tilde{a}_0^\dag)^N]=\frac{N}{\sqrt{L}}(\tilde{a}_0^\dag)^{N-1}$. Further, 
$$a_j(\tilde{a}_0^\dag)^N|0\rangle=(\tilde{a}_0^\dag)^Na_j|0\rangle+\frac{N}{\sqrt{L}}(\tilde{a}_0^\dag)^{N-1}|0\rangle\implies G(i,j)=\frac{N^2}{N!L}\langle 0|(\tilde{a}_0)^{N-1}(\tilde{a}_0^\dag)^{N-1}|0\rangle=\frac{N}{L}=1$$
which is independent of the site basis. For $J<<U$, we have instead a Mott insulator, where the particles are localized (one per site): $|\text{Mott}\rangle=\prod_{i=1}^La_i^\dag|0\rangle$ and hence
$$G(i,j)=\langle\text{Mott}|a_i^\dag a_j|\text{Mott}\rangle=\delta_{ij}$$
which is site-dependent, hence no off-diagonal long range order.
\end{proof}
\begin{remarks}[Bogoliubov theory]
To describe weakly interacting BEC ($U<<J$), we consider an ansatz with off-diagonal long range order. Here, we do not consider the site basis, but instead the plane wave basis:
$$H=\sum_{\mathbf{k}}\varepsilon_{\mathbf{k}}\tilde{a}_{\mathbf{k}}^\dag\tilde{a}_{\mathbf{k}}+\frac{U}{2L}\sum_{{\mathbf{k}},{\mathbf{p}},{\mathbf{q}}}\tilde{a}^\dag_{{\mathbf{k}}+{\mathbf{q}}}\tilde{a}^\dag_{{\mathbf{p}}-{\mathbf{q}}}\tilde{a}_{\mathbf{p}}\tilde{a}_{\mathbf{k}}$$
Mean-field theory predicts broken symmetry and site has off-diagonal long range order. The BEC state is a coherent state $|\text{BEC}\rangle\propto\exp(e^{i\theta}\sqrt{N}\tilde{a}_0^\dag)|0\rangle$ with indefinite particle number $\langle N\rangle=N$ and definite phase $\theta$ (broken gauge symmetry), i.e. $\tilde{a}_0^\dag|\text{BEC}\rangle=\sqrt{N}e^{i\theta}|\text{BEC}\rangle$. For macroscopic occupation $N>>1$ in the ground state, this is dominated by $\tilde{a}_0^\dag$ (zero wavevector) and we can keep interaction terms in which this appears up to second-order:
$$H\approx \sum_{\mathbf{k}}\tilde{E}_{\mathbf{k}}\tilde{a}^\dag_{\mathbf{k}}\tilde{a}_{\mathbf{k}}+\frac{U}{2}n\sum_{\mathbf{k}}\tilde{a}_{\mathbf{k}}^\dag\tilde{a}_{-{\mathbf{k}}}^\dag+\tilde{a}_{-\mathbf{k}}\tilde{a}_{\mathbf{k}}$$
This is bilinear in $\tilde{a}_{\mathbf{k}}$ and $\tilde{a}^\dag_{\mathbf{k}}$ and can be solved exactly. With a further change of variables, we can diagonalize the Hamiltonian:
$$\alpha_{\mathbf{k}}=u_{\mathbf{k}}\tilde{a}_{\mathbf{k}}-v_{\mathbf{k}}\tilde{a}^\dag_{-{\mathbf{k}}},\quad\alpha_{\mathbf{k}}^\dag=u_{\mathbf{k}}^*\tilde{a}^\dag_{\mathbf{k}}-v_{\mathbf{k}}^*\tilde{a}_{-{\mathbf{k}}},\quad[\alpha_{\mathbf{k}},\alpha^\dag_{\mathbf{k'}}]=\delta_{\mathbf{k},\mathbf{k'}}\implies|u_{\mathbf{k}}|^2-|v_{\mathbf{k}}|^2=1$$
which gives $H=\sum_{\mathbf{k}}\hbar\omega_{\mathbf{k}}\alpha_{\mathbf{k}}^\dag\alpha_{\mathbf{k}}\implies\hbar\omega_{\mathbf{k}}=\sqrt{\tilde{E}_{\mathbf{k}}(\tilde{E}_{\mathbf{k}}+2nU)}$, giving Bogoliubov modes (they are gapless Goldstone modes that arise from the broken continuous gauge symmetry) and have a linear dispersion for small $k$.
\end{remarks}
\begin{eg}[Interference of BEC]
Let us denote by $\phi_L(\mathbf{r})$ and $\phi_R(\mathbf{r})$ the ground states of two spatially separated potential wells. First, consider a state where each boson is in a superposition of both (start off from a single well and adiabatically splitting in two). Write this state
$$|\overline{N}_L,\overline{N}_R\rangle_\theta:=\frac{1}{\sqrt{N!}}\bigg[\sqrt{\frac{\overline{N}_L}{N}}e^{-i\theta/2}a_L^\dag+\sqrt{\frac{\overline{N}_R}{N}}e^{i\theta/2}a_R^\dag\bigg]^N|0\rangle$$
where $\overline{N}_{L,R}$ are the expectation values of particle number in each state. The system is evolved for some time $t$, so that the two clouds begin to overlap (free expansion, time of flight evolution by switching off the confining potentials). Ignoring interactions between the particles, the many-particle state is $|\overline{N}_L,\overline{N}_R\rangle_\theta$ with the wavefunctions $\phi_{L,R}$ evolving freely. By considering only non-empty states $\psi(\mathbf{r})=\phi_L(\mathbf{r})a_L+\phi_R(\mathbf{r})a_R$, the density operator $\rho(\mathbf{r})=\psi^\dag(\mathbf{r})\psi(\mathbf{r})$ is
$$\langle\rho(\mathbf{R},t)\rangle_\theta=\overline{N}_L|\phi_L(\mathbf{r},t)|^2+\overline{N}_R|\phi_R(\mathbf{r},t)|^2+2\sqrt{\overline{N}_L\overline{N}_R}\text{Re}[e^{i\theta}\phi_L^*(\mathbf{r},t)\phi_R(\mathbf{r},t)]$$
where the last term is $\rho_{\text{int}}(\mathbf{r},t)$ arising from the quantum interference between the two coherent subsystems. The relative phase has a real physical effect. Consider the evolution of two Gaussian wavepackets with width $R_0$ at $t=0$, separated by a distance $d>>R_0$
$$\phi_{L,R}(\mathbf{r},t)=\frac{1}{(\pi R_t^2)^{3/4}}\exp\bigg[-\frac{(\mathbf{r}\pm\mathbf{d}/2)^2(1-i\hbar t/mR_0^2)}{2R_t^2}\bigg],\quad R_t^2=R_0^2+\bigg(\frac{\hbar t}{mR_0}\bigg)^2$$
After a long period of expansion, the final density distribution is a reflection of the initial momentum distribution. Faster moving atoms fly further, so after time $t$ an atom with velocity $\mathbf{v}$ will be at position $\mathbf{r}=\mathbf{v}t$ from the centre of the trap, provided that this distance is large compared to the initial radius of the gas $R_0$. The $t\rightarrow\infty$ limit is $|\phi_{L,R}(\mathbf{r},t\rightarrow\infty)|^2\propto\exp(-(\frac{mR_0(\mathbf{r}\pm\mathbf{d}/2)}{\hbar t})^2)$, reflecting a Gaussian initial momentum distribution of width $\hbar/R_0$. The final interaction term is
$$\rho_{\text{int}}(\mathbf{r},t)=A(\mathbf{r},t)\cos\bigg(\theta+\frac{\hbar\mathbf{r}\cdot\mathbf{d}}{mR_0^2R_t^2}t\bigg),\quad A(\mathbf{r},t)=\frac{2\sqrt{\overline{N}_L\overline{N}_R}}{\pi^{3/2}R_t^3}\exp\bigg(-\frac{r^2+d^2/4}{R_t^2}\bigg)$$
The interference term consists of regularly spaced fringes with a separation at long times of $2\pi\hbar t/md$. In comparison, suppose we do the same for two condensates of fixed particle number and no phase relation, $|N_L,N_R\rangle_F=\frac{1}{\sqrt{N_L!N_R!}}(a_L^\dag)^{N_L}(a_R^\dag)^{N_R}|0\rangle\implies\langle\rho(\mathbf{r},t)\rangle_F=N_L|\phi_L(\mathbf{r},t)|^2+N_R|\phi_R(\mathbf{r},t)|^2$. For two independent condensates, in each measurement of the density, fringes are present but with a phase that varies between measurements, even if the samples are identically prepared.
\end{eg}
\newpage
\section{Density Matrices}
\subsection{Properties}
Density matrices is the most general description of a quantum system. There are two kinds of uncertainties: 'classical uncertainty' (statistical mixture) and quantum uncertainty.
\begin{defi}[Density operator]
We define the density operator $\rho:~\mathcal{H}\rightarrow\mathcal{H}$ to be
\begin{equation}
\rho=\sum_ip_i|\Psi_i\rangle\langle\Psi_i|\label{density_operator}
\end{equation}
where $p_i$ refers to some classical uncertainty in our knowledge of the system. The matrix representation of $\rho$ is the density matrix.
\end{defi}
\begin{prop}\leavevmode
\begin{enumerate}
    \item $\rho^\dag=\rho$
    \item $\Tr_{\mathcal{H}}\rho=1$
    \item $\langle\chi|\rho|\chi\rangle\geq0$ $\forall|\chi\rangle\in\mathcal{H}$
\end{enumerate}
Essentially, we summarize the above as $\rho\geq0$ (positive semi-definite operator).
\end{prop}
\begin{proof}\leavevmode
\begin{enumerate}
    \item The density operator Eqn.~\ref{density_operator} obeys $\rho^\dag=\rho$ since probabilities are real, i.e. $p_i\in\mathbb{R}$ $\forall i$.
    \item $\Tr_{\mathcal{H}}\rho=1$ since probabilities sum to 1, i.e. $\sum_ip_i=1$.
    \item $\langle\chi|p|\chi\rangle\geq0$ $\forall|\chi\rangle\in\mathcal{H}$ since probabilities are non-negative. 
\end{enumerate}
\end{proof}
\begin{prop}
The expectation of an operator $Q$ written in terms of the density matrix is
\begin{equation}
\langle Q\rangle=\Tr_{\mathcal{H}}(Q\rho)=\sum_\alpha p_\alpha\langle\psi_\alpha|Q|\psi_\alpha\rangle\label{expectation}
\end{equation}
\end{prop}
\begin{proof}
Use the definition of trace,
$$\Tr_{\mathcal{H}}(Q\rho)=\sum_n\langle q_n|\rho Q|q_n\rangle=\sum_{n,\alpha}p_\alpha\langle q_n|\psi_\alpha\rangle\langle\psi_\alpha|Q|q_n\rangle=\sum_\alpha p_\alpha\langle\psi_\alpha|Q|\psi_\alpha\rangle=\langle Q\rangle$$
where we used the cyclic property and completeness relation $\sum_\alpha|\psi_\alpha\rangle\langle\psi_\alpha|=\Id_{\mathcal{H}}$. The result follows from the cyclic property of trace.
\end{proof}
\begin{remarks}
This is also known as an ensemble average, sometimes denoted as $\overline{\langle Q\rangle}$, to distinguish from the typical expectation of a pure state.
\end{remarks}
\begin{defi}[Pure state, mixed state]
If we have perfect knowledge of the system state $|\psi\rangle$ (with probability 1), then $\rho=|\psi\rangle\langle\psi|$. This is a pure state. However, if more than one $p_\alpha>0$ for $\rho=\sum_\alpha p_\alpha|\psi_\alpha\rangle\langle\psi_\alpha|$, then the system is in a mixed state.
\end{defi}
\begin{remarks}\leavevmode
\begin{enumerate}
\item This terminology of purity of a state refers to our incomplete knowledge of the system. The state is a precise state but we are not certain which state it is.
\item The density operator for a statistical mixture can be written down directly. $\rho=\sum_ip_i|\psi_i\rangle\langle\psi_i|$ where $p_i$ is the statistical ratio for each state (being classically mixed together). 
\item To distinguish mixed and pure states, take the trace of $\rho^2$: $\Tr\rho^2<1$ and $\Tr\rho^2=1$ respectively. 
\item Two systems with the same density matrix description are physically identical, and no measurements can distinguish the two systems.
\end{enumerate}
\end{remarks}
\subsection{Two level sphere}
\begin{eg}
Suppose we have a two-state system (qubit) $\{|\uparrow\rangle,|\downarrow\rangle\}$ with $\mathcal{H}\simeq\mathbb{C}^2$. If $\rho=|\uparrow\rangle\langle\uparrow|$, system is definitely in the state $|\uparrow\rangle$, hence a pure state. If $\rho=\frac{1}{2}|\uparrow\rangle\langle\uparrow|+\frac{1}{2}|\downarrow\rangle\langle\downarrow|=\frac{1}{2}\Id$, hence equally likely to be in $|\uparrow\rangle$ or $|\downarrow\rangle$. We are not saying $|\Psi\rangle=\frac{1}{\sqrt{2}}(|\uparrow\rangle+|\downarrow\rangle)$! This is an impure or a mixed state.\\[5pt]
We could also have $\rho=\frac{1}{2}|\uparrow\rangle\langle\uparrow|+\frac{1}{2}|\uparrow_x\rangle\langle\uparrow_x|$, where $S_x|\uparrow_x\rangle=\frac{\hbar}{2}|\uparrow_x\rangle$. Note that $\langle\uparrow_x|\uparrow\rangle\neq0$. In this case, since $|\uparrow_x\rangle=\frac{1}{\sqrt{2}}(|\uparrow\rangle+|\downarrow\rangle)$, we can write this as
$$\rho=\frac{1}{4}\Id_{\mathcal{H}}+\frac{1}{2}|\uparrow\rangle\langle\uparrow|+\frac{1}{4}|\uparrow\rangle\langle\downarrow|+\frac{1}{4}|\downarrow\rangle\langle\uparrow|\implies\rho=\begin{pmatrix}3/4&1/4\\1/4&1/4\\\end{pmatrix}$$
in the $\{|\uparrow\rangle,|\downarrow\rangle\}$ basis. With this $\rho$, we have $\langle S_x\rangle=0.5\hbar=\langle S_z\rangle$ but $\langle S_y\rangle=0$ since we do not know anything of its spin in the $y$ direction.
\end{eg}
\begin{prop}\leavevmode
\begin{enumerate}
    \item For a generic two-level system, the density operator is $\rho=\frac{1}{2}(\Id+r\mathbf{n}\cdot\boldsymbol{\sigma})$, with $0\leq r\leq1$.
    \item At least one eigenvalue of $\rho$ is positive. Moreover, both eigenvalues of $\rho$ are non-negative if $\det(\rho)=0.25(1-r^2)\geq0$.
    \item If $r=1$, then the state is pure. Otherwise if $r<1$, then the state is impure.
\end{enumerate}
\end{prop}
\begin{proof}\leavevmode
\begin{enumerate}
\item The most general 2 by 2 matrix (matrix representation for $\rho$) is $\rho=c_0\Id+c_1\sigma_x+c_2\sigma_y+c_3\sigma_z$. Since $\Tr\rho=1$ for a pure state, $c_0=0.5$. Set $\mathbf{n}=c^{-1}(c_1,c_2,c_3)^T$, then we have $\rho=0.5(\Id+r\mathbf{n}\cdot\boldsymbol{\sigma})$ where $r=|\mathbf{c}|$. The eigenvalues of $\rho$ is $0.5(1\pm r)$. For it to correspond to a physical system, we require $0.5(1+r)\leq 1\implies r\leq 1$.
\item Follows from $\Tr_\mathcal{H}\rho=1$ and the given condition $\det\rho\geq0$. Alternatively, this follows from the product of the eigenvalues $0.5^2(1+r)(1-r)$.
\item If we have $r=1$, then the eigenvalues $0.5(1\pm r)$ are 1 and 0, so the state must be pure, i.e. $\exists|\uparrow_\mathbf{n}\rangle\in\mathcal{H}$ such that $\rho=\frac{1}{2}(\Id+r\mathbf{\hat{n}}\cdot\boldsymbol{\sigma})=|\uparrow_{\mathbf{n}}\rangle\langle\uparrow_{\mathbf{n}}|$. In fact, this state indeed has spin $+0.5\hbar$ along the direction $\mathbf{\hat{n}}$. If $r<1$, then both eigenvalues of $\rho$ are positive, so state is necessary impure since we cannot cast $\rho$ in the form $|\uparrow_\mathbf{n}\rangle\langle\uparrow_\mathbf{n}|$ for any direction $\mathbf{\hat{n}}$.
\end{enumerate}
\end{proof}
\begin{remarks}
For both mixed and pure states, the direction of $\mathbf{n}$ define the polarization of the state (preferential measurement direction).  $\Tr_\mathcal{H}(\sigma)=0$ and $\Tr_{\mathcal{H}}(\Id)=2$, fixing the overall factor.
\end{remarks}
\begin{eg}[Stern-Gerlach experiment]
Since $2s+1$ beams are formed for a spin-$s$ in the Stern-Gerlach experiment, the beam is an ensemble of quantum states. For $s=0.5$, the density matrix is just $\rho=0.5|\uparrow\rangle\langle\uparrow|+0.5|\downarrow\rangle\langle\downarrow|$. In the basis $\{|\uparrow\rangle,|\downarrow\rangle\}$, $\rho=0.5\Id$. 
\end{eg}
\begin{eg}
What is the difference between a 50-50 mixture of $|\uparrow\rangle$ and $|\downarrow\rangle$ states and $|\psi\rangle=\frac{1}{\sqrt{2}}(|\uparrow\rangle+e^{i\phi}|\downarrow\rangle)$. For the former, the density matrix (in the $\{|\uparrow\rangle,|\downarrow\rangle\}$ basis) is $\rho=0.5\diag(1,1)\implies\Tr\rho^2=0.5<1$, i.e. mixed state. For the latter, the density matrix is 
$$\rho=\begin{pmatrix}0.5&0.5e^{-i\phi}\\0.5e^{i\phi}& 0.5\\\end{pmatrix}\implies\rho^2=\rho$$
i.e. the state is pure. Both the $z$-components of spin (diagonal terms) give same results, hence we cannot discriminate via $\sigma_z$ measurements. The coherence is encoded in the off-diagonal elements.
\end{eg}
\newpage
\subsection{Position representation and time evolution}
The density matrix can also work for infinite-dimensional Hilbert spaces.
\begin{defi}[Position basis]
The position representation of the density matrix is
\begin{equation}
    \rho(\mathbf{r},\mathbf{r'})=\langle\mathbf{r}|\rho|\mathbf{r'}\rangle\label{posbasis}
\end{equation}
In terms of the eigenvalues and eigenstates of the density matrix, we have
\begin{equation}
    \rho(\mathbf{r},\mathbf{r'})=\sum_\alpha p_\alpha\phi_\alpha(\mathbf{r})\phi_\alpha^*(\mathbf{r'})\label{posbasis2}
\end{equation}
The position representation of the operator expectation is thus
\begin{equation}
\overline{\langle Q\rangle}=\Tr_{\mathcal{H}}[\rho Q]=\int\rho(\mathbf{r},\mathbf{r'})Q(\mathbf{r'},\mathbf{r})d\mathbf{r}d\mathbf{r'}\label{posbasis3}
\end{equation}
\end{defi}
\begin{remarks}
The diagonal element of the density matrix (position representation) $\rho(\mathbf{r},\mathbf{r})=\sum_\alpha p_\alpha|\phi_\alpha(\mathbf{r})|^2$ gives the probability of finding the particle at position $\mathbf{r}$. This generalizes to the single-particle density operator for many-body systems:
$$\rho_{\text{sp}}(\mathbf{r},\mathbf{r'})=\langle\Psi|\hat{\psi}^\dag(\mathbf{r})\hat{\psi}(\mathbf{r'})|\Psi\rangle$$
For $N>1$, this is only a partial description of the many-particle state. $\rho_{\text{sp}}(\mathbf{r},\mathbf{r'})$ is not pure even if the many-body state is pure.
\end{remarks}
\begin{eg}
The expectation value of the momentum operator in the position representation $\overline{\langle p\rangle}=\Tr[\rho p]$ is
$$\int\int\rho(\mathbf{r},\mathbf{r'})\bigg(-i\hbar\boldsymbol{\nabla_r}\delta^{(3)}(\mathbf{r}-\mathbf{r'})\bigg)d^3\mathbf{r}d^3\mathbf{r'}=-i\hbar\int\int\boldsymbol{\nabla_r}\rho(\mathbf{r},\mathbf{r'})\delta^{(3)}(\mathbf{r}-\mathbf{r'})d^3\mathbf{r}d^3\mathbf{r'}=-i\hbar\int\boldsymbol{\nabla_r}\rho(\mathbf{r},\mathbf{r'})|_{\mathbf{r'}=\mathbf{r}}d^3\mathbf{r}$$
where we performed integration by parts.
\end{eg}
\begin{prop}[von Neumann equation]
The equation of motion for the density operator is
\begin{equation}
i\hbar\frac{d\rho}{dt}=[H,\rho(t)]\label{EOMdensity}
\end{equation}
\end{prop}
\begin{proof}
In the Schr\"{o}dinger's picture, the density operator evolves as
$$\rho(t)=U(t)|\Psi_0\rangle\langle\Psi_0|U^{-1}(t)=U(t)\rho_0U^{-1}(t),\quad U(t)=e^{-iHt/\hbar}$$
$$\implies\frac{\partial\rho}{\partial t}=\frac{\partial U}{\partial t}\rho_0U^{-1}+U\rho_0\frac{\partial U^{-1}}{\partial t}=-i\frac{H}{\hbar}U\rho _0U^{-1}+U\rho_0\frac{i}{\hbar}HU^{-1}$$
But $U$ and $H$ commute, so multiplying by $i\hbar$, we then have
$$i\hbar\frac{\partial\rho}{\partial t}=HU\rho_0U^{-1}+U\rho_0HU^{-1}=U(H\rho_0-\rho_0H)U^{-1}=U[H,\rho_0]U^{-1}=[H,\rho(t)]$$
where we used the definition for Heisenberg operator.
\end{proof}
\begin{remarks}\leavevmode
\begin{enumerate}
\item The equation of motion for the density operator is the quantum analogue of the Liouville's equation $\frac{d\rho}{dt}=\{H,\rho\}$ in classical dynamics for the probability density $\rho$ on phase space.
\item Time evolution preserves the eigenvalues of $\rho$, but each state time evolves: $\rho(t)=\sum_\alpha p_\alpha|\phi_\alpha\rangle\langle\phi_\alpha|$. In general, we can have incoherent processes (damping/decoherence) and may induce transitions.
\end{enumerate}
\end{remarks}
\begin{eg}[Spin precession]
Using von Neumann equation for the 2-level system:
$$[H,\rho]=[0.5\hbar\mathbf{h}\cdot\boldsymbol{\sigma},0.5r\mathbf{n}\cdot\boldsymbol{\sigma}]=\frac{\hbar r h_in_j}{4}[\sigma_i,\sigma_j]=\frac{r\hbar}{4}h_in_j2\varepsilon_{ijk}i\sigma_k=\frac{ir\hbar}{2}(\mathbf{h}\times\mathbf{n})\cdot\boldsymbol{\sigma}\implies\frac{d\rho}{dt}=-ir\frac{\hbar}{2}\frac{i}{\hbar}(\mathbf{h}\times\mathbf{n})\cdot\boldsymbol{\sigma}$$
but, LHS is $0.5r\frac{d\mathbf{n}}{dt}\cdot\boldsymbol{\sigma}$. This gives $\frac{d\mathbf{n}}{dt}=\mathbf{h}\times\mathbf{n}$, similar to Eqn.~\ref{spinprecession}.
\end{eg}
\begin{eg}[Stern-Gerlach]
Consider spin-1/2 particles (ignore the internal structure of the atoms due to adiabatic approximation), our density matrix has the general form $\begin{pmatrix}\rho_{\uparrow\uparrow}(\mathbf{r},\mathbf{r'})&\rho_{\uparrow\downarrow}(\mathbf{r},\mathbf{r'})\\\rho_{\downarrow\uparrow}(\mathbf{r},\mathbf{r'})&\rho_{\downarrow\downarrow}(\mathbf{r},\mathbf{r'})\\\end{pmatrix}$. Suppose the magnetic field varies only in a direction transverse to the beam direction, so the longitudinal motion of the particles factors out of the problem. The Hamiltonian can then be taken to be
$$H=-\frac{\hbar^2}{2m}\partial_x^2-\mu B'x\sigma_z$$
where $B'=\partial B/\partial z$ is the magnetic field gradient and $\mu$ is the magnetic moment. $H$ is diagonal in the spin. Let $\Psi_0(x,t)$ be the wavefunction of a particle in the absence of a field gradient. The density matrix will evolve as $\rho(x,x';t=0)=\diag(0.5,0.5)\Psi_0(x,t)\Psi_0^*(x',t)$. The time evolution turns out to be very simple:
$$\Psi_{\uparrow,\downarrow}(x,t)=e^{i\theta_{\uparrow,\downarrow}(x,t)/\hbar}\Psi_0(x\mp\mu B't^2/2m,t),\quad\theta_{\uparrow,\downarrow}(x,t)=\pm\mu B'xt-\frac{(\mu B')^2t^3}{6m}$$
Apart from the phase factor, the effect of the gradient is to shift the wavepacket to the point $\pm\mu B't^2/2m$, which corresponds to the classical trajectory of a particle in a linear potential. The evolution of the density matrix is then
$$\rho(x,x';t)=\frac{1}{2}\diag(\Psi_\uparrow(x,t)\Psi_\uparrow^*(x',t),\Psi_\downarrow(x,t)\Psi_\downarrow^*(x',t))$$
The diagonal elements take on a particularly simple form $\rho_{\uparrow\uparrow}(x,x;t)=0.5|\Psi_0(x-\mu B't^2/2m,t)|^2$ and $\rho_{\downarrow\downarrow}(x,x;t)=0.5|\Psi_0(x+\mu B't^2/2m,t)|^2$.
\end{eg}
\begin{prop}
Even when we have imperfect knowledge of a system, the expected rate of change of an operator $Q$ is the appropriately weighted average of the rates of change of $Q$ for each of the possible states of the system.
\begin{equation}
i\hbar\frac{d}{dt}\Tr_{\mathcal{H}}(\rho Q)=\Tr_{\mathcal{H}}(\rho[Q,H])\label{EOM_trace}
\end{equation}
\end{prop}
\begin{proof}
Using Eqn.~\ref{EOMdensity}, $i\hbar\frac{d}{dt}\Tr_{\mathcal{H}}(\rho Q)=\Tr_{\mathcal{H}}([H,\rho]Q)=\Tr_{\mathcal{H}}(\rho[Q,H])$. Last equality follows from the cyclic property of trace.
\end{proof}
\subsection{Entropy and thermalization}
\begin{defi}[von Neumann entropy]
To quantify how pure/mixed our system (how imperfect our knowledge of the system) is, we define the von Neumann entropy 
\begin{equation}
S=-k_B\Tr_{\mathcal{H}}(\rho\ln\rho)\label{vonNeumann}
\end{equation}
\end{defi}
\begin{prop}$S=-k_B\sum_r\rho_r\ln\rho_r$
\end{prop}
\begin{proof}
Let $|\phi_r\rangle$ be the orthonormal eigenstates of $\rho=\rho^\dag$ with eigenvalue $\rho_r$. Then, the von Neumann entropy (per $k_B$) is
$$-\Tr_{\mathcal{H}}(\rho\ln\rho)=-\sum_n\langle n|\bigg(\sum_r\rho_r|\phi_r\rangle\langle\phi_r|\bigg)\bigg(\sum_{r'}\ln\rho_{r'}|\phi_{r'}\rangle\langle\phi_{r'}|\bigg)|n\rangle=-\sum_r\rho_r\ln\rho_r|\langle n|\phi_r\rangle|^2=-\sum_r\rho_r\ln\rho_r$$
since $\sum_R|\langle n|\phi_r\rangle|^2=1$, if bases $\{|n\rangle\}$ and $\{|\phi_r\rangle\}$ both properly normalized. 
\end{proof}
\begin{remarks}\leavevmode
\begin{enumerate}
\item We can see that since each $0\leq\rho_r\leq 1$, $S(\rho)\geq0$ with equality iff the system is pure. 
\item $S(\rho)$ is defined in a basis-independent way.
\end{enumerate}
\end{remarks}
\begin{prop}
The maximum entropy is $S(\rho_{max})=k_B\ln\dim(\mathcal{H})$, where $\mathcal{H}$ is the Hilbert space.
\end{prop}
\begin{proof}
Set $k_B=1$. To maximize the entropy, extremize $S(\rho)$ over $\rho$, subject to $\Tr(\rho)=1$,
$$0=\delta(-\Tr(\rho\ln\rho)+\lambda(\Tr\rho-1))=-\Tr(\delta\rho\ln\rho+\rho\rho^{-1}\delta\rho-\lambda\delta\rho)+\delta\lambda(\Tr(\rho)-1)$$
At an extremum, need both $\delta\rho$ and $\delta\lambda$ terms to vanish, which leads to $\rho=\Id e^{\lambda-1}$ and $\Tr(\rho)=1$ respectively. So the proportionality factor is fixed and $\rho_{max}=\frac{1}{\dim(\mathcal{H})}\Id_{\mathcal{H}}$. This is the density operators when all the states are equally likely. We have
$$S(\rho_{max})=\frac{1}{\dim(\mathcal{H})}\bigg[-\Tr_{\mathcal{H}}(\Id)\ln\frac{1}{\dim \mathcal{H}}\bigg]=\ln\dim(\mathcal{H})$$
where the trace of identity is the dimension of the Hilbert space.
\end{proof}
Density operators are very important in statistical physics, because we can't know the exact quantum state of $10^{23}$ atoms.
\begin{prop}[Gibbs distribution]
Suppose we know our system has fixed average energy $U=\Tr_{\mathcal{H}}(\rho H)$ where $H$ is the Hamiltonian, then the $\rho$ with maximum entropy is 
\begin{equation}
\rho_{Gibbs}=\frac{e^{-\beta H}}{\Tr_{\mathcal{H}}e^{-\beta H}}=\frac{1}{Z(\beta)}\sum_ne^{-\beta E_n}|E_n\rangle\langle E_n|\label{Gibbs}
\end{equation}
where $Z(\beta):=\Tr_{\mathcal{H}}(e^{-\beta H})=\sum_k\langle\phi_{\mathbf{k}}|e^{-\beta H}|\phi_{\mathbf{k}}\rangle=\sum_\mathbf{k}e^{-\beta E_\mathbf{k}}$ is the partition function of our system.
\end{prop}
\begin{proof}
Consider a first-order variation again
$$0=\delta[-\Tr_{\mathcal{H}}\rho\ln\rho-\lambda(\Tr_{\mathcal{H}}\rho-1)-\beta(\Tr_{\mathcal{H}}(\rho H)-U)]$$
which gives
$$0=-\Tr_{\mathcal{H}}[\delta\rho(\ln\rho+1+\beta H+\lambda)],\quad 0=\delta\lambda(\Tr_{\mathcal{H}}(\rho)-1),\quad 0=\delta\beta(\Tr_{\mathcal{H}}(\rho H)-U)$$
we obtain respectively $\rho=e^{-\beta H}e^{-\lambda-1}$, $\Tr_{\mathcal{H}}\rho=1$ and $\Tr_{\mathcal{H}}(\rho H)=U$. We have $\rho_{Gibbs}=\frac{e^{-\beta H}}{\Tr_{\mathcal{H}}[e^{-\beta H}]}$. So states of high energy are exponentially suppressed. Here, $\frac{1}{\beta}=k_BT$ plays the role of temperature.
\end{proof}
\begin{eg}
Consider $H$ to be the Hamiltonian of the simple harmonic oscillator. Evaluate the partition function
$$Z=\sum_{n=0}^\infty e^{-\beta\hbar\omega(n+0.5)}=e^{-\hbar\omega\beta/2}\sum_{n=0}^\infty e^{-\beta\hbar\omega n}=\frac{e^{-\hbar\omega\beta/2}}{1-e^{-\beta\hbar\omega}}=\frac{1}{2\sinh(\hbar\omega\beta)}\implies\frac{1}{Z}=2\sinh(\hbar\omega/2k_BT)$$
and so $\rho=2\sinh(\hbar\omega/2k_BT)\sum_{n=0}^\infty e^{-E_n/k_BT}|\phi_n\rangle\langle\phi_n|$. The energy expectation $\langle E\rangle=\Tr_{\mathcal{H}}[\rho H]$ is
$$\Tr_{\mathcal{H}}\bigg[\frac{1}{Z}\sum_{n=0}^\infty E_ne^{-E_n\beta}|\phi_n\rangle\langle\phi_n|\bigg]=\frac{1}{Z}\sum_{n=0}^\infty E_ne^{-E_n\beta}=-\frac{\partial}{\partial\beta}\log Z=\hbar\omega\bigg[\frac{1}{2}+\frac{1}{e^{\hbar\omega/k_BT}-1}\bigg]$$
\end{eg}
\begin{eg}
We have the density operator to be $\rho=\frac{1}{Z}e^{-\beta H}$ where the exact form of $H$ depends on whether it is a canonical ensemble ($H$) or a grand canonical ensemble ($H-\mu N$). The entropy is
$$S=\frac{1}{T}(E-F)=k_B(\beta\overline{\langle H\rangle}+\ln Z)=k_B\Tr(\rho\beta H+\ln Z)=-k_B\Tr[\rho\ln(e^{-\beta H}/Z)]$$
which returns Eqn.~\ref{vonNeumann} irregardless of the form of $H$.
\end{eg}
\begin{cor}
For an isolated system, the von Neumann entropy does not change.
\end{cor}
\begin{proof}
The rate of change of von Neumann entropy $\frac{1}{k_B}\frac{dS}{dt}$ is 
$$-\frac{d}{dt}\Tr_{\mathcal{H}}[\rho \ln\rho]=\frac{i}{\hbar}\Tr_{\mathcal{H}}([H,\rho](\ln\rho+1))=\frac{i}{\hbar}\Tr_{\mathcal{H}}[H\rho(\ln\rho+1)-\rho H(\ln\rho+1)]=\frac{i}{\hbar}\Tr[H\rho(\ln\rho+1)-H(\ln\rho+1)\rho]=0$$
where we used $\frac{d}{dt}\rho\ln\rho=\dot{\rho}(\ln\rho+1)$ and the cyclic property of trace.
\end{proof}
In quantum statistical mechanics, there is another reason to favour the canonical ensemble, which closely resembles the time evolution operator but `evolve' for an imaginary time $t=-i\hbar\beta$ (imaginary time evolution is not unitary). All of the techniques available to compute the time evolution of a quantum state can immediately be applied to calculate the density matrix.
\begin{cor}
The density matrix of a free particle in a box of volume $V$ is
\begin{equation}
    \rho_{\text{free}}(\mathbf{r},\mathbf{r'})=\frac{1}{V}\exp\bigg(\frac{-\pi|\mathbf{r}-\mathbf{r'}|^2}{\lambda^2_{\text{dB}}}\bigg)\label{dB}
\end{equation}
where $\lambda_{\text{dB}}=\sqrt{2\pi\beta\hbar^2/m}$ is the de Broglie wavelength.
\end{cor}
\begin{proof}
The partition function is $Z=\sum_ke^{-\beta E_k}\rightarrow\int e^{-\beta\hbar^2|\mathbf{k}|^2/2m}\frac{V}{(2\pi)^3}d^3\mathbf{k}=\frac{V}{\lambda_{\text{dB}}^3}$ where we replaced the discrete sum as an integral in phase space. In momentum basis, the density operator is $\rho=\sum_ke^{-\beta E_k}\frac{\lambda^3_{\text{dB}}}{V}|\phi_\mathbf{k}\rangle\langle\phi_{\mathbf{k}}|$. To obtain the position representation, we do a Fourier transform.
$$\langle\mathbf{r}|\rho|\mathbf{r'}\rangle=\sum_{\mathbf{k},\mathbf{k'}}\langle\mathbf{r}|\phi_{\mathbf{k}}\rangle\langle\phi_{\mathbf{k}}|\rho|\phi_{\mathbf{k'}}\rangle\langle\phi_{\mathbf{k'}}|\mathbf{r'}\rangle=\sum_{\mathbf{k},\mathbf{k'}}\frac{e^{i\mathbf{k}\cdot\mathbf{r}}}{\sqrt{V}}\delta_{\mathbf{k},\mathbf{k'}}\frac{\lambda_{\text{dB}}^3}{V}e^{-\beta E_k}\frac{e^{-i\mathbf{k'}\cdot\mathbf{r'}}}{\sqrt{V}}=\frac{e^{-\pi|\mathbf{r}-\mathbf{r'}|^2/\lambda_{\text{dB}}^2}}{V}$$
where we completed the square $e^{-(\beta/2m)(\hbar k+i\frac{r-r'}{\hbar}\frac{m}{\beta})^2}e^{-(r-r')^2m/2\hbar^2\beta}$.
Alternatively, we could have done it straight, from the free particle propagator:
$$\rho(\mathbf{r},\mathbf{r'})=\frac{1}{Z}\langle\mathbf{r}|e^{-\beta H}|\mathbf{r'}\rangle=\frac{\lambda_{\text{dB}}^3}{V}\frac{1}{\lambda_{\text{dB}}^3}\exp\bigg[-\frac{m(\mathbf{r}-\mathbf{r'})^2}{2\hbar^2\beta}\bigg]$$
\end{proof}
\subsection{Subsystems}
Suppose $\mathcal{H}_{AB}=\mathcal{H}_A\otimes\mathcal{H}_B$ and we only keep track of $A$, treating $B$ as the environment. Recall that $|\psi\rangle\in\mathcal{H}_A\otimes\mathcal{H}_B$ is called entangled if it cannot be written as a simple product $|\psi\rangle=|\phi\rangle|\chi\rangle$ for $|\phi\rangle\in\mathcal{H}_A$ and $|\chi\rangle\in\mathcal{H}_B$. Reduced density operators enables us to obtain expectation values of a subsystem's observables.
\begin{defi}[Reduced density operator]
If we were to measure some property $Q$ of $A$, represented by $Q\otimes\Id_B$ on $\mathcal{H}_{AB}$. We expect to obtain
\begin{equation}
\overline{\langle Q\rangle}=\Tr_{\mathcal{H}_{AB}}(\rho_{AB}(Q\otimes\Id_B))=\Tr_{\mathcal{H}_A}(\Tr_{\mathcal{H}_B}(\rho_{AB}(Q\otimes\Id_B)))=\Tr_{\mathcal{H}_A}(\rho_AQ)\label{reduced}
\end{equation}
where $\rho_A=\Tr_{\mathcal{H}_B}(\rho_{AB})$ is the reduced density operator for subsystem $A$. Note that the traces were performed independently.
\end{defi}
\begin{eg}
Consider two 2-state systems and the state $|\Psi\rangle=\frac{1}{\sqrt{2}}(|\uparrow\downarrow\rangle-|\downarrow\uparrow\rangle)$, then the density operator is
$$\rho_{AB}=\frac{1}{2}\bigg(|\uparrow\downarrow\rangle-|\downarrow\uparrow\rangle\bigg)\bigg(\langle\uparrow\downarrow|-\langle\downarrow\uparrow|\bigg)=\frac{1}{2}\bigg(|\uparrow\downarrow\rangle\langle\uparrow\downarrow|+|\downarrow\uparrow\rangle\langle\downarrow\uparrow|-|\downarrow\uparrow\rangle\langle\uparrow\downarrow|-|\uparrow\downarrow\rangle\langle\downarrow\uparrow|\bigg)$$
Tracing over directions of the second spins, we have
$$\rho_A=\Tr_{\mathcal{H}_B}(\rho_{AB})=\frac{1}{2}\bigg(|\downarrow\rangle_A\langle\downarrow|_A-|\uparrow\rangle_A\langle\uparrow|_A\bigg)$$
which means $\rho_A=\frac{1}{2}\Id_A$, which is a mixed state and indeed is the state of maximum entropy. Thus, if the total state is correlated/entangled, the reduced density matrix of a subsystem will be mixed. 
\end{eg}
\begin{remarks}\leavevmode
\begin{enumerate}
\item The entropy follows a subadditivity property, i.e. $S(\rho_{AB})\leq S(\rho_A)+S(\rho_B)$. The equality is saturated iff the two subsystems are unentangled so that $\rho_{AB}=\rho_A\otimes\rho_B$.
\item The lack of purity is a reflection of the entanglement between the two systems that prevents us describing the subsystem by a pure quantum state. For a density matrix, the von Neumann entropy measures the departure from a pure state. The von Neumann entropy for a reduced density matrix measures the degree of entanglement between two subsystems $S_{\text{ent}}=-\Tr[\rho_{\text{reduced}}\log\rho_{\text{reduced}}]$ is called the entanglement entropy.
\end{enumerate}
\end{remarks}


\begin{prop}
The evolution of a reduced density matrix of a composite system with initial density operator 
$$\rho_{AB}(t=0)=|\phi\rangle|\chi\rangle\langle\chi|\langle\phi|:=|\Psi_0\rangle\langle\Psi_0|$$
where subsystems A and B are unentangled, is
$$\rho_A(t)=\sum_bM_b(t)\rho_A(0)M_b^\dag(t),\quad M_b(t):~\mathcal{H}_A\rightarrow\mathcal{H}_A,~M_b(t):=\Tr_{\mathcal{H}_B}[U_{AB}(t)|\chi\rangle\langle b|]=\langle b|U_{AB}(t)|\chi\rangle$$
where $\{|b\rangle\}$ is the orthonormal basis of $\mathcal{H}_B$.
\end{prop}
\begin{proof}
We have the reduced density operator for A to be
$$\rho_{A}(t)=\Tr_{\mathcal{H}_B}[U_{AB}(t)|\Psi_0\rangle\langle\Psi_0|U_{AB}^{-1}(t)]=\sum_b\langle b|U_{AB}(t)|\Psi_0\rangle\langle\Psi_0|U_{AB}^{-1}(t)|b\rangle$$
where we perform the trace over $\{|b\rangle\}$, an orthonormal basis of $\mathcal{H}_B$. Since $U_{AB}(t)$ is unitary, we can show that the newly defined operator $M_b(t)$ obeys a completeness relation.
$$\sum_bM_b^\dag(t)M_b(t)=\sum_b\langle\chi|U_{AB}^\dag(t)|b\rangle\langle b|U_{AB}(t)|\chi\rangle=\Id_A$$
it then follows that $\rho_A(t)=\sum_bM_b(t)\rho_A(0)M_b^\dag(t)$.
\end{proof}
\begin{eg}
If $H_{AB}=H_A\otimes\Id_B+\Id_A\otimes H_B$ where our subsystems don't interact with one another. Then, the unitary is $U_{AB}(t)=U_A(t)\otimes U_B(t)$ and we have $M_b(t)=\langle b|U_B(t)|\chi\rangle U_A(t)$.
By completeness, we must have $\rho_A(t)=U_A(t)\rho_A(0)U_A^{-1}(t)$. So, if $\rho_0$ is initially both pure and unentangled (with no interaction with the environment), it will remain pure and unentangled $\forall t$.
\end{eg}
\begin{remarks}
For any density matrix of a single spin, it is a reduced density matrix of a non-unique pure state of two spins. This is called purification of a density matrix.
\end{remarks}
\begin{defi}[Decoherence]
To make a measurement, we must have some interaction between A (system) and B (apparatus). Thus during measurement, A and B do become entangled. This is known as decoherence-interaction between our system and environment/apparatus, cause A to be described by a mixed $\rho_A$.
\end{defi}
\begin{eg}
Let's suppose our system $A$ consists of a single qubit, either $|\uparrow\rangle$ or $|\downarrow\rangle$. Imagine the environment has only three possible states, $|0\rangle$, $|1\rangle$ and $|2\rangle$. An ideal measurement will change the state of the measuring apparatus without affecting the system A. Let's suppose the measurement process is described by a unitary operator $U$, that acts like
$$U|\uparrow\rangle\otimes|0\rangle=|\uparrow\rangle(\sqrt{1-p}|0\rangle+\sqrt{p}|1\rangle),\quad U|\downarrow\rangle\otimes|0\rangle=|\downarrow\rangle(\sqrt{1-p}|0\rangle+\sqrt{p}|2\rangle)$$
Suppose the system $A$ is initially described by some density matrix
$$\rho_A(0)=\begin{pmatrix}\rho_{\uparrow\uparrow}&\rho_{\uparrow\downarrow}\\\rho_{\downarrow\uparrow}&\rho_{\downarrow\downarrow}\\\end{pmatrix}$$
For this evolution, we have
$$M_0=\langle 0|U|0\rangle=\sqrt{1-p}\Id_A,\quad M_1=\langle 1|U|0\rangle=\sqrt{p}|\uparrow\rangle\langle\uparrow|,\quad M_2=\langle 2|U|0\rangle=\sqrt{p}|\downarrow\rangle\langle\downarrow|$$
where they obey $\sum_iM_i^\dag M_i=\Id_A$. Contact with our measuring apparatus thus causes $\rho_A$ to evolve as
$$U:~\begin{pmatrix}\rho_{\uparrow\uparrow}&\rho_{\uparrow\downarrow}\\\rho_{\downarrow\uparrow}&\rho_{\downarrow\downarrow}\\\end{pmatrix}\mapsto M_0(t)\rho_A(0)M_0^\dag(t)+ M_1(t)\rho_A(0)M_1^\dag(t)+ M_2(t)\rho_A(0)M_2^\dag(t)= \begin{pmatrix}\rho_{\uparrow\uparrow}&(1-p)\rho_{\uparrow\downarrow}\\(1-p)\rho_{\downarrow\uparrow}&\rho_{\downarrow\downarrow}\\\end{pmatrix}$$
i.e. suppressing the off-diagonal components. For successive evolution, we discretize an evolution of time $t$ into $N$ discrete steps $\delta t$, then the off-diagonal terms will thus be suppressed by $(1-p)^n\approx e^{-pt/\delta t}$ for large $N:=t/\delta t$. In particular, if we initially prepare A to be in the superposition
$$|\psi\rangle=a|\uparrow\rangle+b|\downarrow\rangle,\quad |a|^2+|b|^2=1$$
then eventually, $A$'s density matrix will become
$$\lim_{t\rightarrow\infty}\rho_A(t)=\begin{pmatrix}|a|^2&0\\0&|b|^2\\\end{pmatrix}$$
The matrix now only have real entries and this is called phase damping.
\end{eg}
\begin{remarks}
The measurement unitary $U$ was defined with respect to a preferred basis. This corresponds to local interactions between the apparatus and the system. 
\end{remarks}
\begin{remarks}
Although the entropy of an isolated system is unchanged under unitary evolution, the entropy of a subsystem (reduced density operator) can vary. The `arrow of time' arises from the growing entanglement with the external environment.
\end{remarks}
\begin{prop}
Consider a system weakly coupled to the reservoir, the reduced density operator of the system evolves like
\begin{equation}
    \frac{d\rho_S}{dt}=-\frac{i}{\hbar}\Tr_R[V(t),\rho_S(t_i)\otimes\rho_R(t_i)]-\frac{1}{\hbar^2}\Tr_R\int_{t_i}^t[V(t),[V(t'),\rho_S(t')\otimes\rho_R(t_i)]]dt'\label{damping}
\end{equation}
\end{prop}
\begin{proof}
Write the combined Hamiltonian as $H=H_0+V(t)$, where $H_0$ is the Hamiltonian for the decoupled subsystem and reservoir, and $V(t)$ is the weak coupling between these two quantum systems. It is convenient to work in the interaction picture (all operators are assigned time dependence due to $H_0$ and the wavefunctions evolve according to $V(t)$). The equation of motion for the entire density operator, in the interaction picture is thus given by 
$$\frac{d\rho}{dt}=-\frac{i}{\hbar}[V(t),\rho]\implies\rho(t)=\rho(t_i)-\frac{i}{\hbar}\int_{t_i}^t[V(t'),\rho(t')]dt'$$
where $t_i$ is an initial time when the coupling $V(t)$ is first turned on. At the initial time, the system and reservoir are decoupled: $\rho(t_i)=\rho_S(t_i)\otimes\rho_R(t_i)$. Assuming that the reservoir is in equilibrium at $t_i$, if the system and reservoir were to remain decoupled ($V=0$), the subsequent density operator would be $\rho_0(t)=\rho_S(t)\otimes\rho_R(t_i)$, which allows for the possibility that $\rho_S$ was not at equilibrium at $t=t_i$. Since we are interested in weak coupling, we look for a solution in which $\rho(t)=\rho_S(t)\otimes\rho_R(t_i)+\delta\rho(t)$, where $\delta\rho(t)$ is proportional to one or more powers of $V$, and must satisfy $\Tr\delta\rho=0$. Substitute $\rho(t)$ into the integral solution, and plugging it back to the equation of motion, we have to the order of $V^2$:
$$\frac{d\rho}{dt}=-\frac{i}{\hbar}[V(t),\rho_S(t_i)\otimes\rho_R(t_i)]-\frac{1}{\hbar^2}\int_{t_i}^t[V(t),[V(t'),\rho_S(t')\otimes\rho_R(t_i)]dt'$$
Taking the trace with the reservoir gives the result.
\end{proof}
\begin{eg}[Two level atom coupled to a thermal reservoir]
The Hamiltonian for the decoupled system and reservoir is $H_0=\frac{1}{2}\hbar\omega\sigma_z+\sum_\mathbf{k}\hbar\omega_\mathbf{k}a^\dag_\mathbf{k}a_\mathbf{k}$, where the first term describes a two-level atom with energy splitting $\hbar\omega$ and the second describes the reservoir of photon modes. Invoke the rotating wave approximation, the interaction is (work in the interaction picture):
$$V(t)=\hbar\sum_\mathbf{k}g_\mathbf{k}\bigg[a_\mathbf{k}^\dag\sigma_-e^{-i(\omega-\omega_\mathbf{k})t}+a_\mathbf{k}\sigma_+e^{i(\omega-\omega_\mathbf{k})t}\bigg]$$
These two terms describe processes in which the atom is de-excited ($\sigma_-$) and emits a photon ($a_\mathbf{k}^\dag$), and in which the atom is excited ($\sigma_+$) and absorbs a photon ($a_\mathbf{k}$). Plug this to Eqn.~\ref{damping}, bearing in mind that $[\sigma_\pm,\rho_R]=[a_k,\rho_S]=[a_k^\dag,\rho_S]=0$ since $\sigma_\pm$ acts on the Hilbert space of $\rho_S$ while $a$, $a^\dag$ acts on the Hilbert space of $\rho_R$. The first group of terms $\Tr_R[V(t),\rho_S\rho_R]$ give:
\begin{eqnarray}
&&-i\sum_kg_k\Tr_R(a_k^\dag\sigma_-\rho_S\rho_Re^{-i(\omega-\omega_k)t}+a_k\sigma_+\rho_S\rho_Re^{i(\omega-\omega_k)t}-\rho_S\rho_Ra_k^\dag\sigma_-e^{-i(\omega-\omega_k)t}-\rho_S\rho_Ra_k\sigma_+e^{i(\omega-\omega_k)t})\nonumber\\&=&-i\sum_kg_k(\sigma_-\rho_S\langle a_k^\dag\rangle e^{-i(\omega-\omega_k)t}+\sigma_+\rho_S\langle a_k\rangle e^{i(\omega-\omega_k)t}-\rho_S\sigma_-\langle a_k^\dag\rangle e^{-i(\omega-\omega_k)t}-\rho_S\sigma_+\langle a_k\rangle e^{i(\omega-\omega_k)t})\nonumber\\&=&-i\sum_kg_k\bigg\{[\sigma_-,\rho_S]\langle a_k^\dag\rangle e^{-i(\omega-\omega_k)t}+[\sigma_+,\rho_S]\langle a_k\rangle e^{i(\omega-\omega_k)t}\bigg\}=-i\sum_kg_k\{[\sigma_-,\rho_S]\langle a_k^\dag\rangle  e^{-i(\omega-\omega_k)t} +\text{h.c.}\}\nonumber
\end{eqnarray}
where we used $\langle O\rangle=\Tr[\rho O]$ for the reservoir, and the cyclic trace property. For the nested commutator, we have $[V,[V,\rho]]=[V,V\rho-\rho V]=V^2\rho-2V\rho V+\rho V^2$. Writing the barebones ($g_k$ and $e^{\pm i(\omega-\omega_k)t}$ to be understood from the context):
\begin{eqnarray}
[V,[V,\rho]]&=&(a^\dag a^\dag\sigma_-\sigma_++a^\dag a\sigma_-\sigma_++aa^\dag\sigma_+\sigma_-+aa\sigma_+\sigma_+)\rho_S\rho_R\nonumber\\&&-2(a^\dag\sigma_-+a\sigma_+)\rho_S\rho_R(a^\dag\sigma_-+a\sigma_+)+\rho_S\rho_R(a^\dag a^\dag\sigma_-\sigma_-+a^\dag a\sigma_-\sigma_++aa^\dag\sigma_+\sigma_-+aa\sigma_+\sigma_+)\nonumber
\end{eqnarray}
Taking the trace will give:
\begin{eqnarray}
&&\sigma_-\sigma_-\rho_S\langle a^\dag a^\dag\rangle+\sigma_-\sigma_+\rho_S\langle a^\dag a\rangle+\sigma_+\sigma_-\rho_S\langle aa^\dag\rangle+\sigma_+\sigma_+\rho_S\langle aa\rangle\nonumber\\&&-2\sigma_-\rho_S\sigma_-\langle a^\dag a^\dag\rangle-2\sigma_+\rho_S\sigma_-\langle a^\dag a\rangle-2\sigma_-\rho_S\sigma_+\langle aa^\dag\rangle-2\sigma_+\rho_S\sigma_+\langle aa\rangle\nonumber\\&&+\rho_S\sigma_-\sigma_-\langle a^\dag a^\dag\rangle+\rho_S\sigma_-\sigma_+\langle a^\dag a\rangle+\rho_S\sigma_+\sigma_-\langle aa^\dag\rangle+\rho_S\sigma_+\sigma_+\langle aa\rangle\nonumber\\&=&\langle a^\dag a^\dag\rangle(\sigma_-\sigma_-\rho_S-2\sigma_-\rho_S\sigma_-+\rho_S\sigma_-\sigma_-)+\langle aa\rangle(\sigma_+\sigma_+\rho_S-2\sigma_+\rho_S\sigma_++\rho_S\sigma_+\sigma_+)\nonumber\\&&+\langle a^\dag a\rangle(\sigma_-\sigma_+\rho_S-2\sigma_+\rho_S\sigma_-+\rho_S\sigma_-\sigma_+)+\langle a a^\dag\rangle(\sigma_+\sigma_-\rho_S-2\sigma_-\rho_S\sigma_++\rho_S\sigma_+\sigma_-)\nonumber\\&=&(\sigma_-\sigma_-\rho_S-2\sigma_-\rho_S\sigma_-+\rho_S\sigma_-\sigma_-)\langle a^\dag a^\dag)+(\sigma_-\sigma_+\rho_S-\sigma_+\rho_S\sigma_-)\langle a^\dag a\rangle+(\sigma_+\sigma_-\rho_S-\sigma_-\rho_S\sigma_+)\langle aa^\dag\rangle+\text{h.c.}\nonumber
\end{eqnarray}
Finally:
\begin{eqnarray}
\frac{d\rho_S}{dt}&=&-i\sum_\mathbf{k}g_\mathbf{k}\bigg\{\langle a^\dag_{\mathbf{k}}\rangle[\sigma_-,\rho_S(t_i)]e^{-i(\omega-\omega_\mathbf{k})t}+\text{h.c.}\bigg\}-\int^t_{t_i}\sum_{\mathbf{k},\mathbf{k'}}g_\mathbf{k}g_{\mathbf{k'}}\bigg\{\nonumber\\&&[\sigma_-\sigma_-\rho_S(t')-2\sigma_-\rho_S(t')\sigma_-+\rho_S(t')\sigma_-\sigma_-]e^{-i(\omega-\omega_\mathbf{k})t-i(\omega-\omega_{\mathbf{k'}})t'}\langle a^\dag_\mathbf{k}a^\dag_{\mathbf{k'}}\rangle+\nonumber\\&&[\sigma_-\sigma_+\rho_S(t')-\sigma_+\rho_S(t')\sigma_-]e^{-i(\omega-\omega_\mathbf{k})t+i(\omega-\omega_{\mathbf{k'}})t'}\langle a^\dag_\mathbf{k}a_{\mathbf{k'}}\rangle+\nonumber\\&&[\sigma_+\sigma_-\rho_S(t')-\sigma_-\rho_S(t')\sigma_+]e^{i(\omega-\omega_\mathbf{k})t+i(\omega-\omega_{\mathbf{k'}})t'}\langle a_\mathbf{k}a_{\mathbf{k'}}^\dag\rangle+\text{h.c}\bigg\}\nonumber
\end{eqnarray}
where the angled brackets indicate the average over the density operator of the reservoir. Choose the reservoir to be in thermal equilibrium, i.e.
$$\rho_R=\prod_\mathbf{k}\frac{1}{Z_\mathbf{k}}e^{-\hbar\omega_\mathbf{k}a^\dag_\mathbf{k}a_\mathbf{k}/k_BT},\quad Z_\mathbf{k}=\frac{1}{1-e^{-\hbar\omega_\mathbf{k}/k_BT}}$$
Since the density operator factorizes into the modes $\mathbf{k}$, each mode can be treated separately. The individual $\langle.\rangle$ terms are simplified to be:
$$\langle a_\mathbf{k}\rangle=\Tr[\rho a_\mathbf{k}]=\frac{1}{Z}\sum_{n=0}^\infty\langle n|e^{-\beta\hbar\omega n}a_\mathbf{k}|n\rangle\propto\sum_{n=0}^\infty\langle n|e^{-\beta\hbar\omega n}|n-1\rangle=0,\quad \langle a^\dag_\mathbf{k}\rangle=0$$
If $\mathbf{k'}\neq\mathbf{k}$, we can decompose the expectation of a product into products of expectations, which are individually zero. Hence,
$$\langle a^\dag_\mathbf{k}a_{\mathbf{k'}}\rangle=\delta_{\mathbf{k},\mathbf{k'}}\frac{1}{2}\sum_{n=0}^\infty\langle n|e^{-\beta\hbar\omega n}n_\mathbf{k}|n\rangle=\frac{1}{e^{\beta\hbar\omega_\mathbf{k}}-1}=\overline{n}_\mathbf{k}\delta_{\mathbf{k},\mathbf{k'}},\quad \langle a_\mathbf{k}a^\dag_{\mathbf{k'}}\rangle=(\overline{n}_{\mathbf{k}}+1)\delta_{\mathbf{k},\mathbf{k'}}$$
$$\langle a^\dag_{\mathbf{k}}a^\dag_{\mathbf{k'}}\rangle=\delta_{\mathbf{k},\mathbf{k'}}\frac{1}{2}\sum_{n=0}^\infty\langle n|e^{-\beta\hbar\omega n}a_\mathbf{k}^2|n\rangle\propto\sum_{n=0}^\infty\langle n|e^{-\beta\hbar\omega n}|n-2\rangle=0,\quad\langle a_{\mathbf{k}}a_{\mathbf{k'}}\rangle=0$$
which gives $\frac{d\rho_S}{dt}$ to be
$$-\int_{t_i}^t\sum_\mathbf{k}g_\mathbf{k}^2\bigg\{[\sigma_-\sigma_+\rho_S(t')-\sigma_+\rho_S(t')\sigma_-]\overline{n}_\mathbf{k}e^{-i(\omega-\omega_\mathbf{k})(t-t')}+[\sigma_+\sigma_-\rho_S(t')-\sigma_-\rho_S(t')\sigma_+](\overline{n}_\mathbf{k}+1)e^{i(\omega-\omega_\mathbf{k})(t-t')}+\text{h.c.}\bigg\}$$
Replace the discrete sum over modes by an integral (allowing for the two polarization states at each wavevector):
$$\sum_\mathbf{k} g_\mathbf{k}^2\overline{n}_\mathbf{k}\rightarrow2\frac{V}{(2\pi)^3}\int_0^{2\pi}d\phi\int_0^\pi\sin\theta\cos^2\theta d\theta\int_0^\infty\overline{n}_\mathbf{k}k^3dk\frac{d^2}{2\hbar c\varepsilon_0V}$$
where $V$ is the volume over which the EM modes are normalized, $g_k^2=\frac{kd^2\cos^2\theta}{2\hbar c\varepsilon_0V}$ for a dipole-allowed transition ($d$ is the dipole matrix element between the two atomic levels). By letting the rate constant to be $\Gamma=\frac{\omega^3d^2}{3\pi\varepsilon_0\hbar c^3}$, we have
$$\frac{d\rho_S(t)}{dt}=-\overline{n}_\omega\frac{\Gamma}{2}[\sigma_-\sigma_+\rho_S(t)-\sigma_+\rho_S(t)\sigma_-]-(\overline{n}_\omega+1)\frac{\Gamma}{2}[\sigma_+\sigma_-\rho_S(t)-\sigma_-\rho_S(t)\sigma_+]+\text{h.c.}$$
It follows that the respective reduced density operators satisfy:
$$\dot{\rho}_{\text{uu}}=\langle u|\dot{\rho}_S|u\rangle=-(\overline{n}_\omega+1)\Gamma\rho_{\text{uu}}+\overline{n}_\omega\Gamma\rho_{\text{dd}}$$
$$\dot{\rho}_{\text{ud}}=\dot{\rho}_{\text{du}}^*=-\bigg(\overline{n}_\omega+\frac{1}{2}\bigg)\Gamma\rho_{\text{ud}}$$
$$\dot{\rho}_{\text{dd}}=\langle d|\dot{\rho}_S|d\rangle=-\overline{n}_\omega\Gamma\rho_{\text{dd}}+(\overline{n}_\omega+1)\Gamma\rho_{\text{uu}}$$
where we have used $\sigma_+=|u\rangle\langle d|$, $\sigma_-=|d\rangle\langle u|$, $\sigma_+|u\rangle=\sigma_-|d\rangle=0$, $\sigma_+|d\rangle=|u\rangle$ and $\sigma_-|u\rangle=|d\rangle$. At long times, the derivatives will vanish, giving
$$\rho_{ud}=0=\rho_{ud}^*,\quad (\overline{n}_\omega+1)\rho_{uu}=\overline{n}_\omega\rho_{dd}$$
But, $\Tr[\rho_S]=\rho_{uu}+\rho_{dd}=1$, hence
$$(\overline{n}_\omega+1)(1-\rho_{dd})=\overline{n}_\omega\rho_{dd}\implies\rho_{uu}=\frac{\overline{n}_\omega}{2\overline{n}_\omega+1}=\frac{1}{e^{\beta\hbar\omega}+1},~\rho_{dd}=\frac{\overline{n}_\omega+1}{2\overline{n}_\omega+1}=\frac{e^{\beta\hbar\omega}}{1+e^{\beta\hbar\omega}}$$
where $\overline{n}_\omega=\frac{1}{e^{\beta\hbar\omega}-1}$.
\end{eg}
%\newpage
%\section{Lie Groups}
%\newpage
%\section{Relativistic Quantum Physics}
\end{document}